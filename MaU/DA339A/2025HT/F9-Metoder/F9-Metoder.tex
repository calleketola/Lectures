\documentclass[aspectratio=169]{beamer}

\mode<presentation>

\usepackage[utf8]{inputenc}
\usepackage[T1]{fontenc}
\usepackage{amsmath}
\usepackage{graphicx}
\usepackage{hyperref}
\usepackage{listings}
\usepackage{xcolor}
\usepackage{multicol}

\usepackage[swedish]{babel}

\usepackage{../../../Common/mau}


\lstset{style=java}


\title{CD339A---F9: Metoder}
\author{Malmö universitet}
\date{2025--09--22}

\institute{Institutionen för datavetenskap och medieteknik}


\begin{document}

\begin{frame}
    \frametitle{Innehåll}

    \textbox{
        \begin{multicols}{2}
            \tableofcontents
        \end{multicols}
    }

\end{frame}

\section{Repetition}

\begin{frame}
    \frametitle{Repetition}
    \framesubtitle{Inbyggda metoder}

    \textbox{
    Vi har tidigare använt inbyggda metoder i Java.
    Exempelvis:}

    \textbox{
    \begin{itemize}
        \item \texttt{System.out.println()}
        \item \texttt{String.length()}
        \item \texttt{Math.sqrt()}
        \item \texttt{Integer.parseInt()}
    \end{itemize}
    }

\end{frame}

\begin{frame}[fragile]
    \frametitle{Exempel}

    \textbox{En liten kodsnutt som använder Pythagoras sats}

    \begin{lstlisting}
public static void Main(String[] args){
    // a^2+b^2=c^2 (Pythaghoras sats)
    double a = 3;
    double b = 4;
    double c = Math.sqrt(Math.pow(a,2)+Math.pow(b,2));
    System.out.println("Om kateterna är " + a + " och " + b + " är hypotenusan " + c);
}   \end{lstlisting}

    \textbox{Här använder vi tre olika metoder: \lstinline!Math.sqrt()!, \lstinline!Math.pow()!, och \lstinline!System.out.println()!}

\end{frame}

\section{Om metoder}

\subsection{Vad är en metod?}

\begin{frame}
    \frametitle{Metod vs funktion}

    \textbox{Några av er har säkert stött på begreppet 
    \textit{funktion} i tidigare programmeringsspråk. 
    När vi pratar om \textit{metoder} i Java så är det 
    i praktiken samma sak som funktioner.}

    \textbox{Nu i början kommer vi att hantera våra 
    metoder som om de vore funktioner.}

\end{frame}

\begin{frame}[fragile]
    \frametitle{Vad är en metod?}

    \textbox{En metod är en samling kommandon som man kan exekvera med ett 
    anrop.}

    \begin{lstlisting}
public static double max(double a, double b){
    if (a >= b){
        double greatest = a;
    }
    else{
        double greatest = b;
    }
    return greatest;
}
    \end{lstlisting}

    \textbox{
    Nu hade vi kunnat anropa funktionen med 
    \lstinline!double c = max(5.2, 7.9)!}

\end{frame}

\subsection{Varför använda metoder?}

\begin{frame}
    \frametitle{Varför använda metoder?}

    \textbox{Det finns flera anledningar till att vi vill arbeta med metoder}

    \textbox{
        \begin{itemize}
            \item Det ökar läsbarheten (detta kan inte överskattas)
            \item Koden blir lättare att felsöka
            \item Vi behöver inte upprepa kod i onödan
            \item Lättare att återanvända kod i andra projekt
        \end{itemize}
    }

\end{frame}

\section{Hur man skapar metoder}

\subsection{Skapa och anropa metoder}

\begin{frame}[fragile]
    \frametitle{Skapa egna metoder}

    \textbox{När du skapar en metod i Java (och flera 
    andra språk) så är det ett par saker vi behöver ange.}

    \textbox{
        \begin{itemize}
            \item Tillgänglighet
            \item Returvärde
            \item Metodnamn
            \item Parametrar
        \end{itemize}
    }

    \textbox{Vi kommer att ta upp \textit{returvärde},
    \textit{metodnamn}, och \textit{parametrar} idag.
    \textit{Tillgänglighet} tar vi upp när vi pratar objekt.}

\end{frame}

\begin{frame}[fragile]
    \frametitle{Skapa egna metoder}

    %\textbox{En metod kan exempelvis se ut:}

    \begin{lstlisting}
public static char myMethod(String s){
        char firstLetter = s.charAt(0);
        return firstLetter;
}
    \end{lstlisting}

    \textbox{
    \begin{itemize}
        \item \lstinline!public! anger tillgänglighet
        \item \lstinline!static! styr vad den får göra
        \item \lstinline!char! anger vad metoder 
        \textit{returnerar}
        \item \lstinline!myMethod! metodens namn
        \item \lstinline!(String s)! parametrar som 
        metoden tar emot
        \item \lstinline!return firstLetter! skickar 
        tillbaka \lstinline!firstLetter!  
    \end{itemize}
    }

\end{frame}

\begin{frame}[fragile]
    \frametitle{Anropa metoder}

    \begin{lstlisting}
public static void main(String[] args){
    String x = "All that is gold does not glitter";
    char a = myMethod(x); // Anropa metoden, spara svaret i a
    System.out.println(a);
}
public static char myMethod(String s){
    char firstLetter = s.charAt(0); // Hitta första bokstaven
    return firstLetter; // Skicka tillbaka första bokstaven
}
    \end{lstlisting}

\end{frame}

\subsection{Parametrar och argument}

\begin{frame}[fragile]
    \frametitle{Parametrar}

    \textbox{En metods \textit{parametrar} är det som 
    metoden tar emot i metodhuvudet. I exemplet på den 
    tidigare bilden var det en sträng med namnet 
    \lstinline!s!. Metoder kan ta emot flera parametrar}

    \begin{lstlisting}
public static double max(double a, double b){
    double greatest;
    if (a > b){
        greatest = a;
    }
    else{
        greatest = b;
    }
    return greatest;
}
    \end{lstlisting}

\end{frame}

\begin{frame}[fragile]
    \frametitle{Parametrar}

    \textbox{Parametrarna kan vara olika datatyper}

    \begin{lstlisting}
public static String secondMethod(String a, char b){
    String newString = a+b; // Konkatenerar texten
    return newString; // Returnerar vår nya sträng
}
    \end{lstlisting}

\end{frame}

\begin{frame}[fragile]
    \frametitle{Argument}

    \textbox{Ett \textit{argument} är värdet vi skickar 
    in till en metod när vi anropar metoden.}

    \begin{lstlisting}
public static void main(String[] args){
    String s = secondMethod("Wow", 'a');
    System.out.println(s);
}
    \end{lstlisting}

    \textbox{Här är \lstinline!\"Wow\"! och 
    \lstinline!'a'! argument som vi skickar 
    in till metoden.}

    \textbox{argument $\rightarrow$ parameter}

\end{frame}

\subsection{Metodhuvud}

\begin{frame}[fragile]
    \frametitle{Returvärde}

    \textbox{Java är ett strikt typat språk, vilket 
    betyder att man behöver hålla koll på variablers 
    datatyper. Det gör att vi vill att en metod alltid 
    returnerar samma datatyp. Detta anger vi i metodhuvudet.}

    \begin{lstlisting}
public static String secondMethod(String a, char b){
    String newString = a+b; // Konkatenerar texten
    return newString; // Returnerar vår nya sträng
}
    \end{lstlisting}

    \textbox{Här ser vi att \lstinline!secondMethod! 
    anger att den skickar tillbaka en sträng.}

    \textbox{I vårt tidigare exmpel \lstinline!myMethod!
    returnerade vi alltid en \lstinline!double!.}

\end{frame}

\begin{frame}[fragile]
    \frametitle{Void-metoder}

    \textbox{Det finns tillfällen när en funktion inte 
    ska returnera något. Då anger man detta med 
    \lstinline!void!}

    \begin{lstlisting}
public static void greet(String name, int age){
    System.out.println("Hello "+name+", it's so cool that you are "+age+" years old!")
}
    \end{lstlisting}

\end{frame}

\begin{frame}[fragile]
    \frametitle{Statiska metoder}

    \textbox{Statiska metoder är lite speciella på så vis 
    att de är kopplade direkt till klassen och inte till 
    en instans av en klass.}

    \textbox{Vi kommer att använda statiska metoder fram 
    tills att vi börjar jobba med klasser.}

    \textbox{Läs mer i Deitel på sida 207.}

\end{frame}

\subsection{Metodkropp}

\begin{frame}[fragile]
    \frametitle{Metodkropp}

    \textbox{Med \textit{metodkropp} menar vi det som 
    kommer mellan måsvingarna.}

    \textbox{Det är i metodkroppen vi skriver all kod som 
    ska exekveras.}

    \begin{lstlisting}
public static String secondMethod(String a, char b){
    String newString = a+b; // Konkatenerar texten
    return newString; // Returnerar vår nya sträng
}
    \end{lstlisting}

    \textbox{I \lstinline!secondMethod! är det raderna 2 
    och 3 som utgör metodkroppen.}

\end{frame}

\begin{frame}
    \frametitle{Metodkropp}

    \textbox{Metodkroppen kan vara hur lång som helst.
    Men en rekommendation är att hålla den kort. Skulle 
    den bli lång kan man fundera på om man kanske ska 
    bryta ut delar av metodkroppen till egna metoder.}

    \textbox{En tumregel är att man ska kunna se hela 
    metoden utan att behöva scrolla på skärmen. 
    \textit{Detta går inte alltid att följa.}}

\end{frame}

\begin{frame}
    \frametitle{Kodningsfilosofi}

    \textbox{Försök att skapa metoder som gör en sak.}
    \textbox{Ge metoderna korta men beskrivande namn. 
    Om du inte kan ge den ett kort namn fundera på om 
    det borde vara flera metoder istället.}
    \textbox{Metoder kan anropa andra metoder}

\end{frame}

\begin{frame}[fragile]
    \frametitle{Metod som anropar andra metoder}

    \textbox{Här har vi ett exempel på en metod 
    \lstinline!checkSensors! som anropar flera andra metoder.}

    \begin{lstlisting}
public void checkSensors(){
    checkDoorSensors();
    checkStaircaseSensors();
    checkLobbySensors();
    checkOfficeSensors();
}
    \end{lstlisting}

    \textbox{Genom att dela upp \lstinline!checkSensors()!
    i flera mindre metoder så är det fortarande tydligt 
    vad metoden gör.}

    %\textbox{Det kan också vara så att vi någon annanstans 
    %vill kolla dörrarnas sensorer, då kan vi återanvända
    %den koden här.}

\end{frame}

\subsection{Return}

\begin{frame}[fragile]
    \frametitle{Return}

    \textbox{Kommandot \lstinline!return! skickar tillbaka 
    det värde som kommer direkt efter på samma rad.}

    \begin{lstlisting}
public static String secondMethod(String a, char b){
    String newString = a+b; // Konkatenerar texten
    return newString; // Returnerar vår nya sträng
}
    \end{lstlisting}

    \textbox{Här är det \lstinline!newString! som skickas 
    tillbaka till anropet.}

\end{frame}

\begin{frame}[fragile]
    \frametitle{Flera return}

    \textbox{Tidigare hade vi följande exempel:}

    \begin{lstlisting}
public static double max(double a, double b){
    double greatest;
    if (a > b){
        greatest = a;
    }
    else{
        greatest = b;
    }
    return greatest
}
    \end{lstlisting}

\end{frame}

\begin{frame}[fragile]
    \frametitle{Flera return}

    \textbox{Vi kan korta ner den koden, och göra 
    den läsligare:}

    \begin{lstlisting}
 public static double max(double a, double b){
    if (a > b){
        return a; // Returnerar a
    }
    else{
        return b; // Returnerar b
    }
}   
    \end{lstlisting}

\end{frame}

\begin{frame}[fragile]
    \frametitle{Flera return}

    \textbox{När programmet når en return så avslutas 
    metoden och värdet efter returneras.}

    \textbox{Koden kommer alltså inte att fortsätta och 
    returnera något mer senare.}

\end{frame}

\section{Mer om metoder}

\subsection{Överlagring av metoder}

\begin{frame}
    \frametitle{Överlagring av metoder}

    \textbox{Det händer att man har en metod där man 
    vill att den ska kunna ta emot olika parametrar.
    Kanske för att man vill kunna ta emot olika mängder av 
    parametrar eller för att man vill göra lite olika 
    saker beroende på vad som skickas in.}

    \textbox{Det är möjligt att skapa flera metoder med 
    samma namn, förutsatt att de alla returnerar samma 
    datatyp, och att de har olika parameterlistor.}

\end{frame}

\begin{frame}[fragile]
    \frametitle{Överlagring av metoder}
    \framesubtitle{Exempel}

    \begin{lstlisting}
public double max(double a, double b){
    if (a>=b){return a;} // Vertical space is prime real estate
    else{return b;}
}
public double max(double a, double b, double c){
    if (a>=b && a>=c){return a;}
    else if (b>=a && b>=c){return b;}
    else{return c;}
}
    \end{lstlisting}

    \textbox{Nu gör kommandona \lstinline!max(1.0,3.0)! och 
    \lstinline!max(5.0,3.0,7.0)! lite olika saker}

\end{frame}

\begin{frame}[fragile]
    \frametitle{Överlagring av metoder}
    \framesubtitle{Exempel}

    \begin{lstlisting}
public double max(double a, double b){
    if (a>=b){return a;} // Vertical space is prime real estate
    else{return b;}
}
public double max(double a, double b, double c){
    return max(a, max(b, c));
}
    \end{lstlisting}

    \textbox{Hur fungerar det här?}

\end{frame}

\begin{frame}[fragile]
    \frametitle{Överlagring av metoder}
    \framesubtitle{Exempel}

    \begin{lstlisting}
public double max(double a, double b){
    if (a>=b){return a;} // Vertical space is prime real estate
    else{return b;}
}
    \end{lstlisting}

    \textbox{Vad händer om man skickar in två 
    \lstinline!int! till \lstinline!max()! nu?}

\end{frame}

\begin{frame}[fragile]
    \frametitle{Överlagring av metoder}
    \framesubtitle{Exempel}

    \textbox{Skulle vi ändra på parametrarna till 
    \lstinline!int! istället och skicka in två 
    \lstinline!double! istället, vad händer då?}

    \begin{lstlisting}
public double max(int a, int b){
    if (a>=b){return a;} // Vertical space is prime real estate
    else{return b;}
}
    \end{lstlisting}

\end{frame}

\subsection{Scope och call stack}

\begin{frame}
    \frametitle{Scope}
    
    \textbox{Som regel kan en metod inte komma åt 
    variabler som skapats utanför metoden (detta är 
    särskilt sant när man använder \lstinline!static!)}

    \textbox{På samma sätt kan andra metoder inte komma 
    åt variabler som instansierats i en annan metod.}

    \textbox{Detta skapar en sorts inkapsling och 
    decentralisering som gör att varje metod bara gör sin
    grej.}

    \textbox{Man säger ibland att variablerna är 
    \textit{lokala}.}

\end{frame}

\begin{frame}
    \frametitle{Call stack}

    \textbox{När du anropar en funktion så skapas något 
    som kallas för en \textit{stack frame}.}

    \textbox{En \textit{stack frame} håller koll på 
    initialiserade variabler och vart de pekar i minnet.}

    \textbox{När en metod avslutas så returnas eventuellt 
    return värde och sen förstörs \textit{stack framen} 
    och alla lokala variabler glöms bort ur minnet.}

\end{frame}

\begin{frame}
    \frametitle{Call stack}
    \framesubtitle{Lite djupare}

    \textbox{För att förstå hur \textit{call stacken} 
    fungerar behöver man förstå hur \textit{stackar} 
    fungerar.}

    \textbox{En stack är en datastruktur som lagrar 
    data enligt modellen \textit{Last-in-First-out (LiFo)}.}

    \textbox{Vi kommer inte att diskutera stackar på djupet
    i den här kursen.}

\end{frame}

\begin{frame}
    \frametitle{Call stack}
    \framesubtitle{Lite djupare}

    \textbox{Förenklat kan vi jämföra en stack med en hög 
    med disk. Du lägger på ny disk överst i högen, och 
    diskar den översta disken först.}

    \textbox{Likt diskhögen hamnar den senast anropade 
    metoden överst. När den avslutas så kastas det som är 
    kvar i \textit{framen} och programmet fortsätter med 
    den näst översta \textit{framen} i stacken.}

    \textbox{När den sista \textit{framen} är avslutad så 
    är vår \lstinline!main!-metod avslutad och programmet 
    stängs ner.}

\end{frame}

\section{Sammanfattning}

\subsection{Sammanfattning}

\begin{frame}
    \frametitle{Sammanfattning}

    \textbox{
        \begin{itemize}
            \item \textbf{metoder ökar läsligheten av 
            vår kod}
            \item metoder ska vara specifika och namnade 
            på ett beskrivande sätt
            \item metoder returnerar alltid samma typ av 
            värden
            \item en metod kan vara void och inte returnera 
            något alls
            \item i metodhuvudet anger vi tillgängligheten, 
            datatyp som returneras, metodnamn och parametrar
            \item vi anropar metoder genom att skriva 
            metodnamnet och skicka med argument
            \item en metod avslutas när den når 
            \lstinline!return!   
        \end{itemize}
    }

\end{frame}

\begin{frame}
    \frametitle{Sammanfattning}

    \textbox{
        \begin{itemize}
            \item \textbf{metoder ökar läsligheten av 
            vår kod} (om någon skulle glömt det)
            \item metoder kan \textit{överskugga} varandra
            om de har samma namn men olika parameter-listor
            \item när vi anropar en metod skapas en ny 
            \textit{stack frame}
            \item en \textit{stack frame} håller reda på alla 
            lokala variabler
        \end{itemize}
    }

\end{frame}

\subsection{Rekommenderad läsning}

\begin{frame}
    \frametitle{Rekommenderad läsning}

    \textbox{Deitel \& Deitel, kapitel 5, sidorna 204--236}

    \textbox{
        \begin{itemize}
            \item \url{https://docs.oracle.com/javase/tutorial/java/javaOO/methods.html}
            \item \url{https://docs.oracle.com/javase/tutorial/java/javaOO/arguments.html}
            \item \url{https://docs.oracle.com/javase/tutorial/java/javaOO/returnvalue.html}
        \end{itemize}
    }
    \textbox{Cornelia Funke, \textit{The Colour of Revenge}}
\end{frame}

\end{document}