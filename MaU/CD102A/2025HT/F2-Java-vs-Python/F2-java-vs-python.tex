\documentclass[aspectratio=169]{beamer}

\mode<presentation>

\usepackage[utf8]{inputenc}
\usepackage[T1]{fontenc}
\usepackage{amsmath}
\usepackage{graphicx}
\usepackage{hyperref}
\usepackage{listings}
\usepackage{xcolor}
\usepackage{multicol}

\usepackage[swedish]{babel}

\usepackage{../../../Common/mau}


\lstset{style=python}


\title{DA102A---F1: Java vs. Python}
\author{Malmö universitet}
\date{2025--??--??}

\institute{Institutionen för datavetenskap och medieteknik}


\begin{document}

\begin{frame}
    \frametitle{Innehåll}

    \begin{multicols}{2}
        \tableofcontents
    \end{multicols}

\end{frame}

\section{Repetition}

\begin{frame}
    \frametitle{Repetition}
    \framesubtitle{Saker vi har gjort i DA100A}

    \textbox{
        \begin{itemize}
            \item Datatyper
            \item Villkorssatser
            \item Loopar
            \item Funktioner
            \item Listor
        \end{itemize}
    }

\end{frame}

\begin{frame}[fragile]
    \frametitle{Repetition}
    \framesubtitle{Datatyper}

    \textbox{Vi har använt datatyperna}
    \textbox{
        \begin{itemize}
            \item \lstinline!int! för heltal
            \item \lstinline!float! för flyttal
            \item \lstinline!bool! för sant/falskt
            \item \lstinline!str! för text
            \item \lstinline!list! för listor (arrayer)
        \end{itemize}
    }

\end{frame}

\begin{frame}[fragile]
    \frametitle{Input/output}

    \begin{lstlisting}
name = input("Your name: ")
age = int(input("Your age: ")) # Vi castar till en int

print("Hello,", name, "you are", age, "years old.")
    \end{lstlisting}

\end{frame}

\begin{frame}[fragile]
   \frametitle{Repetition}
   \framesubtitle{Villkorssatser}

    \textbox{Vi har skapat villkorsatser så att programmet 
    kan agera utifrån olika situationer}
    
    \begin{lstlisting}
name = input("May I have your name, please?" )
if len(name) > 6:
    print("Your name is too long, I do not want it.")
elif len(name) < 3:
    print("Your name is too short, not fit for me.")
else:
    print("Thank you, your name is now mine.")
    \end{lstlisting}

\end{frame}

\begin{frame}[fragile]
    \frametitle{Repetition}
    \framesubtitle{Loopar}

    \textbox{Vi har gjort upprepningar med \lstinline!while! 
    och med \lstinline!for!}

    \begin{lstlisting}
i = 0
while i < 10:
    print(i)
    \end{lstlisting}

    \begin{lstlisting}
for i in range(10):
    print(i)
    \end{lstlisting}

\end{frame}

\begin{frame}[fragile]
    \frametitle{Repetition}
    \framesubtitle{Funktioner}

    \textbox{Vi har skapat och anropat funktioner}

    \begin{lstlisting}
def my_function(par1,par2):
    # Do stuff
    res = par1+par2
    return res
a = 1
b = 2
print(my_function(a,b))
    \end{lstlisting}

\end{frame}

\begin{frame}[fragile]
    \frametitle{Repetition}
    \framesubtitle{Listor}

    \textbox{Vi har skapat listor med flera element}

    \begin{lstlisting}
hobbits = ["Sam", "Frodo", "Merry", "Pippin", "Fatty"]
print(hobbits[0])
    \end{lstlisting}

    \begin{lstlisting}
matrix = [[1,2,3],[4,5,6],[7,8,9]]
for row in range(len(matrix)):
    for col in range(len(matrix[row])):
        print(matrix[row][col])
    \end{lstlisting}

\end{frame}

\section{Java}

\begin{frame}
    \frametitle{Vad är Java?}

    \textbox{Java är ett kompilerat språk (där Python till 
    är ett interpreterat språk). Det innebär att Java behöver 
    översätta källkoden från text till byte-kod innan vi kan 
    köra programmet. (Python gör detta medan programmet körs)}
    \textbox{Du kan inte köra ditt program för ens efter att 
    det är kompilerat.}
    \textbox{Under kompileringen så görs en felsökning av koden
    och hittar den fel så misslyckas kompileringen.}
    \textbox{Java är ett strikt objekt-orienterat språk.}

\end{frame}

\begin{frame}
    \frametitle{Var används Java?}

    \textbox{En gång i tiden var Java sexigt.}
    \textbox{Telefon-appar skrevs i Java.}
    \textbox{Många system kör Java i sin \textit{back-end} 
    (särskilt äldre system)}
    \textbox{\textit{Minecraft} är kodat i Java}
    
\end{frame}

\section{Skillnader}

\subsection{Allmänt}

\begin{frame}
    \frametitle{Skillnader}
    \framesubtitle{Semikolon}

    \textbox{En viktig skillnad mellan Python och Java är 
    att man i Java \textbf{måste} markera när en sats är slut.}

    \textbox{I Python görs detta automagiskt när du byter rad.}

    \textbox{I Java (och C) behöver du markera när en sats är 
    över med ett semikolon '';''}

\end{frame}

\begin{frame}[fragile]
    \frametitle{Skillnader}
    \framesubtitle{Block}

    \textbox{I Python markerar du block-tillhörighet med 
    indragningar. (Exempelvis if-satser och loopar)\\
    I Java markerar du block-tillhörighet med 
    \textit{måsvingar} \lstinline!\{\}!}

    \textbox{Symbolerna \lstinline!\{\}! har flera olika namn:
        \begin{itemize}
            \item Måsvingar (detta är det vedertagna namnet)
            \item Krullparanteser
            \item På engelska: \textit{curly brackets}
            \item På danska: \textit{tuborg-klammer} 
        \end{itemize}
    }


\end{frame}

\begin{frame}[fragile]
    \frametitle{Skillnader}
    \framesubtitle{Kommentarer}

    \begin{lstlisting}[style=python]
# Detta är en kommentar
"""
Detta är en lång kommentar
"""
    \end{lstlisting}

    \begin{lstlisting}[style=java]
// Detta är en kommentar
/*
Detta är en lång kommentar
*/
    \end{lstlisting}

\end{frame}

\begin{frame}[fragile]
    \frametitle{Skillnader}
    \framesubtitle{Namnkonventioner}

    \textbox{I Python använder man ofta understreck i 
    variabelnamn om det är ett namn som innehåller 
    flera ord. Exempelvis: \lstinline!my_long_name!}

    \textbox{I Java använder man istället \textit{mixed case}.
    Exempelvis: \lstinline[style=java]!myLongName!}

\end{frame}

\subsection{Datatyper}

\begin{frame}[fragile]
    \frametitle{Datatyper}

    \textbox{I Java deklarerar man datatypen när man initialiserar
    en variabel.}

    \begin{lstlisting}[style=java]
int a = 5; // heltal
float b = 5.5f // flyttal (32 bitar (7 decimaler)) notera f
double c = 5.5 // decimaltal (64 bitar (15/16 decimaler))
boolean d = true; // notera liten bokstav
String e = "hej"; // notera citattecken och stor bokstav
char f = 'a'; // notera apostrof
    \end{lstlisting}

\end{frame}

\begin{frame}[fragile]
    \frametitle{Datatyper}

    \textbox{Java är ett strikt typat språk. Det innebär att 
    en variabel har en bestämd datatyp redan när den skapas, 
    och den variabeln kan inte innehålla några andra datatyper.}

    \begin{lstlisting}[style=python]
# Detta är okej Python-kod
a = 5
a = 5.5
a = "Najs"
    \end{lstlisting}

    \begin{lstlisting}[style=java]
// Detta är inte okej Java-kod
int a = 5;
a = 5.5; // Error
double a = 5.5; // Error
    \end{lstlisting}

\end{frame}

\begin{frame}[fragile]
    \frametitle{Datatyper}
    \framesubtitle{Float och double}

    \textbox{I Python använder man \lstinline!float! för 
    decimaltal. Medan Java använder både 
    \lstinline[style=java]!float! och 
    \lstinline[style=java]!double!}

    \textbox{En \lstinline!float! i Java är sparad med 32 bitar,
    medan en \lstinline[style=java]!double! är sparad i 64 bitar. Detta 
    betyder att du får en högre precision med \lstinline[style=java]!double!}

    \textbox{Som regel kommer vi att använda 
    \lstinline[style=java]!double! för att lagra decimaltal.}

\end{frame}

\subsection{Input/output}

\begin{frame}[fragile]
    \frametitle{Input/output}

    \begin{lstlisting}[style=python]
# Python
name = input("Name: ")
age = int(input("Age: "))
print(name, age)
    \end{lstlisting}

    \begin{lstlisting}[style=java]
// Java
Scanner input = new Scanner(System.in); // Nödvändig rad
System.out.println("Name: ");
String name = input.nextLine(); // Ta emot sträng
System.out.println("Age: ")
int age = input.nextInt(); // Ta emot heltal
System.out.println(name + " " + age);
    \end{lstlisting}

\end{frame}

\begin{frame}[fragile]
    \frametitle{Input/output}

    \begin{lstlisting}[style=java]
Scanner input = new Scanner(System.in); // Skapar en läsare
String text = input.nextLine(); // Tar emot en sträng
int integer = input.nextInt(); // Tar emot en int
double number = input.nextDouble(); // Tar emot en double
    \end{lstlisting}

\end{frame}

\subsection{Villkorssatser}

\begin{frame}[fragile]
    \frametitle{Villkorssatser}
    \framesubtitle{Python}

    \begin{lstlisting}[style=python]
if a == True and b == False:
    pass
elif c == True or d == True:
    pass
else:
    pass
    \end{lstlisting}

\end{frame}

\begin{frame}[fragile]
    \frametitle{Villkorssatser}
    \framesubtitle{Java}

    \begin{lstlisting}[style=java]
if (a == true && b == false){ // Notera parenteserna
    // Notera att && betyder och
}
else if (c == true || d == true){ // Vi skriver ut elif
    // Notera att || betyder eller
}
else{
    // Notera alla { } som markerar block
}
    \end{lstlisting}

\end{frame}

\subsection{Loopar}

\begin{frame}[fragile]
    \frametitle{Loopar}
    \framesubtitle{While}

    \begin{lstlisting}[style=python]
# En while-loop i Python
i = 0
while i < 10:
    print(i)
    i += 1
    \end{lstlisting}

    \begin{lstlisting}[style=java]
// En while-loop i Java
int i = 0;
while (i < 10){ // Notera återigen parenteserna runt villkoret
    System.out.println(i);
    i++; // Detta motsvarar i += 1
}
    \end{lstlisting}

\end{frame}

\begin{frame}[fragile]
    \frametitle{Loopar}
    \framesubtitle{For}

    \begin{lstlisting}[style=python]
# For-loop i Python
for i in range(0,10,1): # Start, Stop, Steg
    print(i)
    \end{lstlisting}

    \begin{lstlisting}[style=java]
// For-loop i Java
for (int i=0; i<10; i++){ // Start, Stop, Steg
    System.out.println(i); // Notera återigen alla parenteser
}
    \end{lstlisting}

\end{frame}

\subsection{Listor/arrayer}

\begin{frame}[fragile]
    \frametitle{Listor och arrayer}

    \begin{lstlisting}[style=python]
# Lista i Python
myList = [3,1,4,1,5]
print(myList[0])
    \end{lstlisting}

    \begin{lstlisting}[style=java]
// Array i Java
int[] myArray = {3,1,4,1,5};
System.out.println(myArray[0]);
    \end{lstlisting}

\end{frame}

\subsection{Funktioner \& Metoder}

\begin{frame}[fragile]
    \frametitle{Funktioner}
    \framesubtitle{Funktioner i Python}

    \begin{lstlisting}[style=python]
# Funktion i Python
def my_function(par1, par2):
    result = par1+par2
    return result
    \end{lstlisting}

\end{frame}

\begin{frame}[fragile]
    \frametitle{Funktioner}
    \framesubtitle{Metoder i Java}

    \textbox{Java är helt objektorienterat och har istället 
    för funktioner \textit{metoder} (dessa finns i Python 
    också---har du läst kursen Programmering 2 känner du till 
    dem)}

    \textbox{Metoder är i princip samma sak som funktioner,
    men kopplade till objekt.}

    \begin{lstlisting}[style=java]
// Metod i Java
public double myMethod(double par1, double par2){
    double result = par1+par2;
    return result;
}
    \end{lstlisting}

\end{frame}

\begin{frame}[fragile]
    \frametitle{Metoder}
    \framesubtitle{Metodhuvudet}

    \begin{lstlisting}[style=java]
public double myMethod(double par1, double par2){}
    \end{lstlisting}

    \textbox{\lstinline[style=java]!public! anger tillgänglighet}
    \textbox{\lstinline[style=java]!double! (det första) anger returtyp}
    \textbox{Parametrarna behöver också ha bestämda datatyper}
    \textbox{En metod som inte ska returnera något värde har 
    retur-typen \lstinline[style=java]!void!}

\end{frame}

\begin{frame}[fragile]
    \frametitle{Metoder}
    \framesubtitle{Överskuggning}

    \textbox{I Python kunde man ange default-parametrar\\
    I Java behöver vi \textit{överskugga} metoderna}

    \begin{lstlisting}[style=java]
public double myMethod(){
    return 0.0;
}
public double myMethod(double a){
    return 5*a;
}
public double myMethod(double a, double b){
    return 5*(a+b);
}
    \end{lstlisting}

\end{frame}

\section{Klasser}

\subsection{Kort om klasser}

\begin{frame}
    \frametitle{Klasser}
    \framesubtitle{Väldigt kort}

    \textbox{I Java måste all kod ligga i klasser.}

    \textbox{Varje fil får bara innehålla en klass.}

    \textbox{Klassnamnet måste stämma överens med filnamnet.}

\end{frame}

\begin{frame}[fragile]
    \frametitle{Klasser}
    \framesubtitle{Skapa en klass}

    \textbox{Klassnamnet bör börja med stor bokstav.}

    \begin{lstlisting}[style=java]
public class MyJavaFile{
    public static void main(String[] args){
        // Här börjar programmet
    }
}
    \end{lstlisting}

    \textbox{Den klass du kommer att köra ditt program ifrån
    behöver innehålla \lstinline[style=java]!public static void main(String[] args)!}

\end{frame}

\subsection{Kompilera Java-kod}

\begin{frame}[fragile]
    \frametitle{Kompilera Java-kod}

    \textbox{För att kompilera Java-kod i terminalen skriver du:}

    \begin{lstlisting}
javac MyJavaFile.java
    \end{lstlisting}

    \textbox{För att sedan köra ett Java-program i terminalen 
    skriver du:}

    \begin{lstlisting}
java MyJavaFile
    \end{lstlisting}

    \textbox{Notera att du i det första fallet har med 
    \texttt{.java}, medan i det andra fallet inte har det.}

\end{frame}


\section{Sammanfattning}

\begin{frame}
    \frametitle{Sammanfattning}
    
    \textbox{Java och Python är mer lika än olika.}

    \textbox{I Java avslutar vi varje rad med ;}

    \textbox{I Java behöver vi deklarera datatyper}

    \textbox{I Java markerar man block med \{\}}

    \textbox{I Java har vi parenteser runt villkor}

    \textbox{I Java har arrayer bestämda storlekar och markeras med \{\}}

\end{frame}

\begin{frame}
    \frametitle{Rekommenderad läsning}

    \textbox{Deitel \& Deitel, Kapitel 2--6, sidorna 89--297}

    \textbox{Oscar Wilde, \textit{The Picture of Dorian Gray}}

\end{frame}




\end{document}