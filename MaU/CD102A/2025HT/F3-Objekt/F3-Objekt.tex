\documentclass[aspectratio=169]{beamer}

\mode<presentation>

\usepackage[utf8]{inputenc}
\usepackage[T1]{fontenc}
\usepackage{amsmath}
\usepackage{graphicx}
\usepackage{hyperref}
\usepackage{listings}
\usepackage{xcolor}
\usepackage{multicol}

\usepackage[swedish]{babel}

\usepackage{../../../Common/mau}

\lstset{style=java}


\title{DA102A---F?: Klasser}
\author{Malmö universitet}
\date{2025--??--??}

\institute{Institutionen för datavetenskap och medieteknik}


\begin{document}

\begin{frame}
    \frametitle{Innehåll}

    \tableofcontents

\end{frame}

\section{Objekt}

\subsection{Objekt sen tidigare}

% Visa objekt vi stött på tidigare
% Exempelvis Scanner
% Allt vi kan sätta en punkt bakom

\subsection{Vad är ett objekt?}


\subsection{Varför använda objekt?}


\subsection{}

\section{Klassdiagram}

\subsection{Attribut}

\subsection{Operationer}

\subsection{Associationer}

\section{Sammanfattning}

\begin{frame}
    \frametitle{Rekommenderad läsning}

    \textbox{Dietel \& Dietel, Kapitel 7, 298--}

    \textbox{Hemsida}

    \textbox{Tove Janson, \textit{Det osynliga barnet}}

\end{frame}


\end{document}