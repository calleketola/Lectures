\documentclass[aspectratio=169]{beamer}

\mode<presentation>

\usepackage[utf8]{inputenc}
\usepackage[T1]{fontenc}
\usepackage{amsmath}
\usepackage{graphicx}
\usepackage{hyperref}
\usepackage{listings}
\usepackage{xcolor}
\usepackage{multicol}

\usepackage[swedish]{babel}

\usepackage{../../../Common/mau}

\lstset{style=java}


\title{DA102A---F3: Klasser}
\author{Malmö universitet}
\date{2025--??--??}

\institute{Institutionen för datavetenskap och medieteknik}


\begin{document}

\begin{frame}
    \frametitle{Innehåll}

    \tableofcontents

\end{frame}

\section{Objekt}

\subsection{Objekt sen tidigare}

% Visa objekt vi stött på tidigare
% Exempelvis Scanner
% Allt vi kan sätta en punkt bakom
\begin{frame}
    \frametitle{Objekt sen tidigare}

    \textbox{Vi har redan använt objekt tidigare, exempelvis \texttt{Scanner}}

    \textbox{Andra objekt vi använt är \texttt{System.out} och eventuellt \texttt{Math}}

    \textbox{I Java kan man lite förenklat säga att allt du kan sätta en punkt efter är ett objekt.}

\end{frame}

\subsection{Vad är ett objekt?}

\begin{frame}
    \frametitle{Vad är ett objekt?}

    \textbox{Ett objekt är en \textit{instans} av en klass.}

\end{frame}

\begin{frame}
    \frametitle{Vad är en klass?}

    \textbox{En klass är en mall för att skapa objekt.}

\end{frame}

\begin{frame}[fragile]
    \frametitle{Ett exempel}

    \textbox{Tänk dig en cykel. Begreppet cykel är en klass.}

    \textbox{En specifik cykel är ett objekt. Till exempel min cykel.}

    \textbox<2>{Alla cyklar har vissa egenskaper, exempelvis färg och antal växlar. Dessa kallas \textit{attribut}}

    \textbox<2>{Alla cyklar har också saker de kan göra, exempelvis rulla och bromsa. Dessa kallas \textit{operationer}}

\end{frame}

\subsection{Varför använda objekt?}

\begin{frame}
    \frametitle{Varför använda objekt?}

    \textbox{Objekt hjälper oss att strukturera och organisera vår kod.}

    \textbox{Objekt kan användas för att modellera verkliga saker och koncept.}

\end{frame}

\begin{frame}
    \frametitle{Varför använda objekt?}
    \framesubtitle{Exempel}

    \textbox{Tänk att vi ska skapa ett program för att hantera cyklar.}

    \textbox{Vi hade då haft två alternativ:}

    \textbox{\begin{enumerate}
        \item Skapa variabler och funktioner för varje cykel
        \item Skapa en klass \texttt{Cykel} och sedan skapa objekt av den klassen.
    \end{enumerate}}

\end{frame}

\section{Skapa klasser}

\subsection{Skapa klasser}

\begin{frame}
    \frametitle{Skapa klasser}

    \textbox{I Java behöver varje klass ligga i sin egen fil.}

    \textbox{Varje fil måste heta samma sak som klassen.}

    \textbox{(Se tillbaka till tidigare genomgång.)}

\end{frame}

\begin{frame}[fragile]
    \frametitle{Skapa klasser}

    \begin{lstlisting}
public class Bike{
    private String colour; // Attribut
    private int gears; // Attribut
    public Bike(String colour, int gears){ // Konstruktor
        this.colour = colour;
        this.gears = gears;
    }
    public void roll(){ // Operation/metod
        System.out.println("Cykeln rullar");
    }
    public void brake(){ // Operation/metod
        System.out.println("Cykeln bromsar.");
    }
}
    \end{lstlisting}

\end{frame}

\begin{frame}[fragile]
    \frametitle{Skapa objekt}

    \begin{lstlisting}
Bike myBike = new Bike("svart", 3); // Notera Bike i början
Bike yourBike = new Bike("röd", 7); // Notera new

myBike.roll();
yourBike.brake();
    \end{lstlisting}

    \textbox{I koden ovan \textit{instansierar} vi 
    två objekt av klassen \texttt{Bike}}

    \textbox{Sen använder vi objekten genom att 
    anropa deras operationer.}

\end{frame}

\subsection{Konstruktor}

\begin{frame}[fragile]
    \frametitle{Konstruktor}

    \begin{lstlisting}
public class Bike{
    public Bike(String colour, int gears){
        // Code goes here
    }
}
    \end{lstlisting}

    \textbox{En konstruktor är en metod som anropas när ett objekt skapas.}

    \textbox{Konstruktorn har samma namn som klassen och saknar returtyp.}

\end{frame}

\begin{frame}[fragile]
    \frametitle{Konstruktor}

    \begin{lstlisting}
public Bike(String colour, int gears){ // Vår konstruktor
    this.colour = colour;
    this.gears = gears;
}
    \end{lstlisting}

    \textbox{I konstuktorn brukar vi sätta värden på objektets attribut.}

    \textbox{Nyckelordet \lstinline!this! refererar till det aktuella objektet.}

    \textbox{Om vi inte använder \lstinline!this! skulle vi referera till parametrarna. 
    Ha för vana att alltid använda \lstinline!this! när du refererar till attribut.}

\end{frame}

\begin{frame}[fragile]
    \frametitle{Konstruktor}
    \framesubtitle{Flera konstruktorer}

    \begin{lstlisting}
public class Bike{
    public Bike(){
        this.colour = "svart";
        this.gears = 1;
    }
    public Bike(String colour, int gears){
        this.colour = colour;
        this.gears = gears;
    }
}
    \end{lstlisting}

    \textbox{Precis som metoder kan vi ha flera konstruktorer med olika parametrar.}

\end{frame}

\subsection{Attribut}

\begin{frame}[fragile]
    \frametitle{Attribut}

    \textbox{En variabel som är deklarerad i en klass kallas för ett attribut.}

    \textbox{Attribut brukar deklareras överst i klassen.}

    \begin{lstlisting}
public class Bike{
    private String colour; // Attribut
    private int gears; // Attribut
    public Bike(String colour, int gears){
        // Code goes here
    }
}
    \end{lstlisting}

\end{frame}

\begin{frame}
    \frametitle{Attribut}
    \framesubtitle{Åtkomstmodifierare}

    \textbox{Kommandot \texttt{private} och \texttt{public} kallas åtkomstidentifierare.}

    \textbox{Attribut bör nästan alltid vara \texttt{private}.}

    \textbox{Ett \texttt{private} attribut kan endast nås inifrån klassen.}

\end{frame}

\subsection{Operationer}

\begin{frame}[fragile]
    \frametitle{Operationer}

    \textbox{En metod som är deklarerad i en klass kallas för en operation.}

    \begin{lstlisting}
public class Bike{
    public void roll(){ // Operation/metod
        System.out.println("Cykeln rullar");
    }
    public void brake(){ // Operation/metod
        System.out.println("Cykeln bromsar.");
    }
}
    \end{lstlisting}

\end{frame}

\begin{frame}[fragile]
    \frametitle{Operationer}

    \begin{lstlisting}
Bike myBike = new Bike("svart", 3);
myBike.roll();
    \end{lstlisting}

    \textbox{Precis som att vi skriver \lstinline!input.nextInt()! 
    för att anropa metoden \texttt{nextInt} skriver vi \lstinline!myBike.roll()!
    för att anropa metoden \texttt{roll}.}

\end{frame}

\begin{frame}
    \frametitle{Operationer}

    \textbox{I övrigt fungerar operationer precis som vanliga metoder.}

\end{frame}

\begin{frame}[fragile]
    \frametitle{Operationer}
    \framesubtitle{Static}

    \textbox{Vi har tidigare använt statiska metoder. Exempelvis \texttt{Math.sqrt()}}

    \textbox{Jämför vi dem med vanliga metoder så är skillnaden att en 
    statisk metod inte behöver komma från ett initialiserat objekt.}

    \textbox{När vi använder \texttt{Math.sqrt()} så är det inte kopplat till något objekt.}

    \textbox{När vi har använt \texttt{input.nextInt()} så är det kopplat till objektet 
    \texttt{input} av klassen \texttt{Scanner}.}

\end{frame}

\subsection{Åtkomstidentifierare}

\begin{frame}
    \frametitle{Åtkomstidentifierare}
    \framesubtitle{public och private}

    \textbox{Ett attribut eller en operation med 
    \texttt{public} åtkomstidentifierare kan nås 
    från andra klasser.}

    \textbox{Ett attribut eller en operation med 
    \texttt{private} åtkomstidentifierare kan endast 
    nås inom den egna klassen.}


\end{frame}

\begin{frame}[fragile]
    \frametitle{Åtkomstidentifierare}
    \framesubtitle{Exempel}

    \begin{lstlisting}
public class Bike{
    public String colour; // Publikt attribut
    private inte gears; // Privat attribut

    public Bike(String colour, int gears){
        this.colour = colour;
        this.gears = gears;
    }
}
    \end{lstlisting}

\end{frame}

\begin{frame}[fragile]
    \frametitle{Åtkomstidentifierare}
    \framesubtitle{Exempel}

    \begin{lstlisting}
// Detta är en annan klass
Bike myBike = new Bike("svart", 3);
System.out.println(myBike.colour); // Fungerar
System.out.println(myBike.gears); // Fungerar inte
    \end{lstlisting}

\end{frame}

\begin{frame}[fragile]
    \frametitle{Åtkomstidentifierare}
    \framesubtitle{Get och Set}

    \textbox{En vanlig standard är att göra attribut
    \texttt{private} och sedan skapa \texttt{public}
    \texttt{get}- och \texttt{set}-metoder för att 
    läsa och ändra värdet på attributen.}

    \begin{lstlisting}
public String getColour(){
    return this.colour;
}
public void setColour(String colour){
    this.colour = colour;
}
    \end{lstlisting}

\end{frame}

\section{Sammanfattning}

\begin{frame}[fragile]
    \frametitle{Sammanfattning}

    \textbox{Klasser är mallar för att skapa \textit{objekt}.}

    \textbox{Man kan skapa flera objekt av samma klass.}

    \textbox{Objekt har \textit{attribut} (variabler) och 
    \textit{operationer} (metoder).}

    \textbox{Attribut och operationer kan vara 
    \texttt{public} eller \texttt{private}.}

    \textbox{Konstruktorn är en särskild metod som 
    anropas när ett objekt skapas.}

    \textbox{\lstinline!this! refererar till det 
    aktuella objektet.}

\end{frame}

\begin{frame}
    \frametitle{Sammanfattning}

    \textbox{Objekt hjälper oss att strukturera och 
    organisera vår kod.}

    \textbox{Objekt kan användas för att modellera 
    verkliga saker och koncept.}

    \textbox{Vi kommer att kolla på hur vi kan använda 
    flera klasser i samma program senare.}

    \textbox{Det mesta som hjälper oss att läsa och 
    strukturera vår kod är bra. Vi lägger mer tid på 
    att titta på vår kod än att faktiskt skriva den.}

\end{frame}

\begin{frame}
    \frametitle{Rekommenderad läsning}

    \textbox{Dietel \& Dietel, Kapitel 7, 298--338}

    \textbox{Tove Janson, \textit{Det osynliga barnet}}

\end{frame}


\end{document}