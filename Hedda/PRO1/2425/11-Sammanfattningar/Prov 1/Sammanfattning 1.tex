\documentclass[aspectratio=169]{beamer}

\mode<presentation>

\usepackage[utf8]{inputenc}
\usepackage[T1]{fontenc}	%makes å,ä,ö etc. proper symbols
\usepackage{amsmath}
\usepackage{graphicx}
\usepackage{xcolor}
\usepackage{listings}
\usepackage{multicol}
\usepackage{hyperref}
\usepackage[swedish]{babel}

\definecolor{LundaGroen}{RGB}{00,68,71}
\definecolor{StabilaLila}{RGB}{85,19,78}
\definecolor{VarmOrange}{RGB}{237,104,63}

\definecolor{MagnoliaRosa}{RGB}{251,214,209}
\definecolor{LundaHimmel}{RGB}{204,225,225}
\definecolor{LundaLjus}{RGB}{255,242,191}

\usefonttheme{serif}
\usetheme{malmoe}
\setbeamercolor{palette primary}{bg=LundaHimmel, fg=StabilaLila}
\setbeamercolor{palette quaternary}{bg=LundaGroen, fg=MagnoliaRosa}
\setbeamercolor{background canvas}{bg=LundaLjus}
\setbeamercolor{structure}{fg=LundaGroen}

\usepackage[many]{tcolorbox}

\newtcolorbox{cross}{blank,breakable,parbox=false,
  overlay={\draw[red,line width=5pt] (interior.south west)--(interior.north east);
    \draw[red,line width=5pt] (interior.north west)--(interior.south east);}}
    
\newcommand{\code}[1]{\colorbox{white}{\lstinline{#1}}}



\lstset{language=Python} 
\lstset{%language=[LaTeX]Tex,%C++,
    morekeywords={PassOptionsToPackage,selectlanguage,True,False},
    keywordstyle=\color{blue},%\bfseries,
    basicstyle=\small\ttfamily,
    %identifierstyle=\color{NavyBlue},
    commentstyle=\color{red}\ttfamily,
    stringstyle=\color{VarmOrange},
    numbers=left,%
    numberstyle=\scriptsize,%\tiny
    stepnumber=1,
    numbersep=8pt,
    showstringspaces=false,
    breaklines=true,
    %frameround=ftff,
    frame=single,
    belowcaptionskip=.75\baselineskip,
	tabsize=4,
	backgroundcolor=\color{white}
    %frame=L
}


\begin{document}

\lstset{literate=
  {á}{{\'a}}1 {é}{{\'e}}1 {í}{{\'i}}1 {ó}{{\'o}}1 {ú}{{\'u}}1
  {Á}{{\'A}}1 {É}{{\'E}}1 {Í}{{\'I}}1 {Ó}{{\'O}}1 {Ú}{{\'U}}1
  {à}{{\`a}}1 {è}{{\`e}}1 {ì}{{\`i}}1 {ò}{{\`o}}1 {ù}{{\`u}}1
  {À}{{\`A}}1 {È}{{\'E}}1 {Ì}{{\`I}}1 {Ò}{{\`O}}1 {Ù}{{\`U}}1
  {ä}{{\"a}}1 {ë}{{\"e}}1 {ï}{{\"i}}1 {ö}{{\"o}}1 {ü}{{\"u}}1
  {Ä}{{\"A}}1 {Ë}{{\"E}}1 {Ï}{{\"I}}1 {Ö}{{\"O}}1 {Ü}{{\"U}}1
  {â}{{\^a}}1 {ê}{{\^e}}1 {î}{{\^i}}1 {ô}{{\^o}}1 {û}{{\^u}}1
  {Â}{{\^A}}1 {Ê}{{\^E}}1 {Î}{{\^I}}1 {Ô}{{\^O}}1 {Û}{{\^U}}1
  {œ}{{\oe}}1 {Œ}{{\OE}}1 {æ}{{\ae}}1 {Æ}{{\AE}}1 {ß}{{\ss}}1
  {ű}{{\H{u}}}1 {Ű}{{\H{U}}}1 {ő}{{\H{o}}}1 {Ő}{{\H{O}}}1
  {ç}{{\c c}}1 {Ç}{{\c C}}1 {ø}{{\o}}1 {å}{{\r a}}1 {Å}{{\r A}}1
  {€}{{\euro}}1 {£}{{\pounds}}1 {«}{{\guillemotleft}}1
  {»}{{\guillemotright}}1 {ñ}{{\~n}}1 {Ñ}{{\~N}}1 {¿}{{?`}}1
}

\AtBeginSection[ ]
{
\begin{frame}{Innehåll}
    	\tableofcontents[currentsection]
\end{frame}
}

\title{Sammanfattning 1}
\date{2023/24}
\author{Programmering 1}

\maketitle

\section{Input/Output}

\subsection{Input}

\begin{frame}[fragile]
	\frametitle{Input}
	
	\begin{itemize}
		\item När du vill ta emot information av användaren använder du dig av kommandot \code{input()}
	\end{itemize}
	
	\begin{lstlisting}
svar = input("Text som skrivs ut ")
	\end{lstlisting}
	
	\begin{itemize}
		\item Tänk på att Python alltid bara tar emot text-strängar
	\end{itemize}
	
\end{frame}

\subsection{Output}

\begin{frame}[fragile]
	\frametitle{Output}
	
	\begin{itemize}
		\item När du vill skriva ut något så använder du dig av kommandot \code{print()}
	\end{itemize}
	
	\begin{lstlisting}
print("Text i en sträng")
print("Flera", "strängar", "på", "rad")
a = "tja"
print(a)
print(1337)
	\end{lstlisting}
	
\end{frame}

\begin{frame}[fragile]
	\frametitle{Output}
	
	\begin{itemize}
		\item Det finns två saker vi ofta vill justera med \code{print}
		\item Det ena är att vad som ska hända när utskriften är klar
		\item Det andra är vad som ska hända på komma
	\end{itemize}
	
	\begin{lstlisting}
print("Cool text", end="Wow")
print("Hej", "på", "dig", sep=":)")
	\end{lstlisting}
	
\end{frame}

\section{Datatyper}

\begin{frame}
	\frametitle{Datatyper}
	
	\begin{itemize}
		\item Vi kan och vill göra olika saker med olika typer av data
		\item Därför sparar Python variabler som olika datatyper
		\item \code{str} (för \textit{string}) är text
		\item \code{int} (för \textit{integer}) är heltal
		\item \code{float} är flyttal (decimaltal)
		\item \code{bool} (för \textit{boolean}) är sant/falskt
	\end{itemize}
	
\end{frame}

\begin{frame}[fragile]
	\frametitle{Datatyper}
	
	\begin{itemize}
		\item Man kan konvertera (typecaste:a) mellan datatyper
		\item Att konvertera mellan \code{int} och \code{float} är lätt
		\item Att konvertera från \code{int} eller \code{float} till \code{str} är lätt
		\item Att konvertera från \code{str} till \code{int} eller \code{float} kan gå fel
	\end{itemize}
	
	\begin{lstlisting}
a = int(3.14) # Blir 3
b = float(5) # Blir 5.0
c = str(2) # Blir "2"
d = str(3.0) # Blir "3.0"
e = int("4") # Blir 4
f = float("5.7") # Blir 5.7
g = int("1.1") # Kraschar  
	\end{lstlisting}
	
\end{frame}

\section{If-satser}

\begin{frame}[fragile]
	\frametitle{If-satser}
	
	\begin{itemize}
		\item När ett villkor behöver vara uppfyllt för att något ska hända använder du \code{if}-satser
	\end{itemize}
	
	\begin{lstlisting}
svar = input("Hej på dig! ")
if svar.lower() == "hejsan":
    print("Du är trevlig")
elif svar.lower() == "hallå":
    print("Så kan man ju hälsa")
elif svar == svar.upper():
    print("Så du skriker!")
else:
    print("Jahopp")
	\end{lstlisting}
	
\end{frame}

\begin{frame}
	\frametitle{Jämförelseoperatorer}
	
	\begin{itemize}
		\item Än så länge har vi stött på sex jämförelseoperatorer
		\item \code{==} ''lika med''
		\item \code{!=} ''inte lika med''
		\item \code{>} ''större än''
		\item \code{<} ''mindre än''
		\item \code{>=} ''större eller lika med''
		\item \code{<=} ''mindre eller lika med''
	\end{itemize}
	
\end{frame}

\section{Loopar}

\subsection{While}

\begin{frame}[fragile]
	\frametitle{While}

	\begin{itemize}
		\item När man vill upprepa något är \code{while}-loopar bra
		\item \code{while}-loopen fungerar som en \code{if}-sats men kör så länge som villkoret är uppfyllt
	\end{itemize}
	
	\begin{lstlisting}
svar = input('Skriv något långt')
while len(svar) < 10:
    svar = input("Ditt svar var inte långt. Skriv något nytt. ")
	\end{lstlisting}
	
	\begin{itemize}
		\item Det är väldigt viktigt att du uppdaterar variabeln i villkoret
	\end{itemize}

\end{frame}

\subsection{For}

\begin{frame}[fragile]
	\frametitle{For}
	
	\begin{itemize}
		\item När du vet hur många gånger du ska upprepa något är \code{for}-loopen bra
	\end{itemize}
	
	\begin{lstlisting}
summa = 0
for i in range(12):
    summa += i
	\end{lstlisting}
	
\end{frame}

\begin{frame}[fragile]
	\frametitle{range}
	
	\begin{itemize}
		\item Det finns tre sätt att använda \code{range}
	\end{itemize}
	
	\begin{enumerate}
		\item \code{range(slut)} börjar på noll slutar på slut-1
		\item \code{range(start, slut)} börjar på start och slutar på slut-1
		\item \code{range(start, slut, steg)} börjar på start, slutar på slut-1 och ökar i med steg varje gång
	\end{enumerate}
	
\end{frame}

\section{Kommandon}

\begin{frame}
	\frametitle{}
\end{frame}

\end{document}