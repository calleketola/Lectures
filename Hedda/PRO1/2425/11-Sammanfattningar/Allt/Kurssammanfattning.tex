\documentclass[aspectratio=169]{beamer}

\mode<presentation>

\usepackage[utf8]{inputenc}
\usepackage[T1]{fontenc}	%makes å,ä,ö etc. proper symbols
\usepackage{amsmath}
\usepackage{graphicx}
\usepackage{xcolor}
\usepackage{listings}
\usepackage{multicol}
\usepackage{hyperref}
\usepackage[swedish]{babel}

\definecolor{LundaGroen}{RGB}{00,68,71}
\definecolor{StabilaLila}{RGB}{85,19,78}
\definecolor{VarmOrange}{RGB}{237,104,63}

\definecolor{MagnoliaRosa}{RGB}{251,214,209}
\definecolor{LundaHimmel}{RGB}{204,225,225}
\definecolor{LundaLjus}{RGB}{255,242,191}

\usefonttheme{serif}
\usetheme{malmoe}
\setbeamercolor{palette primary}{bg=LundaHimmel, fg=StabilaLila}
\setbeamercolor{palette quaternary}{bg=LundaGroen, fg=MagnoliaRosa}
\setbeamercolor{background canvas}{bg=LundaLjus}
\setbeamercolor{structure}{fg=LundaGroen}

\usepackage[many]{tcolorbox}

\newtcolorbox{cross}{blank,breakable,parbox=false,
  overlay={\draw[red,line width=5pt] (interior.south west)--(interior.north east);
    \draw[red,line width=5pt] (interior.north west)--(interior.south east);}}
    
\newcommand{\code}[1]{\colorbox{white}{\lstinline{#1}}}



\lstset{language=Python} 
\lstset{%language=[LaTeX]Tex,%C++,
    morekeywords={PassOptionsToPackage,selectlanguage,True,False},
    keywordstyle=\color{blue},%\bfseries,
    basicstyle=\small\ttfamily,
    %identifierstyle=\color{NavyBlue},
    commentstyle=\color{red}\ttfamily,
    stringstyle=\color{VarmOrange},
    numbers=left,%
    numberstyle=\scriptsize,%\tiny
    stepnumber=1,
    numbersep=8pt,
    showstringspaces=false,
    breaklines=true,
    %frameround=ftff,
    frame=single,
    belowcaptionskip=.75\baselineskip,
	tabsize=4,
	backgroundcolor=\color{white}
    %frame=L
}


\begin{document}

\lstset{literate=
  {á}{{\'a}}1 {é}{{\'e}}1 {í}{{\'i}}1 {ó}{{\'o}}1 {ú}{{\'u}}1
  {Á}{{\'A}}1 {É}{{\'E}}1 {Í}{{\'I}}1 {Ó}{{\'O}}1 {Ú}{{\'U}}1
  {à}{{\`a}}1 {è}{{\`e}}1 {ì}{{\`i}}1 {ò}{{\`o}}1 {ù}{{\`u}}1
  {À}{{\`A}}1 {È}{{\'E}}1 {Ì}{{\`I}}1 {Ò}{{\`O}}1 {Ù}{{\`U}}1
  {ä}{{\"a}}1 {ë}{{\"e}}1 {ï}{{\"i}}1 {ö}{{\"o}}1 {ü}{{\"u}}1
  {Ä}{{\"A}}1 {Ë}{{\"E}}1 {Ï}{{\"I}}1 {Ö}{{\"O}}1 {Ü}{{\"U}}1
  {â}{{\^a}}1 {ê}{{\^e}}1 {î}{{\^i}}1 {ô}{{\^o}}1 {û}{{\^u}}1
  {Â}{{\^A}}1 {Ê}{{\^E}}1 {Î}{{\^I}}1 {Ô}{{\^O}}1 {Û}{{\^U}}1
  {œ}{{\oe}}1 {Œ}{{\OE}}1 {æ}{{\ae}}1 {Æ}{{\AE}}1 {ß}{{\ss}}1
  {ű}{{\H{u}}}1 {Ű}{{\H{U}}}1 {ő}{{\H{o}}}1 {Ő}{{\H{O}}}1
  {ç}{{\c c}}1 {Ç}{{\c C}}1 {ø}{{\o}}1 {å}{{\r a}}1 {Å}{{\r A}}1
  {€}{{\euro}}1 {£}{{\pounds}}1 {«}{{\guillemotleft}}1
  {»}{{\guillemotright}}1 {ñ}{{\~n}}1 {Ñ}{{\~N}}1 {¿}{{?`}}1
}

\AtBeginSection[ ]
{
\begin{frame}{Innehåll}
    	\tableofcontents[currentsection]
\end{frame}
}

\title{Sammanfattning}
\date{ht 22}
\author{Programmering 1}

\maketitle

\section{Datatyper och variabler}

\begin{frame}[fragile]
	\frametitle{Datatyper och variabler}
	
	\begin{itemize}
		\item \code{int}
		\item \code{float}
		\item \code{bool}
		\item \code{str}/\code{string}
		\item \code{list}
	\end{itemize}

\end{frame}

\begin{frame}[fragile]
	\frametitle{Datatyper}
	\framesubtitle{int}
	
	När man ska sparaett heltal så sparas det som en \code{int}
	
	\begin{lstlisting}
a = 5 # Ett heltal
a = a + 3
a += 2
print(a) # Vad skrivs ut?
	\end{lstlisting}
	
\end{frame}

\begin{frame}[fragile]
	\frametitle{Datatyper}
	\framesubtitle{float}
	
	\code{float} lagrar lite förenklat decimaltal.
	
	\begin{lstlisting}
b = 3.5
b = b*2
print(b) # Vad blir utskriften?
	\end{lstlisting}
	
\end{frame}

\begin{frame}[fragile]
	\frametitle{Datatyper}
	\framesubtitle{bool}
	
	Datatypen \code{bool} lagrar antingen \code{True} eller \code{False}. Detta är användbart vid if-satser och while-loopar.
	
	\begin{lstlisting}
do = True

while do:
    # Kör kod
	\end{lstlisting}
	
\end{frame}

\begin{frame}[fragile]
	\frametitle{Datatyper}
	\framesubtitle{String}
	
	Alla variabler med text är strängar.
	
	\begin{lstlisting}
a = "Hejsan"
a = a + " " + "Svejsan"
print(a) # Vad skrivs ut?
	\end{lstlisting}
	
	När man tar emot värden med \code{input()} så tolkar Python det alltid som strängar.
	
\end{frame}

\begin{frame}[fragile]
	\frametitle{Datatyper}
	\framesubtitle{Listor}
	
	När man vill spara flera värden i en variabel kan man använda sig av en \code{list}.
	
	\begin{lstlisting}
a = [3,4,8]
print(a[1]) # Vad skrivs ut?
	\end{lstlisting}
	
\end{frame}

\begin{frame}[fragile]
	\frametitle{Datatyper}
	\framesubtitle{Typkonvertering}
	
	Om du vill konvertera en variabel till en annan datatyp så gör man det så här:
	
	\begin{lstlisting}
a = int(7.5)
b = float('3.2')
	\end{lstlisting}
	
\end{frame}

\section{If-satser}

\begin{frame}[fragile]
	\frametitle{If-satser}
	
	Om något gör något
	
	\begin{lstlisting}
svar = input("Skriv ett tal: ")
svar = int (svar) # Konverterar till ett heltal
if svar == 3:
    print("Bra svar")
elif svar == 1:
    print("Intressant")
elif svar == 2:
    print("!")
else:
    print("??")
	\end{lstlisting}
	
\end{frame}

\section{While-loopar}

\begin{frame}[fragile]
	\frametitle{While}
	
	Så länge något gör något:
	
	\begin{lstlisting}
svar = "ja"
while svar == "ja":
    svar = input("Vill du fortsätta? ")
	\end{lstlisting}
	
	Det är viktigt att man uppdaterar variabeln i villkoret.
	
\end{frame}

\section{For-loopar}

\begin{frame}[fragile]
	\frametitle{For-loopar}
	\framesubtitle{range()}
	
	\begin{lstlisting}
for i in range(start, slut, steg):
    # Kod
	\end{lstlisting}
	
\end{frame}

\begin{frame}[fragile]
	\frametitle{For-loopar}
	\framesubtitle{Lista}
	
	\begin{lstlisting}
lista = [5,3,8,1,4]
for x in lista:
     print(x)
	\end{lstlisting}
	
\end{frame}

\section{Listor}

\begin{frame}[fragile]
	\frametitle{Listor}
	\framesubtitle{Kommandon}
	
	När man vill spara flera värden i en variabel kan man använda sig av en \code{list}.
	
	\begin{lstlisting}
a = [2,4,6] # Skapar listan
a.append(10) # Lägger till en 10 i slutet
a.insert(3, 8) # Lägger in talet 8 på plats 3
print(a) # Vad skrivs ut?

a.pop() # Tar bort sista talet
a.pop(1) # Tar bort talet på plats 1
a.remove(6) # Tar bort sexan
print(a) # Vad skrivs ut?
	\end{lstlisting}
	
\end{frame}

\begin{frame}[fragile]
	\frametitle{Listor}
	\framesubtitle{Iteration}
	
	För att komma åt ett element i listan skriver man \code{variabelnamn[position]}
	
	\begin{lstlisting}
min_lista = [0,1,2,3,4,5,6,7,8,9]
summa = 0
for i in range(len(min_lista)):
    summa += min_lista[i]
print(summa)
	\end{lstlisting}
	
\end{frame}

\begin{frame}[fragile]
	\frametitle{Listor}
	\framesubtitle{Listor i listor}

	En lista kan innehålla andra listor. Då kommer du åt ett element i den inre listan med \code{variabel[position_i_yttre_listan][position_i_inre_listan]}
	
	\begin{lstlisting}
min_lista = [[1,2,3],[4,5,6],[7,8,9]]
for row in range(len(min_lista)):
    for col in range(len(min_lista[row])):
        print(min_lista[row][col])
	\end{lstlisting}

\end{frame}

\begin{frame}[fragile]
	\frametitle{Listor}
	\framesubtitle{Listor i listor}

	Om du har en lista med listor som innehåller tal kan du rita det som ett ''rutnät'' med tal (en matris). Det hjälper ofta med att navigera rätt
	
	\begin{lstlisting}
min_lista = [[1,2,3],[4,5,6],[7,8,9]]
	\end{lstlisting}
	
	Den listan kan ses som:
	
	\begin{lstlisting}
[[1,2,3],
 [4,5,6],
 [7,8,9]]
	\end{lstlisting}
	
	Då kan man navigera listan genom att tänka sig rader och kolumner

\end{frame}

\section{Funktioner}

\begin{frame}[fragile]
	\frametitle{Funktioner}
	
	\begin{lstlisting}
def min_funktion(var1, var2):
    print(var1)
    print(var2)
    return var1+var2
    
print(min_funktion(2,3))
	\end{lstlisting}
	
\end{frame}

\section{Pseudokod}

\begin{frame}[fragile]
	\frametitle{Pseudokod}
	
	Kod som ligger mellan vanligt språk och programkod.
	
	\begin{lstlisting}
SET x TO 5
WHILE x LESS THAN 10
    SET x TO x+1
END WHILE
	\end{lstlisting}
	
	Det här är ett exempel på strukturerad engelska, som typ är pseudokod.
	
\end{frame}


\end{document}