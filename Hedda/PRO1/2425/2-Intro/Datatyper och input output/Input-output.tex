\documentclass[aspectratio=169]{beamer}

\mode<presentation>

\usepackage[utf8]{inputenc}
\usepackage[T1]{fontenc}	%makes å,ä,ö etc. proper symbols
\usepackage{amsmath}
\usepackage{graphicx}
\usepackage{xcolor}
\usepackage{listings}
\usepackage{multicol}
\usepackage{hyperref}

\usepackage[swedish]{babel}

\definecolor{LundaGroen}{RGB}{00,68,71}
\definecolor{StabilaLila}{RGB}{85,19,78}
\definecolor{VarmOrange}{RGB}{237,104,63}

\definecolor{MagnoliaRosa}{RGB}{251,214,209}
\definecolor{LundaHimmel}{RGB}{204,225,225}
\definecolor{LundaLjus}{RGB}{255,242,191}

\usefonttheme{serif}
\usetheme{malmoe}
\setbeamercolor{palette primary}{bg=VarmOrange}
\setbeamercolor{palette quaternary}{bg=LundaGroen}
\setbeamercolor{background canvas}{bg=LundaLjus}
\setbeamercolor{structure}{fg=LundaGroen}

\usepackage[many]{tcolorbox}

\newtcolorbox{cross}{blank,breakable,parbox=false,
  overlay={\draw[red,line width=5pt] (interior.south west)--(interior.north east);
    \draw[red,line width=5pt] (interior.north west)--(interior.south east);}}

\lstset{language=Python} 
\lstset{%language=[LaTeX]Tex,%C++,
    morekeywords={PassOptionsToPackage,selectlanguage,True,False},
    keywordstyle=\color{blue},%\bfseries,
    basicstyle=\small\ttfamily,
    %identifierstyle=\color{NavyBlue},
    commentstyle=\color{red}\ttfamily,
    stringstyle=\color{VarmOrange},
    numbers=left,%
    numberstyle=\scriptsize,%\tiny
    stepnumber=1,
    numbersep=8pt,
    showstringspaces=false,
    breaklines=true,
    %frameround=ftff,
    frame=single,
    belowcaptionskip=.75\baselineskip,
	tabsize=4,
	backgroundcolor=\color{white}
    %frame=L
}

\newcommand{\code}[1]{\colorbox{white}{\lstinline{#1}}}

\begin{document}

\lstset{literate=
  {á}{{\'a}}1 {é}{{\'e}}1 {í}{{\'i}}1 {ó}{{\'o}}1 {ú}{{\'u}}1
  {Á}{{\'A}}1 {É}{{\'E}}1 {Í}{{\'I}}1 {Ó}{{\'O}}1 {Ú}{{\'U}}1
  {à}{{\`a}}1 {è}{{\`e}}1 {ì}{{\`i}}1 {ò}{{\`o}}1 {ù}{{\`u}}1
  {À}{{\`A}}1 {È}{{\'E}}1 {Ì}{{\`I}}1 {Ò}{{\`O}}1 {Ù}{{\`U}}1
  {ä}{{\"a}}1 {ë}{{\"e}}1 {ï}{{\"i}}1 {ö}{{\"o}}1 {ü}{{\"u}}1
  {Ä}{{\"A}}1 {Ë}{{\"E}}1 {Ï}{{\"I}}1 {Ö}{{\"O}}1 {Ü}{{\"U}}1
  {â}{{\^a}}1 {ê}{{\^e}}1 {î}{{\^i}}1 {ô}{{\^o}}1 {û}{{\^u}}1
  {Â}{{\^A}}1 {Ê}{{\^E}}1 {Î}{{\^I}}1 {Ô}{{\^O}}1 {Û}{{\^U}}1
  {œ}{{\oe}}1 {Œ}{{\OE}}1 {æ}{{\ae}}1 {Æ}{{\AE}}1 {ß}{{\ss}}1
  {ű}{{\H{u}}}1 {Ű}{{\H{U}}}1 {ő}{{\H{o}}}1 {Ő}{{\H{O}}}1
  {ç}{{\c c}}1 {Ç}{{\c C}}1 {ø}{{\o}}1 {å}{{\r a}}1 {Å}{{\r A}}1
  {€}{{\euro}}1 {£}{{\pounds}}1 {«}{{\guillemotleft}}1
  {»}{{\guillemotright}}1 {ñ}{{\~n}}1 {Ñ}{{\~N}}1 {¿}{{?`}}1
}

\AtBeginSection[ ]
{
\begin{frame}{Outline}
    \tableofcontents[currentsection]
\end{frame}
}

\title{Input/Output}
\author{Programmering 1}
\date{2024/2025}

\maketitle

\section{input}

\subsection{Ta emot text}

\begin{frame}[fragile]
	\frametitle{Input}
	\framesubtitle{Att ta emot användarvärden}
	
	\begin{itemize}
		\item För att göra ett program \textit{interaktivt} behöver användaren kunna ge \textit{input}
	\end{itemize}
	
	\begin{lstlisting}
namn = input("Skriv ditt namn")
print("Hej", namn)
	\end{lstlisting}
	
\end{frame}

\subsection{Ta emot heltal}

\begin{frame}[fragile]
	\frametitle{Input}
	\framesubtitle{Att ta emot tal}
	
	\begin{itemize}
		\item Testa att skriva följande bit kod, kör den och se vad som händer
	\end{itemize}
	
	\begin{lstlisting}
tal = input("Skriv ett tal: ")
nytt_tal = tal*2
print("Om man dubblar ditt tal får man:", nytt_tal)
	\end{lstlisting}
	
\end{frame}

\begin{frame}[fragile]
	\frametitle{Input}
	\framesubtitle{Att ta emot tal}
	
	\begin{itemize}
		\item Python antar \textbf{alltid} att man ger text
		\item För att konvertera från text till ett heltal:
	\end{itemize}
	
	\begin{lstlisting}
tal = input("Skriv ett tal: ")
tal = int(tal)
	\end{lstlisting}
	
	\begin{itemize}
		\item Nu kan du använda det som ett heltal
		\item \texttt{int} är en förkortning av \textit{integer} som är engelska för heltal
	\end{itemize}
	
\end{frame}

\subsection{Att ta emot decimaltal}

\begin{frame}[fragile]
	\frametitle{Input}
	\framesubtitle{Att ta emot decimaltal}
	
	\begin{itemize}
		\item Testa ditt program igen, med förändringen att du \textit{konverterar} till ett heltal
		\item Testa nu att skriva in talet 1.2
	\end{itemize}
	
\end{frame}

\begin{frame}[fragile]
	\frametitle{Input}
	\framesubtitle{Att ta emot tal}
	
	\begin{itemize}
		\item \texttt{int} klarar bara av att konvertera text som bara innehåller siffror till tal
		\item För att konvertera från text till ett decimaltal (flyttal):
	\end{itemize}
	
	\begin{lstlisting}
tal = input("Skriv ett tal: ")
tal = float(tal)
	\end{lstlisting}
	
	\begin{itemize}
		\item Nu kan du använda det som ett decimaltal
	\end{itemize}
	
\end{frame}

\section{Output}

\subsection{print}

\begin{frame}[fragile]
	\frametitle{Output}
	\framesubtitle{print}
	
	\begin{itemize}
		\item För att skriva ut saker använder du \lstinline{print()}
		\item För att skriva ut flera saker kan du skriva \lstinline{print(a, b, c)}
	\end{itemize}
	
\end{frame}

\subsection{sep}

\begin{frame}[fragile]
	\frametitle{Output}
	\framesubtitle{sep}
	
	\begin{itemize}
		\item När du skriver ut flera saker kan du påverka vad som ska hända mellan bitarna.
		\item Du kan styra vad kommatecknena ska visas som. Standard är ett mellanslag
		\item Du lägger till \lstinline{, sep="..."} innan den avslutande parantesen
	\end{itemize}
	
	\begin{lstlisting}
print("Hej", "på", "dig", sep="!")
	\end{lstlisting}
	
\end{frame}

\subsection{end}

\begin{frame}[fragile]
	\frametitle{Output}
	\framesubtitle{end}
	
	\begin{itemize}
		\item Ibland vill du inte att det ska göras en radbrytning efter en \lstinline{print()}
		\item Då kan du lägga till \lstinline{, end="..."} innan den avslutande parantesen
	\end{itemize}
	
	\begin{lstlisting}
print("Hej", end=" ")
print("på", end="?")
print("dig", end="!")
	\end{lstlisting}
	
\end{frame}

\section{f-strängar}

\begin{frame}[fragile]
	\frametitle{f-strängar}
	\framesubtitle{Formatera strängar}
	
	\begin{itemize}
		\item När du vill göra en utskrift som ska innehålla mycket text och variabler kan du använda en f-sträng
		\item Det är viktigt att sätta ett \lstinline{f} framför det första citattecknet.
		\item Sen markerar du varje variabel med måsvingar.
	\end{itemize}	
	
	\begin{lstlisting}
a = 12
b = 6
print(f"Det finns {b} positiva heltalsdelare till talet {a}")
	\end{lstlisting}
	
\end{frame}

\end{document}