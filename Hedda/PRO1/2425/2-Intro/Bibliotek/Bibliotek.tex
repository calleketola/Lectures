\documentclass{beamer}

\mode<presentation>

\usepackage[utf8]{inputenc}
\usepackage[T1]{fontenc}	%makes å,ä,ö etc. proper symbols
\usepackage{amsmath}
\usepackage{graphicx}
\usepackage{xcolor}
\usepackage{listings}
\usepackage{multicol}
\usepackage{hyperref}


\definecolor{LundaGroen}{RGB}{00,68,71}
\definecolor{StabilaLila}{RGB}{85,19,78}
\definecolor{VarmOrange}{RGB}{237,104,63}

\definecolor{MagnoliaRosa}{RGB}{251,214,209}
\definecolor{LundaHimmel}{RGB}{204,225,225}
\definecolor{LundaLjus}{RGB}{255,242,191}

\usefonttheme{serif}
\usetheme{malmoe}
\setbeamercolor{palette primary}{bg=VarmOrange}
\setbeamercolor{palette quaternary}{bg=LundaGroen}
\setbeamercolor{background canvas}{bg=LundaLjus}
\setbeamercolor{structure}{fg=LundaGroen}

\usepackage[many]{tcolorbox}

\newtcolorbox{cross}{blank,breakable,parbox=false,
  overlay={\draw[red,line width=5pt] (interior.south west)--(interior.north east);
    \draw[red,line width=5pt] (interior.north west)--(interior.south east);}}



\lstset{language=Python} 
\lstset{%language=[LaTeX]Tex,%C++,
    morekeywords={PassOptionsToPackage,selectlanguage,True,False},
    keywordstyle=\color{blue},%\bfseries,
    basicstyle=\small\ttfamily,
    %identifierstyle=\color{NavyBlue},
    commentstyle=\color{red}\ttfamily,
    stringstyle=\color{VarmOrange},
    numbers=left,%
    numberstyle=\scriptsize,%\tiny
    stepnumber=1,
    numbersep=8pt,
    showstringspaces=false,
    breaklines=true,
    %frameround=ftff,
    frame=single,
    belowcaptionskip=.75\baselineskip,
	tabsize=4,
	backgroundcolor=\color{white}
    %frame=L
}


\begin{document}

\lstset{literate=
  {á}{{\'a}}1 {é}{{\'e}}1 {í}{{\'i}}1 {ó}{{\'o}}1 {ú}{{\'u}}1
  {Á}{{\'A}}1 {É}{{\'E}}1 {Í}{{\'I}}1 {Ó}{{\'O}}1 {Ú}{{\'U}}1
  {à}{{\`a}}1 {è}{{\`e}}1 {ì}{{\`i}}1 {ò}{{\`o}}1 {ù}{{\`u}}1
  {À}{{\`A}}1 {È}{{\'E}}1 {Ì}{{\`I}}1 {Ò}{{\`O}}1 {Ù}{{\`U}}1
  {ä}{{\"a}}1 {ë}{{\"e}}1 {ï}{{\"i}}1 {ö}{{\"o}}1 {ü}{{\"u}}1
  {Ä}{{\"A}}1 {Ë}{{\"E}}1 {Ï}{{\"I}}1 {Ö}{{\"O}}1 {Ü}{{\"U}}1
  {â}{{\^a}}1 {ê}{{\^e}}1 {î}{{\^i}}1 {ô}{{\^o}}1 {û}{{\^u}}1
  {Â}{{\^A}}1 {Ê}{{\^E}}1 {Î}{{\^I}}1 {Ô}{{\^O}}1 {Û}{{\^U}}1
  {œ}{{\oe}}1 {Œ}{{\OE}}1 {æ}{{\ae}}1 {Æ}{{\AE}}1 {ß}{{\ss}}1
  {ű}{{\H{u}}}1 {Ű}{{\H{U}}}1 {ő}{{\H{o}}}1 {Ő}{{\H{O}}}1
  {ç}{{\c c}}1 {Ç}{{\c C}}1 {ø}{{\o}}1 {å}{{\r a}}1 {Å}{{\r A}}1
  {€}{{\euro}}1 {£}{{\pounds}}1 {«}{{\guillemotleft}}1
  {»}{{\guillemotright}}1 {ñ}{{\~n}}1 {Ñ}{{\~N}}1 {¿}{{?`}}1
}

\AtBeginSection[ ]
{
\begin{frame}{Innehåll}
	\begin{multicols}{2}
    		\tableofcontents[currentsection]
	\end{multicols}
\end{frame}
}

\title{Bibliotek}
\date{2022/2023}

\maketitle

\section{Bibliotek}

\subsection{Varför bibliotek}

\begin{frame}
	\frametitle{Bibliotek}
	
	Python har många inbyggda funktioner och kan hantera mycket som det är. Men allt är inte inbyggt. Detta har ett par fördelar:
	
	\begin{itemize}
		\item Utvecklarna av Python är inte experter på allt.
		\item Det hade varit otroligt jobbigt att ladda ner och installera Python.
		\item Program kan optimeras genom att inte ha en massa kod som inte används.
		\item \textit{One size fits all} stämmer inte alltid.
	\end{itemize}

\end{frame}

\subsection{Vad gör ett bibliotek}

\begin{frame}
	\frametitle{Bibliotek}
	\framesubtitle{Vad gör biblioteken?}
	
	När man laddar ett bibliotek så läser Python in en fil med samma namn.
	
	Så om man laddar ett bibliotek (även kallat paket eller modul) så skapas alla variabler och funktioner som finns i biblioteket.

\end{frame}

\begin{frame}
	\frametitle{Bibliotek}
	\framesubtitle{Vad gör biblioteken?}
	
	När man laddar ett bibliotek så läser Python in en fil med samma namn.
	
	Python letar efter den här filen på två ställen:
	
	\begin{itemize}
	\item I samma mapp som filen du kör
	\item I \texttt{Python37-32/Lib} (eller motsvarande)
	\end{itemize}
	
	Sen är det möjligt att lägga till fler platser där Python ska leta.

\end{frame}

\section{Installera bibliotek}

\subsection{Om man installerat Python rätt}

\begin{frame}[fragile]
	\frametitle{Bibliotek}
	\framesubtitle{Installera nya bibliotek}
	
	De allra flesta Python-biblioteken, eller paketen som de också kallas, kan laddas hem genom \texttt{PIP}.
	
	Om man har fått hela PATH-historian att fungera gör man så här:
	
	\begin{verbatim}
Start->cmd
pip install paketnamn
	\end{verbatim}
	
	Då installeras paketet ''paketnamn''.

\end{frame}

\subsection{Om man missat PATH}

\begin{frame}[fragile]
	\frametitle{Bibliotek}
	\framesubtitle{Installera nya bibliotek}
	
	Om man inte har fått den hela PATH-historian att fungera gör man så här:
	
	\begin{itemize}
		\item Öppna mappen Python ligger i.
		\item Öppna mappen Scripts.
		\item Markera adressfältet och kopiera adressen.
		\item \texttt{Start->cmd}
		\item \texttt{cd adressen}
		\item \texttt{pip install paketnamn}
	\end{itemize}
	
	Då installeras paketet ''paketnamn''.

\end{frame}

\section{Använda bibliotek}

\begin{frame}
	\frametitle{Bibliotek}
	\framesubtitle{Använda bibliotek}
	
	Det finns fyra huvudsakliga sätt att använda sig utav bibliotek:
	
	De har alla olika fördelar och nackdelar.

\end{frame}

\subsection{Sätt 1}

\begin{frame}[fragile]
	\frametitle{Bibliotek}
	\framesubtitle{Använda bibliotek 1}
	
	\begin{lstlisting}
import math

tal = math.pi
	\end{lstlisting}
	
	Den här metoden att importera bibliotek är kanske den lättaste. Den försäkrar att man inte av misstag sparar över några variabler eller liknande eftersom om man vill komma åt något i det här biblioteket måste man använda förledet \texttt{math.}

\end{frame}

\subsection{Sätt 2}

\begin{frame}[fragile]
	\frametitle{Bibliotek}
	\framesubtitle{Använda bibliotek 2}
	
	\begin{lstlisting}
import math as m

tal = m.pi
	\end{lstlisting}
	
	Den här metoden att importera bibliotek är inte helt ovanlig och låter en att själv bestämma vad man vill ha som förled när man ska anropa något i biblioteket. 

\end{frame}

\subsection{Sätt 3}

\begin{frame}[fragile]
	\frametitle{Bibliotek}
	\framesubtitle{Använda bibliotek 3}
	
	\begin{lstlisting}
from math import *

tal = pi
	\end{lstlisting}
	
	Detta är det farligaste, men bekvämaste, sättet att importera ett bibliotek. Man slipper använda något förled men man kan råka spara över variabler och liknande som laddas från biblioteket. 

\end{frame}

\subsection{Sätt 4}

\begin{frame}[fragile]
	\frametitle{Bibliotek}
	\framesubtitle{Använda bibliotek 4}
	
	\begin{lstlisting}
from math import pi

tal = pi
	\end{lstlisting}
	
	Vet man att man bara vill använda en specifik del av ett bibliotek är det ibland onödigt att ladda hela biblioteket. Genom att bara definiera vad man vill ladda så kan man bara nå just den biten.

\end{frame}

\subsection{Sätt 4.5}

\begin{frame}[fragile]
	\frametitle{Bibliotek}
	\framesubtitle{Använda bibliotek 4.5}
	
	\begin{lstlisting}
from math import pi as p

tal = p
	\end{lstlisting}
	
	Om man vill kan man välja att dessutom döpa om variabeln. Detta fungerar inte om man laddar hela biblioteket med *.

\end{frame}


\end{document}