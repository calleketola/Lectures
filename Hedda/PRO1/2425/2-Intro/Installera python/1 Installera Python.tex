\documentclass[aspectratio=169]{beamer}

\mode<presentation>

\usepackage[utf8]{inputenc}
\usepackage[T1]{fontenc}	%makes å,ä,ö etc. proper symbols
\usepackage{amsmath}
\usepackage{graphicx}
\usepackage{xcolor}
\usepackage{listings}
\usepackage{multicol}
\usepackage{hyperref}

\definecolor{LundaGroen}{RGB}{00,68,71}
\definecolor{StabilaLila}{RGB}{85,19,78}
\definecolor{VarmOrange}{RGB}{237,104,63}

\definecolor{MagnoliaRosa}{RGB}{251,214,209}
\definecolor{LundaHimmel}{RGB}{204,225,225}
\definecolor{LundaLjus}{RGB}{255,242,191}

\usefonttheme{serif}
\usetheme{malmoe}
\setbeamercolor{palette primary}{bg=VarmOrange}
\setbeamercolor{palette quaternary}{bg=LundaGroen}
\setbeamercolor{background canvas}{bg=LundaLjus}
\setbeamercolor{structure}{fg=LundaGroen}

\usepackage[many]{tcolorbox}

\newtcolorbox{cross}{blank,breakable,parbox=false,
  overlay={\draw[red,line width=5pt] (interior.south west)--(interior.north east);
    \draw[red,line width=5pt] (interior.north west)--(interior.south east);}}

\lstset{language=[LaTeX]Tex,%C++,
    morekeywords={PassOptionsToPackage,selectlanguage},
    keywordstyle=\color{blue},%\bfseries,
    basicstyle=\small\ttfamily,
    %identifierstyle=\color{NavyBlue},
    commentstyle=\color{red}\ttfamily,
    stringstyle=\color{orange},
    numbers=left,%
    numberstyle=\scriptsize,%\tiny
    stepnumber=1,
    numbersep=8pt,
    showstringspaces=false,
    breaklines=true,
    %frameround=ftff,
    frame=single,
    belowcaptionskip=.75\baselineskip,
	tabsize=4
    %frame=L
}
\lstset{language=Python} 
\lstset{backgroundcolor=\color{white}}

\newcounter{uppgifter}

\begin{document}

\lstset{literate=
  {á}{{\'a}}1 {é}{{\'e}}1 {í}{{\'i}}1 {ó}{{\'o}}1 {ú}{{\'u}}1
  {Á}{{\'A}}1 {É}{{\'E}}1 {Í}{{\'I}}1 {Ó}{{\'O}}1 {Ú}{{\'U}}1
  {à}{{\`a}}1 {è}{{\`e}}1 {ì}{{\`i}}1 {ò}{{\`o}}1 {ù}{{\`u}}1
  {À}{{\`A}}1 {È}{{\'E}}1 {Ì}{{\`I}}1 {Ò}{{\`O}}1 {Ù}{{\`U}}1
  {ä}{{\"a}}1 {ë}{{\"e}}1 {ï}{{\"i}}1 {ö}{{\"o}}1 {ü}{{\"u}}1
  {Ä}{{\"A}}1 {Ë}{{\"E}}1 {Ï}{{\"I}}1 {Ö}{{\"O}}1 {Ü}{{\"U}}1
  {â}{{\^a}}1 {ê}{{\^e}}1 {î}{{\^i}}1 {ô}{{\^o}}1 {û}{{\^u}}1
  {Â}{{\^A}}1 {Ê}{{\^E}}1 {Î}{{\^I}}1 {Ô}{{\^O}}1 {Û}{{\^U}}1
  {œ}{{\oe}}1 {Œ}{{\OE}}1 {æ}{{\ae}}1 {Æ}{{\AE}}1 {ß}{{\ss}}1
  {ű}{{\H{u}}}1 {Ű}{{\H{U}}}1 {ő}{{\H{o}}}1 {Ő}{{\H{O}}}1
  {ç}{{\c c}}1 {Ç}{{\c C}}1 {ø}{{\o}}1 {å}{{\r a}}1 {Å}{{\r A}}1
  {€}{{\euro}}1 {£}{{\pounds}}1 {«}{{\guillemotleft}}1
  {»}{{\guillemotright}}1 {ñ}{{\~n}}1 {Ñ}{{\~N}}1 {¿}{{?`}}1
}

\lstset{escapeinside={(*@}{@*)}}

% NEW COMMANDS
\AtBeginSection[ ]
{
\begin{frame}{Outline}
\setbeamercolor{section in toc shaded}{fg=LundaGroen}
\setbeamercolor{subsection in toc shaded}{fg=black}
    \tableofcontents[currentsection]

\end{frame}
}


\title{Programmering 1}
\date{2024/2025}

\maketitle

\section{Lärare}

\begin{frame}
\frametitle{Programmering 1}
\framesubtitle{Lärare}

\begin{tabular}{ll}
Lärare 1:& Calle Ketola\\
		& calle.ketola@lund.se \\
Lärare 2:& Enrique Zuloaga Hormazabal\\
		& enrique.zuloagahormazabal@lund.se\\
Plats: 	& C1111
\end{tabular}

\end{frame}

\section{Om kursen}

\subsection{Innehåll}

\begin{frame}
\frametitle{Programmering 1}
\framesubtitle{Vad är det för kurs?}

Vad har ni för förväntningar?\\ \pause
Vad vi ska lära oss:\\
\begin{itemize}
\item if-satser
\item loopar
\item funktioner
\item pseudokod
\item program
\end{itemize}

Vi kommer att använda Python.

\end{frame}

\section{Installera Python}

\begin{frame}
	\frametitle{Programmering 1}
	\framesubtitle{Installera Python}
	
	Hemsida: \url{http://www.python.org/}\\
	Download \textbf{Python 3.12.5}\\
	Följ instruktionerna och klicka i följande:\\
	\begin{enumerate}
		\item Install pip
		\item Add to PATH
	\end{enumerate}
	
	Det gör vårterminen mycket enklare.

\end{frame}

\section{Starta Python}

\subsection{Skapa en ny fil}

\begin{frame}

\frametitle{Programmering 1}
\framesubtitle{Komma igång}
Olika sätt att skapa en fil:\\
1. Win$\rightarrow$Python$\rightarrow$IDLE$\rightarrow$File$\rightarrow$New File\\
2. Win$\rightarrow$IDLE$\rightarrow$File$\rightarrow$New File\\
3. Skapa ett nytt textdokument $\rightarrow$ byt filändelsen till .py $\rightarrow$ Edit with IDLE

\end{frame}

\subsection{Öppna en Pythonfil}

\begin{frame}
	\frametitle{Programmering 1}
	\framesubtitle{Öppna en fil}
	
	Om man vill öppna en Python-fil är det lättast att högerklicka och välja ''Edit with IDLE''
	
\end{frame}

\subsection{Köra en Pythonfil}

\begin{frame}
	\frametitle{Programmering 1}
	\framesubtitle{Köra en fil}
	
	Det finns tre sätt at köra en Python-fil i Windows:
	
	\begin{enumerate}
		\item Öppna filen i IDLE och klicka \texttt{run}
		\item Dubbelklicka på filen i Windows
		\item Genom kommandotolken
	\end{enumerate}
	
\end{frame}

\end{document}


