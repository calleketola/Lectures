\documentclass[aspectratio=169]{beamer}

\mode<presentation>

\usepackage[utf8]{inputenc}
\usepackage[T1]{fontenc}	%makes å,ä,ö etc. proper symbols
\usepackage{amsmath}
\usepackage{graphicx}
\usepackage{xcolor}
\usepackage{listings}
\usepackage{multicol}
\usepackage{hyperref}
\usepackage[swedish]{babel}

\definecolor{LundaGroen}{RGB}{00,68,71}
\definecolor{StabilaLila}{RGB}{85,19,78}
\definecolor{VarmOrange}{RGB}{237,104,63}

\definecolor{MagnoliaRosa}{RGB}{251,214,209}
\definecolor{LundaHimmel}{RGB}{204,225,225}
\definecolor{LundaLjus}{RGB}{255,242,191}

\usefonttheme{serif}
\usetheme{malmoe}
\setbeamercolor{palette primary}{bg=LundaHimmel, fg=StabilaLila}
\setbeamercolor{palette quaternary}{bg=LundaGroen, fg=MagnoliaRosa}
\setbeamercolor{background canvas}{bg=LundaLjus}
\setbeamercolor{structure}{fg=LundaGroen}

\usepackage[many]{tcolorbox}

\newtcolorbox{cross}{blank,breakable,parbox=false,
  overlay={\draw[red,line width=5pt] (interior.south west)--(interior.north east);
    \draw[red,line width=5pt] (interior.north west)--(interior.south east);}}
    
\newcommand{\code}[1]{\colorbox{white}{\lstinline{#1}}}



\lstset{language=Python} 
\lstset{%language=[LaTeX]Tex,%C++,
    morekeywords={PassOptionsToPackage,selectlanguage,True,False},
    keywordstyle=\color{blue},%\bfseries,
    basicstyle=\small\ttfamily,
    %identifierstyle=\color{NavyBlue},
    commentstyle=\color{red}\ttfamily,
    stringstyle=\color{VarmOrange},
    numbers=left,%
    numberstyle=\scriptsize,%\tiny
    stepnumber=1,
    numbersep=8pt,
    showstringspaces=false,
    breaklines=true,
    %frameround=ftff,
    frame=single,
    belowcaptionskip=.75\baselineskip,
	tabsize=4,
	backgroundcolor=\color{white}
    %frame=L
}


\begin{document}

\lstset{literate=
  {á}{{\'a}}1 {é}{{\'e}}1 {í}{{\'i}}1 {ó}{{\'o}}1 {ú}{{\'u}}1
  {Á}{{\'A}}1 {É}{{\'E}}1 {Í}{{\'I}}1 {Ó}{{\'O}}1 {Ú}{{\'U}}1
  {à}{{\`a}}1 {è}{{\`e}}1 {ì}{{\`i}}1 {ò}{{\`o}}1 {ù}{{\`u}}1
  {À}{{\`A}}1 {È}{{\'E}}1 {Ì}{{\`I}}1 {Ò}{{\`O}}1 {Ù}{{\`U}}1
  {ä}{{\"a}}1 {ë}{{\"e}}1 {ï}{{\"i}}1 {ö}{{\"o}}1 {ü}{{\"u}}1
  {Ä}{{\"A}}1 {Ë}{{\"E}}1 {Ï}{{\"I}}1 {Ö}{{\"O}}1 {Ü}{{\"U}}1
  {â}{{\^a}}1 {ê}{{\^e}}1 {î}{{\^i}}1 {ô}{{\^o}}1 {û}{{\^u}}1
  {Â}{{\^A}}1 {Ê}{{\^E}}1 {Î}{{\^I}}1 {Ô}{{\^O}}1 {Û}{{\^U}}1
  {œ}{{\oe}}1 {Œ}{{\OE}}1 {æ}{{\ae}}1 {Æ}{{\AE}}1 {ß}{{\ss}}1
  {ű}{{\H{u}}}1 {Ű}{{\H{U}}}1 {ő}{{\H{o}}}1 {Ő}{{\H{O}}}1
  {ç}{{\c c}}1 {Ç}{{\c C}}1 {ø}{{\o}}1 {å}{{\r a}}1 {Å}{{\r A}}1
  {€}{{\euro}}1 {£}{{\pounds}}1 {«}{{\guillemotleft}}1
  {»}{{\guillemotright}}1 {ñ}{{\~n}}1 {Ñ}{{\~N}}1 {¿}{{?`}}1
}

\AtBeginSection[ ]
{
\begin{frame}{Innehåll}
    	\tableofcontents[currentsection]
\end{frame}
}

\title{For-loopar}
\date{2024/25}
\author{Programmering 1}

\maketitle

\section{Start}

\subsection{While}

\begin{frame}[fragile]
	\frametitle{Loopa}
	\framesubtitle{While}
	
	En enkel \code{while}-loop som körs 10 gånger:
	
	\begin{lstlisting}
i = 0
while i < 10:
    print(i)
    i += 1
	\end{lstlisting}
	
\end{frame}

\subsection{for}

\begin{frame}[fragile]
	\frametitle{Loopa}
	\framesubtitle{For}
	
	Vi kan göra samma sak med en \code{for}-loop:
	
	\begin{lstlisting}
for i in range(10):
    print(i)
	\end{lstlisting}

\end{frame}

\subsection{range()}

\begin{frame}[fragile]
	\frametitle{For}
	\framesubtitle{range(n)}
	
	Kommandot \code{range(n)} skapar en ''lista'' med ordnade heltal från och med 0 till \code{n} (exkluderat \code{n}). Så totalt får man \code{n} tal.
	
\end{frame}

\subsection{range(a,b)}

\begin{frame}[fragile]
	\frametitle{For}
	\framesubtitle{range(a,b)}
	
	\begin{lstlisting}
for i in range(2, 5):
    print(i)
	\end{lstlisting}
	
	Skriver ut talen: \texttt{2 3 4}. Alltså från och med \texttt{a} till \texttt{b} (precis som med en lista \code{lista[2:5]}).
	
\end{frame}

\subsection{range(a,b,c)}

\begin{frame}[fragile]
	\frametitle{For}
	\framesubtitle{range(a,b,c)}
	
	\begin{lstlisting}
for i in range(0, 10,3):
    print(i)
	\end{lstlisting}
	
	Skriver ut talen: \texttt{0 3 6 9}. Alltså från och med \texttt{a} till \texttt{b} med \texttt{c} stegs mellanrum (precis som med en lista \code{lista[0:10:3]}).
	
\end{frame}

\subsection{Baklänges}

\begin{frame}[fragile]
	\frametitle{For}
	\framesubtitle{Baklänges}
	
	\begin{lstlisting}
for i in range(5, 0,-1):
    print(i)
	\end{lstlisting}
	
	Skriver ut talen: \texttt{5 4 3 2 1}. Alltså från och med \texttt{a} till \texttt{b} med \texttt{c} stegs mellanrum (precis som med en lista \code{lista[5:0:-1]}).
	
\end{frame}

\section{Övningar}

\subsection{Blad 1}

\begin{frame}
	\frametitle{Övningar}
	\framesubtitle{Blad 1}

	\begin{enumerate}
		\item Skriv en \code{while}-loop som skriver ut alla tal mellan 0 och 13.
		\item Skriv en \code{for}-loop som skriver ut alla tal mellan 0 och 13.
		\item Skriv en \code{while}-loop som skriver ut var tredje tal från 2 till 30.
		\item Skriv en \code{for}-loop som skriver ut var tredje tal från 2 till 30.
		\item Skriv en \code{while}-loop som skriver ut talen 10, 9, ... -9, -10.
		\item Skriv en \code{for}-loop som skriver ut talen 10, 9, ... -9, -10.
		\item Skriv en \code{for}-loop som ritar ut följande mönster:\\
		*\\
		**\\
		...\\
		********* (9 st)
	\end{enumerate}

\end{frame}

\subsection{Blad 2}

\begin{frame}[fragile]
	\frametitle{Övningar}
	\framesubtitle{Blad 2}
	
	\begin{enumerate}
		\setcounter{enumi}{7}
		\item Utveckla koden till att skriva ut följande mönster istället:
		\begin{verbatim}
        *
       ***
      *****
...
*******************
       ###
		\end{verbatim}
		\item Använd \code{random} för att skriva ut julgranskulor (\code{O}) i granen.
		\item Lägg också till julgransljus \code{i} i granen
	\end{enumerate}
\end{frame}


\end{document}