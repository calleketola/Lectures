\documentclass[aspectratio=169]{beamer}

\mode<presentation>

\usepackage[utf8]{inputenc}
\usepackage[T1]{fontenc}	%makes å,ä,ö etc. proper symbols
\usepackage{amsmath}
\usepackage{graphicx}
\usepackage{xcolor}
\usepackage{listings}
\usepackage{multicol}
\usepackage{hyperref}
\usepackage[swedish]{babel}

\definecolor{LundaGroen}{RGB}{00,68,71}
\definecolor{StabilaLila}{RGB}{85,19,78}
\definecolor{VarmOrange}{RGB}{237,104,63}

\definecolor{MagnoliaRosa}{RGB}{251,214,209}
\definecolor{LundaHimmel}{RGB}{204,225,225}
\definecolor{LundaLjus}{RGB}{255,242,191}

\usefonttheme{serif}
\usetheme{malmoe}
\setbeamercolor{palette primary}{bg=LundaHimmel, fg=StabilaLila}
\setbeamercolor{palette quaternary}{bg=LundaGroen, fg=MagnoliaRosa}
\setbeamercolor{background canvas}{bg=LundaLjus}
\setbeamercolor{structure}{fg=LundaGroen}

\usepackage[many]{tcolorbox}

\newtcolorbox{cross}{blank,breakable,parbox=false,
  overlay={\draw[red,line width=5pt] (interior.south west)--(interior.north east);
    \draw[red,line width=5pt] (interior.north west)--(interior.south east);}}
    
\newcommand{\code}[1]{\colorbox{white}{\lstinline{#1}}}



\lstset{language=Python} 
\lstset{%language=[LaTeX]Tex,%C++,
    morekeywords={PassOptionsToPackage,selectlanguage,True,False},
    keywordstyle=\color{blue},%\bfseries,
    basicstyle=\small\ttfamily,
    %identifierstyle=\color{NavyBlue},
    commentstyle=\color{red}\ttfamily,
    stringstyle=\color{VarmOrange},
    numbers=left,%
    numberstyle=\scriptsize,%\tiny
    stepnumber=1,
    numbersep=8pt,
    showstringspaces=false,
    breaklines=true,
    %frameround=ftff,
    frame=single,
    belowcaptionskip=.75\baselineskip,
	tabsize=4,
	backgroundcolor=\color{white}
    %frame=L
}


\begin{document}

\lstset{literate=
  {á}{{\'a}}1 {é}{{\'e}}1 {í}{{\'i}}1 {ó}{{\'o}}1 {ú}{{\'u}}1
  {Á}{{\'A}}1 {É}{{\'E}}1 {Í}{{\'I}}1 {Ó}{{\'O}}1 {Ú}{{\'U}}1
  {à}{{\`a}}1 {è}{{\`e}}1 {ì}{{\`i}}1 {ò}{{\`o}}1 {ù}{{\`u}}1
  {À}{{\`A}}1 {È}{{\'E}}1 {Ì}{{\`I}}1 {Ò}{{\`O}}1 {Ù}{{\`U}}1
  {ä}{{\"a}}1 {ë}{{\"e}}1 {ï}{{\"i}}1 {ö}{{\"o}}1 {ü}{{\"u}}1
  {Ä}{{\"A}}1 {Ë}{{\"E}}1 {Ï}{{\"I}}1 {Ö}{{\"O}}1 {Ü}{{\"U}}1
  {â}{{\^a}}1 {ê}{{\^e}}1 {î}{{\^i}}1 {ô}{{\^o}}1 {û}{{\^u}}1
  {Â}{{\^A}}1 {Ê}{{\^E}}1 {Î}{{\^I}}1 {Ô}{{\^O}}1 {Û}{{\^U}}1
  {œ}{{\oe}}1 {Œ}{{\OE}}1 {æ}{{\ae}}1 {Æ}{{\AE}}1 {ß}{{\ss}}1
  {ű}{{\H{u}}}1 {Ű}{{\H{U}}}1 {ő}{{\H{o}}}1 {Ő}{{\H{O}}}1
  {ç}{{\c c}}1 {Ç}{{\c C}}1 {ø}{{\o}}1 {å}{{\r a}}1 {Å}{{\r A}}1
  {€}{{\euro}}1 {£}{{\pounds}}1 {«}{{\guillemotleft}}1
  {»}{{\guillemotright}}1 {ñ}{{\~n}}1 {Ñ}{{\~N}}1 {¿}{{?`}}1
}

\AtBeginSection[ ]
{
\begin{frame}{Innehåll}
    	\tableofcontents[currentsection]
\end{frame}
}

\title{For-loopar och listor}
\date{ht 23}
\author{Programmering 1}

\maketitle

\section{Repetition}

\subsection{Listor}

\begin{frame}[fragile]
	\frametitle{Repetition}
	\framesubtitle{Listor}
	
	\begin{lstlisting}
djur = ["Apa", "Katt", "Hund", "Lejon", "Mås"]
print(djur[3]) # Skriver ut "Lejon"
	\end{lstlisting}
	
	\pause
	
	\begin{lstlisting}
djur.pop(2) # Plockar bort "Hund"
djur.append("Get") # Lägger till "Get" sist
djur.remove("Apa") # Plockar bort "Apa"
djur.insert(1, "Val") # Lägger in "Val" på plats 1
	\end{lstlisting}
	
	
\end{frame}

\subsection{For-loopar}

\begin{frame}[fragile]
	\frametitle{Repetition}
	\framesubtitle{For-loopar}
	
	\begin{lstlisting}
for i in range(10):
    print(i) # Skriver ut talen 0-9
	\end{lstlisting}
	
	\pause
	
	\begin{lstlisting}
for i in range(len(djur)):
    print(djur[i]) # Skriver ut djuren i ordning	
	\end{lstlisting}
	
	\pause
	
	\begin{lstlisting}
for i in range(-10, 10, 2):
    print(i) # Skriver ut -10, -8, -6... 8
	\end{lstlisting}
	
\end{frame}

\section{For och listor}

\subsection{Loopa igenom listan}

\begin{frame}[fragile]
	\frametitle{Loopa genom en lista}
	\framesubtitle{For och listor}
	
	\begin{lstlisting}
djur = ["Apa", "Katt", "Hund", "Lejon", "Mås"]
for i in range(len(djur)):
    print(djur[i])
	\end{lstlisting}
	
	\pause
	
	\begin{lstlisting}
djur = ["Apa", "Katt", "Hund", "Lejon", "Mås"]
for x in djur:
    print(x)
	\end{lstlisting}
	
\end{frame}

\subsection{Listor i listor}

\begin{frame}[fragile]
	\frametitle{Listor i listor}
	
	\begin{lstlisting}
djur = [ ["Katt", "Lejon", "Tiger"], ["Hund",   "Varg", "Räv", "Hyena"]]
for i in range(len(djur)):
    print(djur[i]) # Skriver ut vad?
	\end{lstlisting}
	
	\pause
	
	\begin{lstlisting}
djur = [ ["Katt", "Lejon", "Tiger"], ["Hund",   "Varg", "Räv", "Hyena"]]
for i in range(len(djur)):
    for j in range(len(djur[i])):
        print(djur[i][j]) # Skriver ut alla djur
	\end{lstlisting}
	
\end{frame}

\begin{frame}[fragile]
	\frametitle{Listor i listor}
	
	\begin{lstlisting}
djur = [ ["Katt", "Lejon", "Tiger"], ["Hund",   "Varg", "Räv", "Hyena"]]
for lista in djur:
    for d in lista:
        print(d) # Skriver ut alla djur
	\end{lstlisting}

\end{frame}

\section{Generera listor}

\subsection{Generera listor}

\begin{frame}[fragile]
	\frametitle{Generera listor}
	
	Man kan skapa listor med en rad kod så här:
	
	\begin{lstlisting}
listan = [i for i in range(100)]
	\end{lstlisting}
	
	\pause
	
	Om du vill ha flera listor i en lista:
	
	\begin{lstlisting}
x = [[i for i in range(10)] for j in range(10)]
	\end{lstlisting}

\end{frame}

\section{Övningar}

\begin{frame}
	\frametitle{Övningar}
	
	Ladda ner filen \texttt{forochlistor.py} från Classroom och följ instruktionerna i den.
	
\end{frame}


%\begin{frame}
%\frametitle{Övningar}
%\framesubtitle{Blad 3}
%
%Två-dimensionella objekt brukar indexeras på följande vis: \( a_{rad, kolumn} \)
%
%\[
%\left[
%\begin{array}{ccc}
%a_{0,0} & a_{0,1} & a_{0,2}\\
%a_{1,0} & a_{1,1} & a_{1,2}\\
%a_{2,0} & a_{2,1} & a_{2,2}
%\end{array}
%\right]
%\Rightarrow
%\left[ \begin{array}{ccc}
%a_{0} & a_{1} & a_{2}\\
%a_{3} & a_{4} & a_{5}\\
%a_{6} & a_{7} & a_{8}
%\end{array} \right]
%\]
% 
%Skriv en kod som kan identifiera varje element i den tvådimensionella matrisen med ett indextal. Koden ska ta emot ett tal mellan 0 och 8 och spotta ut två tal som motsvarar raden och kolumnen i.e. \( i = 5 \Rightarrow (1,2)\). Få den att klara av \(n\times m\)-objekt ($n$ rader och $m$ kolumner). Gör en kod som går på andra hållet. Från rad och kolumn till i.
% 
%\[
%\left[ \begin{array}{ccc}
%31 & 14 & 42\\
%29 & 93 & 77\\
%55 & 88 & 66
%\end{array} \right]
%\begin{split}
%a_{0,0} = 31 = a_0,\\
%a_{1,1} = 93 = a_4,\\
%a_{2,2} = 66 = a_8 
%\end{split}
% \]
%
%\end{frame}


\end{document}