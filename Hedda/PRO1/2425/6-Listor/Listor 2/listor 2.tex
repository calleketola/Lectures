\documentclass{beamer}

\mode<presentation>

\usepackage[utf8]{inputenc}
\usepackage[T1]{fontenc}	%makes å,ä,ö etc. proper symbols
\usepackage{amsmath}
\usepackage{graphicx}
\usepackage{xcolor}
\usepackage{listings}
\usepackage{multicol}
\usepackage{hyperref}
\usepackage[swedish]{babel}

\definecolor{LundaGroen}{RGB}{00,68,71}
\definecolor{StabilaLila}{RGB}{85,19,78}
\definecolor{VarmOrange}{RGB}{237,104,63}

\definecolor{MagnoliaRosa}{RGB}{251,214,209}
\definecolor{LundaHimmel}{RGB}{204,225,225}
\definecolor{LundaLjus}{RGB}{255,242,191}

\usefonttheme{serif}
\usetheme{malmoe}
\setbeamercolor{palette primary}{bg=LundaHimmel, fg=StabilaLila}
\setbeamercolor{palette quaternary}{bg=LundaGroen, fg=MagnoliaRosa}
\setbeamercolor{background canvas}{bg=LundaLjus}
\setbeamercolor{structure}{fg=LundaGroen}

\usepackage[many]{tcolorbox}

\newtcolorbox{cross}{blank,breakable,parbox=false,
  overlay={\draw[red,line width=5pt] (interior.south west)--(interior.north east);
    \draw[red,line width=5pt] (interior.north west)--(interior.south east);}}
    
\newcommand{\code}[1]{\colorbox{white}{\lstinline{#1}}}



\lstset{language=Python} 
\lstset{%language=[LaTeX]Tex,%C++,
    morekeywords={PassOptionsToPackage,selectlanguage,True,False},
    keywordstyle=\color{blue},%\bfseries,
    basicstyle=\small\ttfamily,
    %identifierstyle=\color{NavyBlue},
    commentstyle=\color{red}\ttfamily,
    stringstyle=\color{VarmOrange},
    numbers=left,%
    numberstyle=\scriptsize,%\tiny
    stepnumber=1,
    numbersep=8pt,
    showstringspaces=false,
    breaklines=true,
    %frameround=ftff,
    frame=single,
    belowcaptionskip=.75\baselineskip,
	tabsize=4,
	backgroundcolor=\color{white}
    %frame=L
}


\begin{document}

\lstset{literate=
  {á}{{\'a}}1 {é}{{\'e}}1 {í}{{\'i}}1 {ó}{{\'o}}1 {ú}{{\'u}}1
  {Á}{{\'A}}1 {É}{{\'E}}1 {Í}{{\'I}}1 {Ó}{{\'O}}1 {Ú}{{\'U}}1
  {à}{{\`a}}1 {è}{{\`e}}1 {ì}{{\`i}}1 {ò}{{\`o}}1 {ù}{{\`u}}1
  {À}{{\`A}}1 {È}{{\'E}}1 {Ì}{{\`I}}1 {Ò}{{\`O}}1 {Ù}{{\`U}}1
  {ä}{{\"a}}1 {ë}{{\"e}}1 {ï}{{\"i}}1 {ö}{{\"o}}1 {ü}{{\"u}}1
  {Ä}{{\"A}}1 {Ë}{{\"E}}1 {Ï}{{\"I}}1 {Ö}{{\"O}}1 {Ü}{{\"U}}1
  {â}{{\^a}}1 {ê}{{\^e}}1 {î}{{\^i}}1 {ô}{{\^o}}1 {û}{{\^u}}1
  {Â}{{\^A}}1 {Ê}{{\^E}}1 {Î}{{\^I}}1 {Ô}{{\^O}}1 {Û}{{\^U}}1
  {œ}{{\oe}}1 {Œ}{{\OE}}1 {æ}{{\ae}}1 {Æ}{{\AE}}1 {ß}{{\ss}}1
  {ű}{{\H{u}}}1 {Ű}{{\H{U}}}1 {ő}{{\H{o}}}1 {Ő}{{\H{O}}}1
  {ç}{{\c c}}1 {Ç}{{\c C}}1 {ø}{{\o}}1 {å}{{\r a}}1 {Å}{{\r A}}1
  {€}{{\euro}}1 {£}{{\pounds}}1 {«}{{\guillemotleft}}1
  {»}{{\guillemotright}}1 {ñ}{{\~n}}1 {Ñ}{{\~N}}1 {¿}{{?`}}1
}

\AtBeginSection[ ]
{
\begin{frame}{Innehåll}
    	\tableofcontents[currentsection]
\end{frame}
}

\title{Listor}
\date{ht 23}
\author{Programmering 1}

\maketitle

\section{Intro}

\subsection{Skapa en lista}

\begin{frame}[fragile]
	\frametitle{Listor}
	
	Ganska ofta vill man hantera flera objekt samtidigt. Till exempel alla domarresultat från en simhopppstävling, eller namnen på alla i en familj. Då kan man använda sig utav \texttt{listor}.
	
	\begin{lstlisting}
poäng = [7.56, 5.3, 9.34, 8.2] # Poäng från domarna

namn = ['Dwalin','Balin','Kili','Fili','Dori',  'Nori','Ori','Oin','Gloin','Bifur','Bofur',  'Bombur','Thorin']
	\end{lstlisting}
\end{frame}

\begin{frame}[fragile]
	\frametitle{Listor}
	
	\begin{lstlisting}
lista = [] # En tom lista
namn = ['Dwalin','Balin','Kili','Fili','Dori',  'Nori','Ori','Oin','Gloin','Bifur','Bofur',  'Bombur','Thorin']
	\end{lstlisting}
	
	En lista känns igen på hakparanteserna, \code{[ ]}. Dessa markerar var listan börjar och slutar. Varje element separeras med ett komma. Så \code{'Dwalin'} är ett element och \code{'Balin'} är ett annat.

\end{frame}

\subsection{Datatyper}

\begin{frame}[fragile]
\frametitle{Listor}

En lista kan innehålla objekt av olika datatyper.

\begin{lstlisting}
supercoollista = ['Snövit', 7, 'T-rex', 3.54, False]
\end{lstlisting}

En lista kan till och med innehålla flera listor:

\begin{lstlisting}
coolare_lista = [ ['Snövit','Bambi','Fantasia'], 1, 7, ['Dwalin', 'Balin']]
\end{lstlisting}

Nu innehåller \code{coolare\_lista} de fyra elementen:

\begin{lstlisting}
['Snövit', 'Bambi', 'Fantasia']
1
7
['Dwalin', 'Balin']
\end{lstlisting}

\end{frame}

\section{Indexering}

\subsection{Hitta element}

\begin{frame}[fragile]
	\frametitle{Listor}
	\frametitle{Indexering}

	Vill man komma åt enskilda objekt i en lista skriver man listans namn, åtföljt av hakparanteser med indexplatsen emellan.

	\begin{lstlisting}
hober = ['Frodo', 'Sam', 'Merry', 'Pippin']

print(hober[2])
	\end{lstlisting}

\end{frame}

\subsection{Börjar på 0}

\begin{frame}[fragile]
\frametitle{Listor}
\frametitle{Indexering}

För att utnyttja datorns minne till max början man att räkna på 0. (Här är en argumenterande text om detta: \href{https://www.cs.utexas.edu/users/EWD/ewd08xx/EWD831.PDF}{Why numbering should start at zero,  prof. dr. Edsger W. Dijkstra})

\begin{lstlisting}
platser = [0,1,2,3,4,5,6] # Sju lång

istari = ['Saruman', 'Gandalf', 'Radagast']
#          Plats 0    Plats 1     Plats 2
print(istari[0])
-> 'Saruman'
print(istari[1])
-> 'Gandalf'
print(istari[2])
-> 'Radagast'
\end{lstlisting}

\end{frame}

\begin{frame}[fragile]
\frametitle{Listor}
\frametitle{Indexering}

Man kan också komma åt element i listan genom att räkna baklänges:

\begin{lstlisting}
istari = ['Saruman', 'Gandalf', 'Radagast']
#          Plats 0    Plats 1     Plats 2
#          Plats -3   Plats -2    Plats -1
print(istari[-1]) # Första elementet bakifrån
-> 'Radagast'
print(istari[-2]) # Andra elementet bakifrån
-> 'Gandalf'
print(istari[-3]) # Tredje elementet bakifrån
-> 'Saruman'
\end{lstlisting}

\end{frame}

\section{Lägg till \& ta bort}

\subsection{append()}

\begin{frame}[fragile]
\frametitle{Listor}
\frametitle{append()}

Vill man lägga till ett element i en lista gör man det lättast genom att använda kommandot \code{listan.append(sak)}

\begin{lstlisting}
tal = [2,4,8,89]
tal.append(3)
print(tal)
-> [2,4,8,89,3] # En trea har lagts till
tal.append('Hej')
print(tal)
-> [2,4,8,89,3,'Hej']
\end{lstlisting}

Med \code{append} hamnar det nya elementet alltid sist.

\end{frame}

\subsection{insert()}

\begin{frame}[fragile]
\frametitle{Listor}
\frametitle{insert()}

Vill man inte att det nya objektet ska hamna sist, utan man vill att det ska hamna på en specifik position kan man använda sig utav \code{listan.insert(index, objekt)}

\begin{lstlisting}
tal = [2,4,8,89]
tal.insert(3, 'Hej') # Lägger till 'Hej' på position 3
print(tal)
-> [2,4,8,'Hej',89]
\end{lstlisting}

Kom ihåg att index börjar räkna på 0.

\end{frame}

\subsection{remove()}

\begin{frame}[fragile]
\frametitle{Listor}
\frametitle{remove()}

Vill man ta bort ett element ur en lista så kan man använda sig utav kommandot \code{remove}:

\begin{lstlisting}
istari = ['Saruman', 'Gandalf', 'Radagast']
istari.remove('Saruman')
print(istari)
->['Gandalf', 'Radagast']
\end{lstlisting}

\end{frame}

\begin{frame}[fragile]
\frametitle{Listor}
\frametitle{remove()}

Om samma element förekommer flera gånger så tas enbart det första elementet bort:

\begin{lstlisting}
listan = [1,2,75,6,7,75,6,7,75,6,7]
listan.remove(75)
print(listan)
-> [1,2,6,7,75,6,7,75,6,7]
namn = ['Knatte', 'Fnatte', 'Tjatte', 'Tjatte', 'Fnatte', 'Knatte']
namn.remove('Fnatte')
print(namn)
-> ['Knatte', 'Tjatte', 'Tjatte', 'Fnatte', 'Knatte']
\end{lstlisting}

Tänk på att man bara kan ta bort element som finns i listan.

\end{frame}

\begin{frame}[fragile]
\frametitle{Listor}
\framesubtitle{remove()}

Man kan försäkra sig om att bara ta bort element som finns i listan på följande vis:

\begin{lstlisting}
lista = [en lista]
while 'hej' in lista:
    lista.remove('hej')
\end{lstlisting}

Den här loopen håller på så länge elementet \code{'hej'} finns i \code{lista}.

\end{frame}

\subsection{pop()}

\begin{frame}[fragile]
\frametitle{Listor}
\framesubtitle{pop()}

Ibland är man mer intresserad av att ta bort element på en särskild plats. Då kan man använda sig utav \code{pop()}. Anger man inget argument i \code{pop()} så plockar man bort det sista elementet. Det som särskilt skiljer \code{pop()} från \code{remove()} är att \code{pop()} \textit{returnerar} det tal som ''poppas''. Man kan alltså skriva \code{a = lista.pop()} så får \code{a} samma värde som sista elementet i listan som försvinner.

\begin{lstlisting}
lista = ['Aragorn','Boromir','Legolas','Gimli']
lista.pop()
print(lista)
-> ['Aragorn', 'Boromir', 'Legolas']
\end{lstlisting}

\end{frame}

\begin{frame}[fragile]
\frametitle{Listor}
\framesubtitle{pop(index)}

Vill man plocka bort ett särskilt element kan man ange dess index.

\begin{lstlisting}
lista = ['Aragorn','Boromir','Legolas','Gimli']
lista.pop(1)
print(lista)
-> ['Aragorn', 'Legolas', 'Gimli']
\end{lstlisting}
\end{frame}

\section{Längden på en lista}

\begin{frame}[fragile]
\frametitle{Längd på listan}

Om man vill veta längden på en lista så kan man använda sig utav kommandot \code{len(listan)}

\begin{lstlisting}
lista = ["Frodo", "Sam", "Merry", "Pippin", "Gandalf", "Aragorn", "Boromir", "Legolas", "Gimli"]
print(len(lista))
-> 9
\end{lstlisting}

\end{frame}

\section{Sammanfattning}

\begin{frame}[fragile]
\frametitle{Listor}
\framesubtitle{Sammanfattning}

Följande kommandon är inbygda i listor:

\begin{center}
\begin{tabular}{|lp{6cm}|}
\hline
	\textbf{Kommando} & \textbf{Effekt} \\ \hline
	\code{append(sak)}	& Lägg till \code{sak} i slutet på listan \\ 
	\code{insert(index, sak)} & Lägg till \code{sak} på plats \code{index} \\ 
	\code{pop()}			& Tar bort och returnerar det sista elementet\\ 
	\code{pop(index)}		& Tar bort och returnerar elementet på plats \code{index} \\
	\code{remove(sak)}	& Tar bort den första förekomsten av \code{sak} i listan. \\ \hline
\end{tabular}
\end{center}

\end{frame}

\section{Övningar}

\begin{frame}
\frametitle{Listor}
\framesubtitle{Övningar}

Utgå från filen \texttt{listor 1.py}

\begin{enumerate}
\item Lägg till \code{'Gimli'} i slutet på listan
\item Lägg till \code{'Merry'} och \code{'Pippin'} mellan \code{'Sam'} och \code{'Aragorn'}
\item Ta bort \code{'Boromir'} ur listan.
\item Vänd på listan utan att använda \code{reverse()}
\item Ta bort alla udda tal ur listan.
\item Dela upp listan i tre nya listor, en för varje datatyp (\code{int}, \code{float} och \code{str}).
\item Gör varje element i listan dubbelt så stort.
\item Räkna hur många gånger varje element förekommer i listan.
\item Sortera listan utan att använda \code{sort()}
\end{enumerate}

\end{frame}

\end{document}