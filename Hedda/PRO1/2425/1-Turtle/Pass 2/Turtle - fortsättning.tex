\documentclass[aspectratio=169]{beamer}

\mode<presentation>

\usepackage[utf8]{inputenc}
\usepackage[T1]{fontenc}	%makes å,ä,ö etc. proper symbols
\usepackage{amsmath}
\usepackage{graphicx}
\usepackage{xcolor}
\usepackage{listings}
\usepackage{multicol}
\usepackage{hyperref}


\definecolor{LundaGroen}{RGB}{00,68,71}
\definecolor{StabilaLila}{RGB}{85,19,78}
\definecolor{VarmOrange}{RGB}{237,104,63}

\definecolor{MagnoliaRosa}{RGB}{251,214,209}
\definecolor{LundaHimmel}{RGB}{204,225,225}
\definecolor{LundaLjus}{RGB}{255,242,191}

\usefonttheme{serif}
\usetheme{malmoe}
\setbeamercolor{palette primary}{bg=VarmOrange}
\setbeamercolor{palette quaternary}{bg=LundaGroen}
\setbeamercolor{background canvas}{bg=LundaLjus}
\setbeamercolor{structure}{fg=LundaGroen}

\usepackage[many]{tcolorbox}

\newtcolorbox{cross}{blank,breakable,parbox=false,
  overlay={\draw[red,line width=5pt] (interior.south west)--(interior.north east);
    \draw[red,line width=5pt] (interior.north west)--(interior.south east);}}



\lstset{language=Python} 
\lstset{%language=[LaTeX]Tex,%C++,
    morekeywords={PassOptionsToPackage,selectlanguage,True,False},
    keywordstyle=\color{blue},%\bfseries,
    basicstyle=\small\ttfamily,
    %identifierstyle=\color{NavyBlue},
    commentstyle=\color{red}\ttfamily,
    stringstyle=\color{VarmOrange},
    numbers=left,%
    numberstyle=\scriptsize,%\tiny
    stepnumber=1,
    numbersep=8pt,
    showstringspaces=false,
    breaklines=true,
    %frameround=ftff,
    frame=single,
    belowcaptionskip=.75\baselineskip,
	tabsize=4,
	backgroundcolor=\color{white}
    %frame=L
}

\newcommand{\code}[1]{\colorbox{white}{\lstinline{#1}}}

\begin{document}

\lstset{literate=
  {á}{{\'a}}1 {é}{{\'e}}1 {í}{{\'i}}1 {ó}{{\'o}}1 {ú}{{\'u}}1
  {Á}{{\'A}}1 {É}{{\'E}}1 {Í}{{\'I}}1 {Ó}{{\'O}}1 {Ú}{{\'U}}1
  {à}{{\`a}}1 {è}{{\`e}}1 {ì}{{\`i}}1 {ò}{{\`o}}1 {ù}{{\`u}}1
  {À}{{\`A}}1 {È}{{\'E}}1 {Ì}{{\`I}}1 {Ò}{{\`O}}1 {Ù}{{\`U}}1
  {ä}{{\"a}}1 {ë}{{\"e}}1 {ï}{{\"i}}1 {ö}{{\"o}}1 {ü}{{\"u}}1
  {Ä}{{\"A}}1 {Ë}{{\"E}}1 {Ï}{{\"I}}1 {Ö}{{\"O}}1 {Ü}{{\"U}}1
  {â}{{\^a}}1 {ê}{{\^e}}1 {î}{{\^i}}1 {ô}{{\^o}}1 {û}{{\^u}}1
  {Â}{{\^A}}1 {Ê}{{\^E}}1 {Î}{{\^I}}1 {Ô}{{\^O}}1 {Û}{{\^U}}1
  {œ}{{\oe}}1 {Œ}{{\OE}}1 {æ}{{\ae}}1 {Æ}{{\AE}}1 {ß}{{\ss}}1
  {ű}{{\H{u}}}1 {Ű}{{\H{U}}}1 {ő}{{\H{o}}}1 {Ő}{{\H{O}}}1
  {ç}{{\c c}}1 {Ç}{{\c C}}1 {ø}{{\o}}1 {å}{{\r a}}1 {Å}{{\r A}}1
  {€}{{\euro}}1 {£}{{\pounds}}1 {«}{{\guillemotleft}}1
  {»}{{\guillemotright}}1 {ñ}{{\~n}}1 {Ñ}{{\~N}}1 {¿}{{?`}}1
}

\AtBeginSection[ ]
{
\begin{frame}{Outline}
    \tableofcontents[currentsection]
\end{frame}
}

\title{Sköldpaddsprogrammering}
\date{2024/2025}

\maketitle

\section{Kommandon}

\subsection{Förflyttning}

\begin{frame}
	\frametitle{Kommandon}
	\framesubtitle{Förflyttning}
	
	Här är en lista med kommandon som flyttar på paddan:
	
	\begin{tabular}{ll}
		\code{forward(x)} & Går x steg framåt\\
		\code{back(x)} & Går x steg bakåt\\
		\code{right(x)} & Roterar x grader medurs\\
		\code{left(x)} & Roterar x grader moturs\\
		\code{setposition((x,y))} & Placerar paddan i position (x,y)\\
		\code{setheading(x)} & Roterar paddan till x grader
	\end{tabular}
	
\end{frame}

\subsection{Andra kommandon}

\begin{frame}
	\frametitle{Kommandon}
	\framesubtitle{Andra kommandon}
	
	\begin{tabular}{ll}
		\code{penup()} & Slutar rita\\
		\code{pendown()} & Börjar rita\\
		\code{color("färg")} & Ändrar färgen\\
		\code{begin_fill()} & \\
		\code{end_fill()} & \\
		\code{fillcolor('färg')} & Ändrar den inre färgen\\
		\code{shape('turtle')} & Ändrar formen till en padda\\
		\code{clear()} & Tömmer skärmen
	\end{tabular}
	
\end{frame}

\section{Loopar}

\begin{frame}[fragile]
	\frametitle{Upprepningar}
	
	Om man vill rita en femhörning kan man göra så här:
	
	\begin{lstlisting}
from turtle import * # Laddar turtle

forward(100) # Gå framåt
right(72) # Sväng höger
forward(100)
right(72)
forward(100)
right(72)
forward(100)
right(72)
forward(100)
	\end{lstlisting}
	
\end{frame}

\subsection{Matten bakom siffrorna}

\begin{frame}
	\frametitle{Upprepningar}
	\framesubtitle{Var kom 72 från?}

\begin{equation*}
	\begin{aligned}
		\text{Vinkelsumma: } & 180\cdot(\text{hörn}-2)=\\
		& 180\cdot(5-2)=\\
		& 180\cdot3=540\\
		\text{Innervinkel: } & \dfrac{540}{5}=108\\
		\text{Yttervinkel: } & 180-108=72\\
		\text{Alternativt: } & \dfrac{360}{5}=72
	\end{aligned}
\end{equation*}

\end{frame}

\subsection{Loop}

\begin{frame}[fragile]
	\frametitle{Upprepningar}
	\framesubtitle{while}
	
	Vi kan förkorta förra slidens tio rader kod till sex rader och resultatet blir det samma:
	
	\begin{lstlisting}
from turtle import * # Laddar turtle

sidor = 0 # 

while sidor < 5: 
    forward(100) # Gå framåt
    right(72) # Sväng höger
    sidor = sidor + 1 #
	\end{lstlisting}
	
Vad står \texttt{hörn} för?

\end{frame}

\subsection{while}

\begin{frame}[fragile]
	\frametitle{While}
	
	\begin{lstlisting}
sidor = 0
while sidor < 5:
    sidor = sidor + 1
	\end{lstlisting}
	
	En \lstinline{while}-loop körs så länge som villkoret högst upp är uppfyllt --- i det här fallet så länge som antalet ritade sidor är mindre än fem.
	
	\pause
	
	Vad hade hänt om man tog bort den sista raden?
	
\end{frame}

\section{Smart programmering}

\subsection{Startpunkt}

\begin{frame}[fragile]
	\frametitle{Smart programmering}
	\framesubtitle{Här är vi}
	
	
	\begin{lstlisting}
from turtle import * # Laddar turtle

sidor = 0 # 

while sidor < 5: 
    forward(100) # Gå framåt
    right(72) # Sväng höger
    sidor = sidor + 1 #
	\end{lstlisting}
	
	Hur ska vi göra för att ändra till en sex-hörning (hexagon)?

\end{frame}

\subsection{Hexagon}

\begin{frame}[fragile]
	\frametitle{Smart programmering}
	\framesubtitle{Hexagon}
	
	\begin{lstlisting}
from turtle import * # Laddar turtle

hörn = 0 # 

while sidor < 6: 
    forward(100) # Gå framåt
    right(60) # Sväng höger
    sidor = sidor + 1 #
	\end{lstlisting}
	
	\begin{itemize}
		\item Hur ska vi göra för att ändra till en åtta-hörning (octagon)?
		\item Hur ska vi göra för att ändra till en \(n\)-hörning?
	\end{itemize}

\end{frame}

\subsection{\(n\)-hörning}

\begin{frame}[fragile]
	\frametitle{\(n\)-hörning}
	
	\begin{center}
	\includegraphics[width=.5\textwidth]{unicorn.jpg}
	\end{center}
	
\end{frame}

\begin{frame}[fragile]
	\frametitle{Smart programmering}
	\framesubtitle{\(n\)-hörning}
	
	\begin{lstlisting}
from turtle import * # Laddar turtle

sidor = 0 # Ritade sidor
totala_sidor = 5 # sidor på figuren

vinkel = 180-180*(totala_sidor-2)/totala_sidor

while sidor < totala_sidor: # Så länge vi inte ritat alla sidor
    forward(100) # Gå framåt
    right(vinkel) # Sväng höger
    sidor = sidor + 1 # Öka antalet ritade sidor
	\end{lstlisting}
	
\end{frame}

\section{Övningar}

\subsection{Blad 1}

\begin{frame}
	\frametitle{Övningar}
	\framesubtitle{Blad 1}
	
	\begin{enumerate}
		\item Ladda ner filen \texttt{turtle2.py} från Classroom
		\item Justera den så att den kan rita ut n-hörningar
		\item Justera den så att den kan rita ut n-uddiga stjärnor
		\item Justera din kod så att den ritar ut en stjärna i mitten av din polygon
		\item Justera din kod så att den ritar ut n-uddiga stjärnor i varje hörn på din polygon
		\item Justera din kod så att den gör ett spiralmönster med stjärnor som får fler och fler uddar.
	\end{enumerate}

\end{frame}

\subsection{Blad 2}

\begin{frame}
	\frametitle{Övningar}
	\framesubtitle{Blad 2}
	
	\begin{enumerate}
		\item Rita en spiral av kvadrater
		\item Rita en von Koch-kurva
	\end{enumerate}

\end{frame}




\end{document}


