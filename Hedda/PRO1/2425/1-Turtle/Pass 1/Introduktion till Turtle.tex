\documentclass[aspectratio=169]{beamer}

\mode<presentation>

\usepackage[utf8]{inputenc}
\usepackage[T1]{fontenc}	%makes å,ä,ö etc. proper symbols
\usepackage{amsmath}
\usepackage{graphicx}
\usepackage{xcolor}
\usepackage{listings}
\usepackage{multicol}
\usepackage{hyperref}


\definecolor{LundaGroen}{RGB}{00,68,71}
\definecolor{StabilaLila}{RGB}{85,19,78}
\definecolor{VarmOrange}{RGB}{237,104,63}

\definecolor{MagnoliaRosa}{RGB}{251,214,209}
\definecolor{LundaHimmel}{RGB}{204,225,225}
\definecolor{LundaLjus}{RGB}{255,242,191}

\usefonttheme{serif}
\usetheme{malmoe}
\setbeamercolor{palette primary}{bg=VarmOrange}
\setbeamercolor{palette quaternary}{bg=LundaGroen}
\setbeamercolor{background canvas}{bg=LundaLjus}
\setbeamercolor{structure}{fg=LundaGroen}

\usepackage[many]{tcolorbox}

\newtcolorbox{cross}{blank,breakable,parbox=false,
  overlay={\draw[red,line width=5pt] (interior.south west)--(interior.north east);
    \draw[red,line width=5pt] (interior.north west)--(interior.south east);}}



\lstset{language=Python} 
\lstset{%language=[LaTeX]Tex,%C++,
    morekeywords={PassOptionsToPackage,selectlanguage,True,False},
    keywordstyle=\color{blue},%\bfseries,
    basicstyle=\small\ttfamily,
    %identifierstyle=\color{NavyBlue},
    commentstyle=\color{red}\ttfamily,
    stringstyle=\color{VarmOrange},
    numbers=left,%
    numberstyle=\scriptsize,%\tiny
    stepnumber=1,
    numbersep=8pt,
    showstringspaces=false,
    breaklines=true,
    %frameround=ftff,
    frame=single,
    belowcaptionskip=.75\baselineskip,
	tabsize=4,
	backgroundcolor=\color{white}
    %frame=L
}

\newcommand{\code}[1]{\colorbox{white}{\lstinline{#1}}}


\begin{document}

\lstset{literate=
  {á}{{\'a}}1 {é}{{\'e}}1 {í}{{\'i}}1 {ó}{{\'o}}1 {ú}{{\'u}}1
  {Á}{{\'A}}1 {É}{{\'E}}1 {Í}{{\'I}}1 {Ó}{{\'O}}1 {Ú}{{\'U}}1
  {à}{{\`a}}1 {è}{{\`e}}1 {ì}{{\`i}}1 {ò}{{\`o}}1 {ù}{{\`u}}1
  {À}{{\`A}}1 {È}{{\'E}}1 {Ì}{{\`I}}1 {Ò}{{\`O}}1 {Ù}{{\`U}}1
  {ä}{{\"a}}1 {ë}{{\"e}}1 {ï}{{\"i}}1 {ö}{{\"o}}1 {ü}{{\"u}}1
  {Ä}{{\"A}}1 {Ë}{{\"E}}1 {Ï}{{\"I}}1 {Ö}{{\"O}}1 {Ü}{{\"U}}1
  {â}{{\^a}}1 {ê}{{\^e}}1 {î}{{\^i}}1 {ô}{{\^o}}1 {û}{{\^u}}1
  {Â}{{\^A}}1 {Ê}{{\^E}}1 {Î}{{\^I}}1 {Ô}{{\^O}}1 {Û}{{\^U}}1
  {œ}{{\oe}}1 {Œ}{{\OE}}1 {æ}{{\ae}}1 {Æ}{{\AE}}1 {ß}{{\ss}}1
  {ű}{{\H{u}}}1 {Ű}{{\H{U}}}1 {ő}{{\H{o}}}1 {Ő}{{\H{O}}}1
  {ç}{{\c c}}1 {Ç}{{\c C}}1 {ø}{{\o}}1 {å}{{\r a}}1 {Å}{{\r A}}1
  {€}{{\euro}}1 {£}{{\pounds}}1 {«}{{\guillemotleft}}1
  {»}{{\guillemotright}}1 {ñ}{{\~n}}1 {Ñ}{{\~N}}1 {¿}{{?`}}1
}

\AtBeginSection[ ]
{
\begin{frame}{Outline}
    \tableofcontents[currentsection]
\end{frame}
}

\title{Sköldpaddsprogrammering}
\date{2024/2025}

\maketitle{}

\section{Turtle}

\begin{frame}
	\frametitle{Sköldpaddsprogrammering}
	\framesubtitle{Turtle}
	
	I Python finns ett bibliotek som gör det möjligt att styra en padda över fönsterrutan.
	
	Biblioteket heter turtle och är väldigt lättanvänt.

\end{frame}

\subsection{Exempel}

\begin{frame}[fragile]
	\frametitle{Turtle}
	\framesubtitle{Exempel}
	
	Här är ett litet exempelprogram:
	
	\begin{lstlisting}
from turtle import * # Detta ger oss turtle

forward(100) # Går framåt 100 steg
right(90) # Svänger 90 grader åt höger
forward(100)
left(90) # Svänger 90 grader åt vänster
forward(100)
	\end{lstlisting}
	
	Om sköldpaddan börjar i mitten av skärmen och pekar åt höger, vart har den hamnat efter instruktionerna?
	
\end{frame}

\section{Kommandon}

\subsection{Förflyttning}

\begin{frame}
	\frametitle{Kommandon}
	\framesubtitle{Förflyttning}
	
	Här är en lista med kommandon som flyttar på paddan:
	
	\begin{tabular}{ll}
		\code{forward(x)} & Går x steg framåt\\
		\code{back(x)} & Går x steg bakåt\\
		\code{right(x)} & Roterar x grader medurs\\
		\code{left(x)} & Roterar x grader moturs\\
		\code{setposition((x,y))} & Placerar paddan i position (x,y)\\
		\code{setheading(x)} & Roterar paddan till x grader
	\end{tabular}
	
\end{frame}

\subsection{Andra kommandon}

\begin{frame}
	\frametitle{Kommandon}
	\framesubtitle{Andra kommandon}
	
	\begin{tabular}{ll}
		\code{penup()} & Slutar rita\\
		\code{pendown()} & Börjar rita\\
		\code{color("färg")} & Ändrar färgen\\
		\code{begin_fill()} & \\
		\code{end_fill()} & \\
		\code{fillcolor('färg')} & Ändrar den inre färgen\\
		\code{shape('turtle')} & Ändrar formen till en padda\\
		\code{clear()} & Tömmer skärmen
	\end{tabular}
	
\end{frame}

\section{Övningar}

\begin{frame}
	\frametitle{Övningar}
	
	\begin{enumerate}
		\item Ladda ner filen ''Start.py'' från Classroom
		\item Rita en kvadrat
		\item Rita en rektangel
		\item Rita en triangel
		\item Rita ett hus
		\item Rita en femhörning
		\item Rita en stjärna med fem spetsar
	\end{enumerate}

\end{frame}



\end{document}


