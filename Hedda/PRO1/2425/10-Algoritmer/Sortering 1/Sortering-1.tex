\documentclass[aspectratio=169]{beamer}

\mode<presentation>

\usepackage[utf8]{inputenc}
\usepackage[T1]{fontenc}	%makes å,ä,ö etc. proper symbols
\usepackage{amsmath}
\usepackage{graphicx}
\usepackage{xcolor}
\usepackage{listings}
\usepackage{multicol}
\usepackage{hyperref}
\usepackage[swedish]{babel}

\definecolor{LundaGroen}{RGB}{00,68,71}
\definecolor{StabilaLila}{RGB}{85,19,78}
\definecolor{VarmOrange}{RGB}{237,104,63}

\definecolor{MagnoliaRosa}{RGB}{251,214,209}
\definecolor{LundaHimmel}{RGB}{204,225,225}
\definecolor{LundaLjus}{RGB}{255,242,191}

\usefonttheme{serif}
\usetheme{malmoe}
\setbeamercolor{palette primary}{bg=LundaHimmel, fg=StabilaLila}
\setbeamercolor{palette quaternary}{bg=LundaGroen, fg=MagnoliaRosa}
\setbeamercolor{background canvas}{bg=LundaLjus}
\setbeamercolor{structure}{fg=LundaGroen}

\usepackage[many]{tcolorbox}

\newtcolorbox{cross}{blank,breakable,parbox=false,
  overlay={\draw[red,line width=5pt] (interior.south west)--(interior.north east);
    \draw[red,line width=5pt] (interior.north west)--(interior.south east);}}
    
\newcommand{\code}[1]{\colorbox{white}{\lstinline{#1}}}



\lstset{language=Python} 
\lstset{%language=[LaTeX]Tex,%C++,
    morekeywords={PassOptionsToPackage,selectlanguage,True,False},
    keywordstyle=\color{blue},%\bfseries,
    basicstyle=\small\ttfamily,
    %identifierstyle=\color{NavyBlue},
    commentstyle=\color{red}\ttfamily,
    stringstyle=\color{VarmOrange},
    numbers=left,%
    numberstyle=\scriptsize,%\tiny
    stepnumber=1,
    numbersep=8pt,
    showstringspaces=false,
    breaklines=true,
    %frameround=ftff,
    frame=single,
    belowcaptionskip=.75\baselineskip,
	tabsize=4,
	backgroundcolor=\color{white}
    %frame=L
}


\begin{document}

\lstset{literate=
  {á}{{\'a}}1 {é}{{\'e}}1 {í}{{\'i}}1 {ó}{{\'o}}1 {ú}{{\'u}}1
  {Á}{{\'A}}1 {É}{{\'E}}1 {Í}{{\'I}}1 {Ó}{{\'O}}1 {Ú}{{\'U}}1
  {à}{{\`a}}1 {è}{{\`e}}1 {ì}{{\`i}}1 {ò}{{\`o}}1 {ù}{{\`u}}1
  {À}{{\`A}}1 {È}{{\'E}}1 {Ì}{{\`I}}1 {Ò}{{\`O}}1 {Ù}{{\`U}}1
  {ä}{{\"a}}1 {ë}{{\"e}}1 {ï}{{\"i}}1 {ö}{{\"o}}1 {ü}{{\"u}}1
  {Ä}{{\"A}}1 {Ë}{{\"E}}1 {Ï}{{\"I}}1 {Ö}{{\"O}}1 {Ü}{{\"U}}1
  {â}{{\^a}}1 {ê}{{\^e}}1 {î}{{\^i}}1 {ô}{{\^o}}1 {û}{{\^u}}1
  {Â}{{\^A}}1 {Ê}{{\^E}}1 {Î}{{\^I}}1 {Ô}{{\^O}}1 {Û}{{\^U}}1
  {œ}{{\oe}}1 {Œ}{{\OE}}1 {æ}{{\ae}}1 {Æ}{{\AE}}1 {ß}{{\ss}}1
  {ű}{{\H{u}}}1 {Ű}{{\H{U}}}1 {ő}{{\H{o}}}1 {Ő}{{\H{O}}}1
  {ç}{{\c c}}1 {Ç}{{\c C}}1 {ø}{{\o}}1 {å}{{\r a}}1 {Å}{{\r A}}1
  {€}{{\euro}}1 {£}{{\pounds}}1 {«}{{\guillemotleft}}1
  {»}{{\guillemotright}}1 {ñ}{{\~n}}1 {Ñ}{{\~N}}1 {¿}{{?`}}1
}

\AtBeginSection[ ]
{
\begin{frame}{Innehåll}
    	\tableofcontents[currentsection]
\end{frame}
}

\title{Sortering}
\date{vt 23}
\author{Programmering 1}

\maketitle

\section{Sorteringsalgroritmer}

\begin{frame}
	\frametitle{Vad är en sorteringsalgoritm?}
	
	\begin{itemize}
		\item En sorteringsalgoritm är ett system för att sortera element
		\item Varierande komplexitet
		\item Varierande hastighet
		\item Varierande minnesanvändning
	\end{itemize}
	
\end{frame}

\begin{frame}
	\frametitle{Varför ska vi lära oss det här?}
	
	\begin{itemize}
		\item Det finns färdiga funktioner. Så varför?
		\pause
		\item Förståelse: det krävs mycket kunskap för att genomföra
		\item Utmanande: många algoritmer är tekniskt svåra
	\end{itemize}
	
\end{frame}

\section{Aktivitet}

\begin{frame}
	\frametitle{Aktivitet}
	\framesubtitle{Sortera kort}
	
	\begin{itemize}
		\item Du kommer att få tretton kort, Ess--Kung (1--13)
		\item Blanda korten
		\item Sortera korten i storleksordning
		\item Hur gjorde du?
		\item Varför gjorde du så?
	\end{itemize}
	
\end{frame}

\begin{frame}
	\frametitle{Min gissning på er algoritm}
	\framesubtitle{Insertion sort}
	
	\begin{enumerate}
		\item Första kortet är ''rätt''
		\item lägg på rätt sida om första
		\item stoppa in på rätt plats, bland de ordnade
		\item stoppa in på rätt plats, bland de ordnade
		\item ...
	\end{enumerate}	
	
\end{frame}

\section{Komplexitet}

\begin{frame}
	\frametitle{Komplexitet}
	
	\begin{itemize}
		\item Tidskomplexitet handlar förenklat om hur många jämförelser som behöver göras
		\item Om du loopar igenom en lista \(n\)-gånger är tidskompexiteten \(O(n)\)
		\item Har du en loop i en loop och loopar igenom \(n\) gånger för varje element i listan är tidskompexiteten \(O(n^2)\)
		\item Minneskomplexitet är hur mycket datorminne som går åt.
		\item Man brukar mäta två två sätt:
		\begin{enumerate}
			\item Total minnesåtgång
			\item Extra minnesåtgång
		\end{enumerate}
		\item Här nedan kommer exemplena bara vara med extra åtgång.
		\item Skapar man ingen extra lista för att sortera så är komplexiteten \(O(1)\)
	\end{itemize}
	
\end{frame}

\section{Sorteringsalgoritmer}

\begin{frame}
	\frametitle{Olika sorteringsalgoritmer}
	
	\begin{itemize}
		\item Bubble Sort
		\item Insertion Sort
		\item Selection Sort
		\item Quick Sort
	\end{itemize}
	
\end{frame}

\subsection{Bubble Sort}

\begin{frame}
	\frametitle{Bubble Sort}
	
	\begin{itemize}
		\item Intuitiv algoritm
		\item Lätt att implementera
		\item Tidskomplexitet: \(O(n^2)\) (väldigt långsam)
		\item Minneskomplexitet: \(O(1)\) (minneseffektiv)
	\end{itemize}
	
	\begin{enumerate}
		\item Jämför första elementet med sin granne till höger
		\item Ordna de två elementen i storleksordning
		\item Jämför nästa två element o.s.v.
		\item Upprepa för alla element i listan
	\end{enumerate}
	
\end{frame}

\subsection{Insertion Sort}

\begin{frame}
	\frametitle{Insertion Sort}
	
	\begin{itemize}
		\item Intuitiv algoritm
		\item Tidskomplexitet: \(O(n^2)\) (väldigt långsam)
		\item Minneskomplexitet: \(O(1)\) (minneseffektiv)
	\end{itemize}
	
	\begin{enumerate}
		\item Anta att första elementet är rätt
		\item Bubble sorta de två första elementen
		\item Bubble sorta tills det tredje är på rätt plats 
		\item Upprepa för alla element i listan
	\end{enumerate}
	
\end{frame}

\subsection{Selection Sort}

\begin{frame}
	\frametitle{Selection Sort}
	
	\begin{itemize}
		\item Intuitiv algoritm
		\item Tidskomplexitet: \(O(n^2)\) (väldigt långsam)
		\item Minneskomplexitet: \(O(1)\) (minneseffektiv)
	\end{itemize}
	
	\begin{enumerate}
		\item Hitta det största elementet i listan
		\item Placera det sist
		\item Hitta näst största elementet och placera näst sist
		\item Upprepa för alla element i listan
	\end{enumerate}

\end{frame}

\subsection{Quick Sort}

\begin{frame}
	\frametitle{Quick Sort}
	
	\begin{itemize}
		\item Rekursiv algoritm
		\item Tidskomplexitet: \(nO(n)\) (väldigt långsam)
		\item Minneskomplexitet: \(O(1)\) (minneseffektiv)
	\end{itemize}
	
	\begin{enumerate}
		\item Välj ett element i listan
		\item Placera alla element mindre än det valda före och alla större efter
		\item Välj ett element ur den nedre halvan och upprepa
		\item Välj ett element ur övre halvan och upprepa
	\end{enumerate}

\end{frame}

\section{Övningar}

\begin{frame}
	\frametitle{Övningar}
	
	\begin{enumerate}
		\item Implementera Bubble Sort
		\item Implementera Insertion Sort
		\item Implementera Selection Sort
		\item Implementera Quick Sort
		\item Implementera Stalin Sort
	\end{enumerate}

\end{frame}


\end{document}