\documentclass[aspectratio=169]{beamer}

\mode<presentation>

\usepackage[utf8]{inputenc}
\usepackage[T1]{fontenc}	%makes å,ä,ö etc. proper symbols
\usepackage{amsmath}
\usepackage{graphicx}
\usepackage{xcolor}
\usepackage{listings}
\usepackage{multicol}
\usepackage{hyperref}
\usepackage[swedish]{babel}


\definecolor{LundaGroen}{RGB}{00,68,71}
\definecolor{StabilaLila}{RGB}{85,19,78}
\definecolor{VarmOrange}{RGB}{237,104,63}

\definecolor{MagnoliaRosa}{RGB}{251,214,209}
\definecolor{LundaHimmel}{RGB}{204,225,225}
\definecolor{LundaLjus}{RGB}{255,242,191}

\usefonttheme{serif}
\usetheme{malmoe}
\setbeamercolor{palette primary}{bg=VarmOrange}
\setbeamercolor{palette quaternary}{bg=LundaGroen}
\setbeamercolor{background canvas}{bg=LundaLjus}
\setbeamercolor{structure}{fg=LundaGroen}

\usepackage[many]{tcolorbox}

\newtcolorbox{cross}{blank,breakable,parbox=false,
  overlay={\draw[red,line width=5pt] (interior.south west)--(interior.north east);
    \draw[red,line width=5pt] (interior.north west)--(interior.south east);}}



\lstset{language=Python} 
\lstset{%language=[LaTeX]Tex,%C++,
    morekeywords={PassOptionsToPackage,selectlanguage,True,False},
    keywordstyle=\color{blue},%\bfseries,
    basicstyle=\small\ttfamily,
    %identifierstyle=\color{NavyBlue},
    commentstyle=\color{red}\ttfamily,
    stringstyle=\color{VarmOrange},
    numbers=left,%
    numberstyle=\scriptsize,%\tiny
    stepnumber=1,
    numbersep=8pt,
    showstringspaces=false,
    breaklines=true,
    %frameround=ftff,
    frame=single,
    belowcaptionskip=.75\baselineskip,
	tabsize=4,
	backgroundcolor=\color{white}
    %frame=L
}

\newcommand{\code}[1]{\colorbox{white}{\lstinline{#1}}}

\begin{document}

\lstset{literate=
  {á}{{\'a}}1 {é}{{\'e}}1 {í}{{\'i}}1 {ó}{{\'o}}1 {ú}{{\'u}}1
  {Á}{{\'A}}1 {É}{{\'E}}1 {Í}{{\'I}}1 {Ó}{{\'O}}1 {Ú}{{\'U}}1
  {à}{{\`a}}1 {è}{{\`e}}1 {ì}{{\`i}}1 {ò}{{\`o}}1 {ù}{{\`u}}1
  {À}{{\`A}}1 {È}{{\'E}}1 {Ì}{{\`I}}1 {Ò}{{\`O}}1 {Ù}{{\`U}}1
  {ä}{{\"a}}1 {ë}{{\"e}}1 {ï}{{\"i}}1 {ö}{{\"o}}1 {ü}{{\"u}}1
  {Ä}{{\"A}}1 {Ë}{{\"E}}1 {Ï}{{\"I}}1 {Ö}{{\"O}}1 {Ü}{{\"U}}1
  {â}{{\^a}}1 {ê}{{\^e}}1 {î}{{\^i}}1 {ô}{{\^o}}1 {û}{{\^u}}1
  {Â}{{\^A}}1 {Ê}{{\^E}}1 {Î}{{\^I}}1 {Ô}{{\^O}}1 {Û}{{\^U}}1
  {œ}{{\oe}}1 {Œ}{{\OE}}1 {æ}{{\ae}}1 {Æ}{{\AE}}1 {ß}{{\ss}}1
  {ű}{{\H{u}}}1 {Ű}{{\H{U}}}1 {ő}{{\H{o}}}1 {Ő}{{\H{O}}}1
  {ç}{{\c c}}1 {Ç}{{\c C}}1 {ø}{{\o}}1 {å}{{\r a}}1 {Å}{{\r A}}1
  {€}{{\euro}}1 {£}{{\pounds}}1 {«}{{\guillemotleft}}1
  {»}{{\guillemotright}}1 {ñ}{{\~n}}1 {Ñ}{{\~N}}1 {¿}{{?`}}1
}

\AtBeginSection[ ]
{
\begin{frame}{Outline}
    \tableofcontents[currentsection]
\end{frame}
}


\title{While-loopar}
\date{2024/25}
\author{Programmering 1}

\maketitle

\section{Upprepa}

\subsection{if-satsen}

\begin{frame}[fragile]
	\frametitle{Upprepa}
	\framesubtitle{If-satsen}
	
	Vi har tidigare kollat på \code{if}-satser: Om något, gör något. Typ:
	
	\begin{lstlisting}
tal = input("Skriv ett tal: ")
tal = int(tal)

if tal > 10:
    print("Ditt tal är skitstort!")
	\end{lstlisting}
	
	Det vi ska kolla på idag är: ''Så länge något, gör något''

\end{frame}


\subsection{while}

\begin{frame}[fragile]
	\frametitle{Upprepa}
	\framesubtitle{Så länge något}
	
	Tänk att vi har kodsnutten:
	
	\begin{lstlisting}
tal = input("Skriv ett heltal: ")
tal = int(tal)

if tal != 0:
    dubbel = tal*2
    print(dubbel, "är dubbelt så mycket som", tal)
	\end{lstlisting}
	
	Och så vill vi att programmet ska fortsätta att fråga tills vi är nöjda (och skriver 0).
	
\end{frame}

\begin{frame}[fragile]
	\frametitle{Upprepa}
	\framesubtitle{Så länge något}
	
	\begin{lstlisting}
tal = input("Skriv ett heltal: ")
tal = int(tal) # Konverterar till heltal

while tal != 0:
    dubbel = tal*2 # Dubblar talet
    print(dubbel, "är dubbelt så mycket som", tal)
    tal = input("Skriv ett heltal: ")
    tal = int(tal)
	\end{lstlisting}
	
\end{frame}

\begin{frame}[fragile]
	\frametitle{Upprepa}
	\framesubtitle{while}
	
	Vi kan också ha en kod som ska vi redan innan vet hur många gånger den ska upprepa. Då behöver vi en räknare, eller en \textit{iterator}. Det vanligaste är att man döper den variabeln till \code{i}.
	
	\begin{lstlisting}
i = 0

while i < 10:
    print(i)
    i += 1 # Jätteviktigt
	\end{lstlisting}
	
	När man gör en \code{while}-loop är det viktigt att man uppdaterar någon av variablerna i villkoret. Annars fastnar man i en oändlig loop.

\end{frame}

\begin{frame}[fragile]
	\frametitle{Upprepa}
	\framesubtitle{Antal iterationer}
	
	Vi kan låta användaren bestämma antalet iterationer:
	
	\begin{lstlisting}
antal = input("Hur många tal vill du ha? ")
antal = int(antal)
i = 0

while i < antal:
    print(2**i)
	\end{lstlisting}
	
\end{frame}

\subsection{Loop i en loop}

\begin{frame}[fragile]
	\frametitle{Upprepa}
	\frametitle{Loopa i en loop}
	
	\begin{lstlisting}
i = 1
while i < 11:
    j = 1
    while j < 11:
        print(i, j, i*j, end=" | ") # !!
        j += 1
    i += 1
    print()
	\end{lstlisting}
	
	Vad händer på rad 5?

\end{frame}

\section{Speciella kommandon}

\begin{frame}
	\frametitle{Speciella kommandon}
	
	Det finns tre speciella kommandon som är kopplade till loopar i Python:
	
	\begin{enumerate}
		\item \code{pass}
		\item \code{continue}
		\item \code{break}
	\end{enumerate}
	
\end{frame}

\subsection{Pass}

\begin{frame}[fragile]
	\frametitle{Speciella kommandon}
	\framesubtitle{Pass}
	
	Kommandot \code{pass} används mest under tiden man bygger sitt program och vill testköra utan att behöva göra klart alla delar.
	
	\begin{lstlisting}
i = 0
while i < 10:
    if i%2==0:
        pass
    i+=1
	\end{lstlisting}
	
	Här vill vi att något särskilt ska hända när \code{i} är delbart med två, men vi har inte börjat skriva den delen än. Då kan vi använda \code{pass} direkt efter \code{if}-satsen för att förhindra ett \code{error}.
	
\end{frame}

\subsection{Continue}

\begin{frame}[fragile]
	\frametitle{Speciella kommandon}
	\framesubtitle{Continue}
	
	Kommandot \code{continue} kan användas för att få loopen att fortsätta med nästa iteration direkt. Detta är vanligare att använda i \code{for}-loopar som kommer senare.
	
	\begin{lstlisting}
tal = int(input("Skriv: "))
while tal != -1:
    if tal == 0:
        tal = int(input("Skriv inte 0: "))
        continue
    print(1/tal)
    tal = int(input(tal))
	\end{lstlisting}

	Eftersom vi inte får dela med noll så behöver vi ta hand om det fallet. Här hade det också funnits andra lösningar.
	
\end{frame}


\subsection{Break}

\begin{frame}[fragile]
	\frametitle{Speciella kommandon}
	\framesubtitle{Break}
	
	Kommandot \code{break} kan användas för att avsluta loopen under en iteration.
	
	\begin{lstlisting}
tal = int(input("Skriv: "))
while tal != -1:
    if tal == 0:
        break
    print(1/tal)
    tal = int(input(tal))
	\end{lstlisting}

	Helst ska man försöka undvika att använda \code{break} för att avsluta en loop.
	
\end{frame}


\section{Övningar}

\subsection{Blad 1}

\begin{frame}
	\frametitle{Övningar}
	\framesubtitle{Blad 1}
	
	\begin{enumerate}
		\item Skriv en kod som skriver ut alla tal från 0 till 100
		\item Skriv en kod som skriver ut talen 0.1, 0.2, 0.3 ... 1.0
		\item Skriv en kod som skriver ut varannat tal från 10 till -10.
		\item Skriv ut ettan till tolvans gångertabell.
		\item Skriv en bit kod som skriver ut Fibonacci-serien upp till det hundrade talet (1, 1, 2, 3, 5, 8, 13...)
		\item Skriv en loop som skriver ut A--Z (tips kolla upp \href{https://docs.python.org/3/library/functions.html\#chr}{\code{chr()}})
		\item Skriv ett program som frågar efter ett lösenord. Om användaren skriver \texttt{admin} ska det svara \texttt{Lösenord accepterat} annars ska det fråga efter lösenordet på nytt.
	\end{enumerate}

\end{frame}

\subsection{Blad 2}

\begin{frame}[fragile]
	\frametitle{Övningar}
	\framesubtitle{Blad 2}
	
	Skriv ett program som frågar efter:
	
	\begin{itemize}
		\item Lånebelopp
		\item Räntesats
		\item Amortering/år
	\end{itemize}
	
	Och som sen skriver ut:
	
	\begin{itemize}
		\item Tid för att betala tillbaka lånet
		\item Totalt betalad ränta
	\end{itemize}
	
	Ex:
	
	\begin{lstlisting}
Hur mycket ska du låna? 2500000
Vad är räntesatsen (i %)?  2
Hur mycket ska du amortera? 60000
Det kommer att ta dig 42 år att betala tillbaka lånet
Du kommer att ha betalat 1066800 kr i ränta
	\end{lstlisting}
	
	
\end{frame}





\end{document}