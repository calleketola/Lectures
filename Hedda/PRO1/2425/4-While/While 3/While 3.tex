\documentclass[aspectratio=169]{beamer}

\mode<presentation>

\usepackage[utf8]{inputenc}
\usepackage[T1]{fontenc}	%makes å,ä,ö etc. proper symbols
\usepackage{amsmath}
\usepackage{graphicx}
\usepackage{xcolor}
\usepackage{listings}
\usepackage{multicol}
\usepackage{hyperref}
\usepackage[swedish]{babel}


\definecolor{LundaGroen}{RGB}{00,68,71}
\definecolor{StabilaLila}{RGB}{85,19,78}
\definecolor{VarmOrange}{RGB}{237,104,63}

\definecolor{MagnoliaRosa}{RGB}{251,214,209}
\definecolor{LundaHimmel}{RGB}{204,225,225}
\definecolor{LundaLjus}{RGB}{255,242,191}

\usefonttheme{serif}
\usetheme{malmoe}
\setbeamercolor{palette primary}{bg=VarmOrange}
\setbeamercolor{palette quaternary}{bg=LundaGroen}
\setbeamercolor{background canvas}{bg=LundaLjus}
\setbeamercolor{structure}{fg=LundaGroen}

\usepackage[many]{tcolorbox}

\newtcolorbox{cross}{blank,breakable,parbox=false,
  overlay={\draw[red,line width=5pt] (interior.south west)--(interior.north east);
    \draw[red,line width=5pt] (interior.north west)--(interior.south east);}}



\lstset{language=Python} 
\lstset{%language=[LaTeX]Tex,%C++,
    morekeywords={PassOptionsToPackage,selectlanguage,True,False},
    keywordstyle=\color{blue},%\bfseries,
    basicstyle=\small\ttfamily,
    %identifierstyle=\color{NavyBlue},
    commentstyle=\color{red}\ttfamily,
    stringstyle=\color{VarmOrange},
    numbers=left,%
    numberstyle=\scriptsize,%\tiny
    stepnumber=1,
    numbersep=8pt,
    showstringspaces=false,
    breaklines=true,
    %frameround=ftff,
    frame=single,
    belowcaptionskip=.75\baselineskip,
	tabsize=4,
	backgroundcolor=\color{white}
    %frame=L
}

\newcommand{\code}[1]{\colorbox{white}{\lstinline{#1}}}

\begin{document}

\lstset{literate=
  {á}{{\'a}}1 {é}{{\'e}}1 {í}{{\'i}}1 {ó}{{\'o}}1 {ú}{{\'u}}1
  {Á}{{\'A}}1 {É}{{\'E}}1 {Í}{{\'I}}1 {Ó}{{\'O}}1 {Ú}{{\'U}}1
  {à}{{\`a}}1 {è}{{\`e}}1 {ì}{{\`i}}1 {ò}{{\`o}}1 {ù}{{\`u}}1
  {À}{{\`A}}1 {È}{{\'E}}1 {Ì}{{\`I}}1 {Ò}{{\`O}}1 {Ù}{{\`U}}1
  {ä}{{\"a}}1 {ë}{{\"e}}1 {ï}{{\"i}}1 {ö}{{\"o}}1 {ü}{{\"u}}1
  {Ä}{{\"A}}1 {Ë}{{\"E}}1 {Ï}{{\"I}}1 {Ö}{{\"O}}1 {Ü}{{\"U}}1
  {â}{{\^a}}1 {ê}{{\^e}}1 {î}{{\^i}}1 {ô}{{\^o}}1 {û}{{\^u}}1
  {Â}{{\^A}}1 {Ê}{{\^E}}1 {Î}{{\^I}}1 {Ô}{{\^O}}1 {Û}{{\^U}}1
  {œ}{{\oe}}1 {Œ}{{\OE}}1 {æ}{{\ae}}1 {Æ}{{\AE}}1 {ß}{{\ss}}1
  {ű}{{\H{u}}}1 {Ű}{{\H{U}}}1 {ő}{{\H{o}}}1 {Ő}{{\H{O}}}1
  {ç}{{\c c}}1 {Ç}{{\c C}}1 {ø}{{\o}}1 {å}{{\r a}}1 {Å}{{\r A}}1
  {€}{{\euro}}1 {£}{{\pounds}}1 {«}{{\guillemotleft}}1
  {»}{{\guillemotright}}1 {ñ}{{\~n}}1 {Ñ}{{\~N}}1 {¿}{{?`}}1
}

\title{Övningar}
\date{2024/25}
\author{Programmering 1}

\maketitle

\section{Övningar}

\subsection{Blad 1}

\begin{frame}
	\frametitle{Övningar}
	\framesubtitle{Blad 1}
	
	\begin{enumerate}
		 \item En sjö har en area på 1000 m$^2$. I ett hörn av sjön ligger alger med en utbredning på 13 m$^2$. Varje vecka växer algerna och täcker 27 m$^2$ mer av sjön. Efter hur många veckor är sjön täckt med alger?
		 \item Grannsjön, som är 1337 m$^2$ stor, är redan täckt med alger. Därför har man hällt i \textit{Algdödarmedel X} i sjön. Varje vecka minskar ytan algerna täcker med 3~\%. Efter hur många veckor har hälften av algerna försvunnit?
		 \item Skriv ett program som tar emot ett tal \(n\) och räknar ut följande summa: \[1+4+9+16+...+n^2\]
		 \item Räkna nu ut följande summa istället: \[\dfrac{1}{1}+\dfrac{1}{2}+\dfrac{1}{4}+\dfrac{1}{8}+...+\dfrac{1}{2^n}\]
	\end{enumerate}

\end{frame}

\subsection{Blad 2}

\begin{frame}
	\frametitle{Övningar}
	\framesubtitle{Blad 2}
	
	\begin{enumerate}%[resume]
		\setcounter{enumi}{4}
		\item I rollspelet Dungeons and Dragons rullar man en 20-sidig tärning för att se om man lyckas med något. Utöver vad tärningen visar tar man plus sin bonus (som kan vara negativ) och lägger till det på tärningsresultatet. Skriv ett program som rullar en 20-sidig tärning och frågar efter din bonus (oftast ett tal mellan -2 och +20) och sen frågar efter svårigheten att lyckas. Om tärningen plus bonusen blir högre eller lika med svårigheten har man lyckats.
		\item I rollspelet Eon så kastade man istället flera sexsidiga tärningar och räknade ihop resultatet, men den regeln att om man slog en sexa så slog man om den tärningen och kastade en till. Testa att kasta skitmånga tärningar. Vad blir medelvärdet för en tärning? (Räkna med att totalen är lika med antalet tärningar du kastade från början).
	\end{enumerate}
		
\end{frame}

\subsection{Blad 3}

\begin{frame}
	\frametitle{Övningar}
	\framesubtitle{Blad 3}
	
	\begin{enumerate}
		\setcounter{enumi}{6}
		\item I rollspelen Svärdets sång, Mutant år noll m.fl. så rullar man precis som i Eon flera sexsidiga tärningar (oftast mellan 2 och 15 stycken) för att se om man lyckas. Tillskillnad från Eon så räknar man bara antalet sexor. Skriv ett program som rullar \(n\) tärningar och skriver ut hur många sexor man får.
		\item Skriv ett program som tar \(a\) delat med \(b\) och försöker hitta en approximation på \(\pi\) genom att uppdatera \(a\) och \(b\) så att du kommer närmare och närmare \(\pi\). Skriv sen ut differensen mellan \(\pi\) och ditt bråk.
		\item I Dungeons and Dragons så får man ibland rulla två 20-sidiga tärningar och välja den tärning som slog högst. Rulla två 20-sidiga tärningar 1~000~000~gånger. Vad är medelvärdet för den bästa tärningen i varje kast?
		\item Gör samma experiment igen men välj den sämre tärningen.
	\end{enumerate}


\end{frame}



\end{document}