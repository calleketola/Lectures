\documentclass[aspectratio=169]{beamer}

\mode<presentation>

\usepackage[utf8]{inputenc}
\usepackage[T1]{fontenc}	%makes å,ä,ö etc. proper symbols
\usepackage{amsmath}
\usepackage{graphicx}
\usepackage{xcolor}
\usepackage{listings}
\usepackage{multicol}
\usepackage{hyperref}
\usepackage[swedish]{babel}

\definecolor{LundaGroen}{RGB}{00,68,71}
\definecolor{StabilaLila}{RGB}{85,19,78}
\definecolor{VarmOrange}{RGB}{237,104,63}

\definecolor{MagnoliaRosa}{RGB}{251,214,209}
\definecolor{LundaHimmel}{RGB}{204,225,225}
\definecolor{LundaLjus}{RGB}{255,242,191}

\usefonttheme{serif}
\usetheme{malmoe}
\setbeamercolor{palette primary}{bg=LundaHimmel, fg=StabilaLila}
\setbeamercolor{palette quaternary}{bg=LundaGroen, fg=MagnoliaRosa}
\setbeamercolor{background canvas}{bg=LundaLjus}
\setbeamercolor{structure}{fg=LundaGroen}

\usepackage[many]{tcolorbox}

\newtcolorbox{cross}{blank,breakable,parbox=false,
  overlay={\draw[red,line width=5pt] (interior.south west)--(interior.north east);
    \draw[red,line width=5pt] (interior.north west)--(interior.south east);}}
    
\newcommand{\code}[1]{\colorbox{white}{\lstinline{#1}}}



\lstset{language=Python} 
\lstset{%language=[LaTeX]Tex,%C++,
    morekeywords={PassOptionsToPackage,selectlanguage,True,False},
    keywordstyle=\color{blue},%\bfseries,
    basicstyle=\small\ttfamily,
    %identifierstyle=\color{NavyBlue},
    commentstyle=\color{red}\ttfamily,
    stringstyle=\color{VarmOrange},
    numbers=left,%
    numberstyle=\scriptsize,%\tiny
    stepnumber=1,
    numbersep=8pt,
    showstringspaces=false,
    breaklines=true,
    %frameround=ftff,
    frame=single,
    belowcaptionskip=.75\baselineskip,
	tabsize=4,
	backgroundcolor=\color{white}
    %frame=L
}


\begin{document}

\lstset{literate=
  {á}{{\'a}}1 {é}{{\'e}}1 {í}{{\'i}}1 {ó}{{\'o}}1 {ú}{{\'u}}1
  {Á}{{\'A}}1 {É}{{\'E}}1 {Í}{{\'I}}1 {Ó}{{\'O}}1 {Ú}{{\'U}}1
  {à}{{\`a}}1 {è}{{\`e}}1 {ì}{{\`i}}1 {ò}{{\`o}}1 {ù}{{\`u}}1
  {À}{{\`A}}1 {È}{{\'E}}1 {Ì}{{\`I}}1 {Ò}{{\`O}}1 {Ù}{{\`U}}1
  {ä}{{\"a}}1 {ë}{{\"e}}1 {ï}{{\"i}}1 {ö}{{\"o}}1 {ü}{{\"u}}1
  {Ä}{{\"A}}1 {Ë}{{\"E}}1 {Ï}{{\"I}}1 {Ö}{{\"O}}1 {Ü}{{\"U}}1
  {â}{{\^a}}1 {ê}{{\^e}}1 {î}{{\^i}}1 {ô}{{\^o}}1 {û}{{\^u}}1
  {Â}{{\^A}}1 {Ê}{{\^E}}1 {Î}{{\^I}}1 {Ô}{{\^O}}1 {Û}{{\^U}}1
  {œ}{{\oe}}1 {Œ}{{\OE}}1 {æ}{{\ae}}1 {Æ}{{\AE}}1 {ß}{{\ss}}1
  {ű}{{\H{u}}}1 {Ű}{{\H{U}}}1 {ő}{{\H{o}}}1 {Ő}{{\H{O}}}1
  {ç}{{\c c}}1 {Ç}{{\c C}}1 {ø}{{\o}}1 {å}{{\r a}}1 {Å}{{\r A}}1
  {€}{{\euro}}1 {£}{{\pounds}}1 {«}{{\guillemotleft}}1
  {»}{{\guillemotright}}1 {ñ}{{\~n}}1 {Ñ}{{\~N}}1 {¿}{{?`}}1
}

\AtBeginSection[ ]
{
\begin{frame}{Innehåll}
    	\tableofcontents[currentsection]
\end{frame}
}

\title{Pseudokod}
\date{vt 25}
\author{Programmering 1}

\maketitle

\section{Pseudokod}

\begin{frame}
Pseudokod är lite förenklat kod som en människa kan läsa.

Eftersom olika programmeringsspråk struktureras lite olika så finns det inget bestämt sätt att skriva pseudokod.
\end{frame}

\begin{frame}
Det finns lite olika idéer om hur pseudokod ska skrivas. Somliga vill ha det väldigt nära språket man programmerar i medan andra vill ha det lite mer frånkopplat och allmänt.
\end{frame}

\subsection{Ett exempel}

\begin{frame}[fragile]

\begin{lstlisting}
a = 1
for i in range(10):
  a = a*i
\end{lstlisting}

\begin{lstlisting}
a <- 1
repeat x10:
  a <- a*i
\end{lstlisting}

Från pseudokoden kan vi inte säga om det är en \code{while}-loop eller en \code{for}-loop

\end{frame}

\subsection{Hitta längsta ordet}

\begin{frame}[fragile]
	\frametitle{Hitta längsta ordet}
	\framesubtitle{Pseudokod}

	Följande algoritm hittar det längsta ordet i en lista med text.

	\begin{lstlisting}[language=TeX]
Ta emot en lista
Skapa variabeln "längst" och sätt till första ordet i listan
Skapa variabeln "i" och sätt till 1
Upprepa så länge "i" är mindre än längden på listan
    Om längden av ordet på plats i är större än längden av "längst"
        Sätt "längst" till ordet på plats "i"
    Öka variabeln "i" med 1
Skriv ut variabeln "längst"
	\end{lstlisting}

\end{frame}

\begin{frame}[fragile]
	\frametitle{Hitta längsta ordet}
	\framesubtitle{Strukturerad engelska}

	Strukturerad engelska är ett mellanting mellan Pseudokod och vanlig engelska:

	\begin{lstlisting}[language=TeX]
GET lista
SET längst TO first element
SET i TO 1
DO WHILE i LESS THAN length of lista
    IF length of lista at position i MORE THAN length of längst
        THEN SET längst TO lista at position i
    END IF
    ADD 1 TO i
PRINT längst
	\end{lstlisting}
	
\end{frame}

\begin{frame}[fragile]
	\frametitle{Hitta längsta ordet}
	\framesubtitle{Tolkning}

	Här är en tolkning av de två Pseudokoderna tidigare

	\begin{lstlisting}
lista = [...] # Vi antar att vi får en lista med ord
längst = lista[0] # Längden av första ordet
i = 1
while i < len(lista):
    if len(lista[i]) > len(längst):
        längst = lista[i]
    i += 1
print(längst) # Skriver ut längsta ordet
	\end{lstlisting}

\end{frame}

\subsection{Bubble Sort}

\begin{frame}[fragile]
\frametitle{Bubblesort, steg 1}
\framesubtitle{Pseudokod}

Följande algoritm beskriver det första steget i algoritmen Bubble Sort:

\begin{lstlisting}[language=TeX]
Ta emot en lista
Skapa variabeln i och sätt till 0
Upprepa så länge i är mindre än längden på listan minus 1:
  Om listan plats i är större än listan plats i+1:
    Byt plats på listan plats i med listan plats i+1
  Öka i med 1
\end{lstlisting}

\end{frame}

\section{Strukturerad engelska}

\begin{frame}[fragile]
	\frametitle{Bubblesort, steg 1}
	\framesubtitle{Strukturerad engelska}

Strukturerad engelska är ett mellanting mellan Pseudokod och vanlig engelska:

\begin{lstlisting}[language=TeX]
BUBBLE SORT STEP 1
GET lista
SET i TO 0
DO WHILE i LESS THAN length of lista minus 1
	IF lista position i is greater than lista position i+1
		THEN swap lista position i with lista position i+1 
	END IF
	ADD 1 TO i
END WHILE
\end{lstlisting}

\end{frame}

\section{Tolkningen}

\begin{frame}[fragile]
	\frametitle{Bubblesort, steg 1}
	\framesubtitle{Pythonkod}

\begin{lstlisting}
def BubbleSort(lista):
  i = 0
  while i < len(lista)-1:
    if lista[i] > lista[i+1]:
      lista[i], lista[i+1] = lista[i+1], lista[i]
    i += 1
\end{lstlisting}

\end{frame}

\section{Strukturerad engelska}

\begin{frame}[fragile]
	\frametitle{Bubblesort, hela}
	\framesubtitle{Strukturerad engelska}

\begin{lstlisting}[language=TeX]
BUBBLE SORT HELA
GET lista
SET j TO 0
DO WHILE j LESS than length of lista minus 1
    SET i TO 0
    DO WHILE i LESS than length of lista minus 1
	    IF lista position i is greater than lista position i+1
		    THEN swap lista position i with lista position i+1 
	    END IF
	    ADD 1 to i
    END WHILE
    ADD 1 to j
END WHILE
\end{lstlisting}

\end{frame}

\section{Tolkningen}

\begin{frame}[fragile]
	\frametitle{Bubblesort, hela}
	\framesubtitle{Pythonkod}

\begin{lstlisting}
def BubbleSort(lista):
    j = 0
    while i < len(lista)-1:
        i = 0
        while i < len(lista)-1:
            if lista[i] > lista[i+1]:
                lista[i], lista[i+1] = lista[i+1], lista[i]
            i += 1
        j += 1
\end{lstlisting}

\end{frame}

\section{Övningar}

\begin{frame}
	\frametitle{Övningar}
	
	\begin{itemize}
		\item Utgå från filen \texttt{pseudokod-1.py} på Classroom
	\end{itemize}	

\end{frame}


\end{document}