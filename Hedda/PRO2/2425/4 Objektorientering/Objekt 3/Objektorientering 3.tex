\documentclass[aspectratio=169]{beamer}

\mode<presentation>

\usepackage[utf8]{inputenc}
\usepackage[T1]{fontenc}	%makes å,ä,ö etc. proper symbols
\usepackage{amsmath}
\usepackage{graphicx}
\usepackage{xcolor}
\usepackage{listings}
\usepackage{multicol}
\usepackage{multirow}
\usepackage{hyperref}
\usepackage[swedish]{babel}

\definecolor{LundaGroen}{RGB}{00,68,71}
\definecolor{StabilaLila}{RGB}{85,19,78}
\definecolor{VarmOrange}{RGB}{237,104,63}

\definecolor{MagnoliaRosa}{RGB}{251,214,209}
\definecolor{LundaHimmel}{RGB}{204,225,225}
\definecolor{LundaLjus}{RGB}{255,242,191}

\usefonttheme{serif}
\usetheme{malmoe}
\setbeamercolor{palette primary}{bg=LundaHimmel, fg=StabilaLila}
\setbeamercolor{palette quaternary}{bg=LundaGroen, fg=MagnoliaRosa}
\setbeamercolor{background canvas}{bg=LundaLjus}
\setbeamercolor{structure}{fg=LundaGroen}

\usepackage[many]{tcolorbox}

\newtcolorbox{cross}{blank,breakable,parbox=false,
  overlay={\draw[red,line width=5pt] (interior.south west)--(interior.north east);
    \draw[red,line width=5pt] (interior.north west)--(interior.south east);}}
    
\newcommand{\code}[1]{\colorbox{white}{\lstinline{#1}}}



\lstset{language=Python} 
\lstset{%language=[LaTeX]Tex,%C++,
    morekeywords={PassOptionsToPackage,selectlanguage,True,False},
    keywordstyle=\color{blue},%\bfseries,
    basicstyle=\small\ttfamily,
    %identifierstyle=\color{NavyBlue},
    commentstyle=\color{red}\ttfamily,
    stringstyle=\color{VarmOrange},
    numbers=left,%
    numberstyle=\scriptsize,%\tiny
    stepnumber=1,
    numbersep=8pt,
    showstringspaces=false,
    breaklines=true,
    %frameround=ftff,
    frame=single,
    belowcaptionskip=.75\baselineskip,
	tabsize=4,
	backgroundcolor=\color{white}
    %frame=L
}
\lstset{
	escapeinside={(*@}{@*)}
}


\begin{document}

\lstset{literate=
  {á}{{\'a}}1 {é}{{\'e}}1 {í}{{\'i}}1 {ó}{{\'o}}1 {ú}{{\'u}}1
  {Á}{{\'A}}1 {É}{{\'E}}1 {Í}{{\'I}}1 {Ó}{{\'O}}1 {Ú}{{\'U}}1
  {à}{{\`a}}1 {è}{{\`e}}1 {ì}{{\`i}}1 {ò}{{\`o}}1 {ù}{{\`u}}1
  {À}{{\`A}}1 {È}{{\'E}}1 {Ì}{{\`I}}1 {Ò}{{\`O}}1 {Ù}{{\`U}}1
  {ä}{{\"a}}1 {ë}{{\"e}}1 {ï}{{\"i}}1 {ö}{{\"o}}1 {ü}{{\"u}}1
  {Ä}{{\"A}}1 {Ë}{{\"E}}1 {Ï}{{\"I}}1 {Ö}{{\"O}}1 {Ü}{{\"U}}1
  {â}{{\^a}}1 {ê}{{\^e}}1 {î}{{\^i}}1 {ô}{{\^o}}1 {û}{{\^u}}1
  {Â}{{\^A}}1 {Ê}{{\^E}}1 {Î}{{\^I}}1 {Ô}{{\^O}}1 {Û}{{\^U}}1
  {œ}{{\oe}}1 {Œ}{{\OE}}1 {æ}{{\ae}}1 {Æ}{{\AE}}1 {ß}{{\ss}}1
  {ű}{{\H{u}}}1 {Ű}{{\H{U}}}1 {ő}{{\H{o}}}1 {Ő}{{\H{O}}}1
  {ç}{{\c c}}1 {Ç}{{\c C}}1 {ø}{{\o}}1 {å}{{\r a}}1 {Å}{{\r A}}1
  {€}{{\euro}}1 {£}{{\pounds}}1 {«}{{\guillemotleft}}1
  {»}{{\guillemotright}}1 {ñ}{{\~n}}1 {Ñ}{{\~N}}1 {¿}{{?`}}1
}

\AtBeginSection[ ]
{
\begin{frame}{Innehåll}
    	\tableofcontents[currentsection]
\end{frame}
}

\title{Objektorientering 3}
\date{vt 25}

\maketitle

\section{Klasser}

\subsection{Repetition}

\begin{frame}
	\frametitle{Vad är en klass?}

	\begin{itemize}
		\item \textit{Klasser} i programmering är en sorts \textit{abstraktion}
		\item Används för att \textit{modellera} verkligheten
	\end{itemize}

\end{frame}

\begin{frame}
	\frametitle{Objektifiering}
	
	Om vi skulle vilja modellera en sköldpadda så behöver vi bestämma vilka egenskaper en sköldpadda kan ha:
	
	\begin{itemize}
		\item Attribut, så som färg, födelseår, namn m.m.
		\item Förmågor, så som gå, äta m.m.
	\end{itemize}

\end{frame}

\section{Övningen}

\subsection{Mål}

\begin{frame}
	\frametitle{Dagens mål}
	
	\begin{itemize}
		\item Under dagens pass ska vi skapa en sköldpadda som kan gå över skärmen och styras för att äta upp godisbitar
		\item Vi kommer att behöva skapa två klasser för detta
		\begin{enumerate}
			\item Sköldpadda
			\item Godis
		\end{enumerate}
		\item Båda klasserna har en del gemensamt:
			\begin{itemize}
				\item Position på skärmen
				\item Förmåga att ritas ut
			\end{itemize}
	\end{itemize}
	
\end{frame}

\subsection{Klassdiagram}

\begin{frame}
	\frametitle{Klassen sköldpadda}
	
	\begin{itemize}
		\item Klassen sköldpadda kommer att behöva innehålla följande:
	\end{itemize}
	\centering
	\begin{tabular}{|l|}
		\hline
		\multicolumn{1}{|c|}{Sköldpadda} \\ \hline
		x: int \\
		y: int \\
		hastighet: int \\
		riktning: int\\ \hline
		gå():\\
		upp():\\
		ner():\\
		vänster():\\
		höger():\\
		rita():\\ \hline
	\end{tabular}
	
\end{frame}


\begin{frame}
	\frametitle{Klassen godis}
	
	\begin{itemize}
		\item Klassen godis kommer att behöva innehålla följande:
	\end{itemize}
	
	\centering
	
	\begin{tabular}{|l|}
		\hline
		\multicolumn{1}{|c|}{Godis} \\ \hline
		x: int \\
		y: int \\ \hline
		ny\_position():\\
		rita():\\ \hline
	\end{tabular}
	
\end{frame}

\subsection{Turtle}

\begin{frame}[fragile]
	\frametitle{Turtle-specifikt}
	
	\begin{itemize}
		\item För att rita ut på skärmen behöver våra två klasser några \texttt{Turtle}-specifika rader kod
	\end{itemize}
	
	\begin{lstlisting}
# Följande ska ligga i __init__
self.turtle = turtle.Turtle() # Skapar något att ha på skärmen
self.turtle.shape('turtle') # alternativt 'square' i godis
self.turtle.pu() # Lyfter pennan såp att vi inte ritar streck
	\end{lstlisting}
	
	\begin{lstlisting}
def rita(self):
    self.turtle.seth(self.riktning*90) # Behövs inte i Godis
    self.turtle.setpos((self.x, self.y)) # Uppdatera positionen på skärmen
	\end{lstlisting}
	
\end{frame}

\begin{frame}[fragile]
	\frametitle{Turtle-specifikt}
	
	\begin{itemize}
		\item Dessutom behöver vi ha lite mer kod för att programmet ska göra som vi vill
	\end{itemize}
	
	\begin{lstlisting}
import turtle
import time
window = turtle.Screen() # Skapar fönstret
window.tracer(0) # Gör att saker händer direkt
window.listen() # Lyssnar efter knapptryck
window.onkey(sköldis.vänster, 'a') # Om man trycker på a...
...
while True:
    time.sleep(0.2) # Gör att allt inte går för snabbt
    window.update() # Uppdaterar fönstret
    sköldis.move() # Flyttar sköldpaddan
    sköldis.draw() # Ritar sköldpaddan
    candy.draw() # Ritar godiset
	\end{lstlisting}
	
\end{frame}

\begin{frame}
	\frametitle{Övningar}
	
	\begin{enumerate}
		\item Implementera klassen Sköldpadda
		\item Implementera klassen Godis
		\item Skapa en funktion som kollar om sköldpaddan nuddar godiset
		\item Flytta på godiset efter att sköldpaddan ätit upp det.
		\item Lägg till att man inte kan vända sig om 180 grader.
		\item Lägg till en ruta som följer efter den första rutan när man ätit en godis.
	\end{enumerate}
	
\end{frame}

\end{document}