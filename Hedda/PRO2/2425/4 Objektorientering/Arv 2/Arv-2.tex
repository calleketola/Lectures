\documentclass[aspectratio=169]{beamer}

\mode<presentation>

\usepackage[utf8]{inputenc}
\usepackage[T1]{fontenc}	%makes å,ä,ö etc. proper symbols
\usepackage{amsmath}
\usepackage{graphicx}
\usepackage{xcolor}
\usepackage{listings}
\usepackage{multicol}
\usepackage{multirow}
\usepackage{hyperref}
\usepackage[swedish]{babel}

\definecolor{LundaGroen}{RGB}{00,68,71}
\definecolor{StabilaLila}{RGB}{85,19,78}
\definecolor{VarmOrange}{RGB}{237,104,63}

\definecolor{MagnoliaRosa}{RGB}{251,214,209}
\definecolor{LundaHimmel}{RGB}{204,225,225}
\definecolor{LundaLjus}{RGB}{255,242,191}

\usefonttheme{serif}
\usetheme{malmoe}
\setbeamercolor{palette primary}{bg=LundaHimmel, fg=StabilaLila}
\setbeamercolor{palette quaternary}{bg=LundaGroen, fg=MagnoliaRosa}
\setbeamercolor{background canvas}{bg=LundaLjus}
\setbeamercolor{structure}{fg=LundaGroen}

\usepackage[many]{tcolorbox}

\newtcolorbox{cross}{blank,breakable,parbox=false,
  overlay={\draw[red,line width=5pt] (interior.south west)--(interior.north east);
    \draw[red,line width=5pt] (interior.north west)--(interior.south east);}}
    
\newcommand{\code}[1]{\colorbox{white}{\lstinline{#1}}}



\lstset{language=Python} 
\lstset{%language=[LaTeX]Tex,%C++,
    morekeywords={PassOptionsToPackage,selectlanguage,True,False},
    keywordstyle=\color{blue},%\bfseries,
    basicstyle=\small\ttfamily,
    %identifierstyle=\color{NavyBlue},
    commentstyle=\color{red}\ttfamily,
    stringstyle=\color{VarmOrange},
    numbers=left,%
    numberstyle=\scriptsize,%\tiny
    stepnumber=1,
    numbersep=8pt,
    showstringspaces=false,
    breaklines=true,
    %frameround=ftff,
    frame=single,
    belowcaptionskip=.75\baselineskip,
	tabsize=4,
	backgroundcolor=\color{white}
    %frame=L
}
\lstset{
	escapeinside={(*@}{@*)}
}


\begin{document}

\lstset{literate=
  {á}{{\'a}}1 {é}{{\'e}}1 {í}{{\'i}}1 {ó}{{\'o}}1 {ú}{{\'u}}1
  {Á}{{\'A}}1 {É}{{\'E}}1 {Í}{{\'I}}1 {Ó}{{\'O}}1 {Ú}{{\'U}}1
  {à}{{\`a}}1 {è}{{\`e}}1 {ì}{{\`i}}1 {ò}{{\`o}}1 {ù}{{\`u}}1
  {À}{{\`A}}1 {È}{{\'E}}1 {Ì}{{\`I}}1 {Ò}{{\`O}}1 {Ù}{{\`U}}1
  {ä}{{\"a}}1 {ë}{{\"e}}1 {ï}{{\"i}}1 {ö}{{\"o}}1 {ü}{{\"u}}1
  {Ä}{{\"A}}1 {Ë}{{\"E}}1 {Ï}{{\"I}}1 {Ö}{{\"O}}1 {Ü}{{\"U}}1
  {â}{{\^a}}1 {ê}{{\^e}}1 {î}{{\^i}}1 {ô}{{\^o}}1 {û}{{\^u}}1
  {Â}{{\^A}}1 {Ê}{{\^E}}1 {Î}{{\^I}}1 {Ô}{{\^O}}1 {Û}{{\^U}}1
  {œ}{{\oe}}1 {Œ}{{\OE}}1 {æ}{{\ae}}1 {Æ}{{\AE}}1 {ß}{{\ss}}1
  {ű}{{\H{u}}}1 {Ű}{{\H{U}}}1 {ő}{{\H{o}}}1 {Ő}{{\H{O}}}1
  {ç}{{\c c}}1 {Ç}{{\c C}}1 {ø}{{\o}}1 {å}{{\r a}}1 {Å}{{\r A}}1
  {€}{{\euro}}1 {£}{{\pounds}}1 {«}{{\guillemotleft}}1
  {»}{{\guillemotright}}1 {ñ}{{\~n}}1 {Ñ}{{\~N}}1 {¿}{{?`}}1
}

\AtBeginSection[ ]
{
\begin{frame}{Innehåll}
    	\tableofcontents[currentsection]
\end{frame}
}

\title{Arv 2}
\date{vt 24}

\maketitle

\begin{frame}
	\frametitle{Övning}
	
	\begin{itemize}
		\item Skapa klasserna \texttt{Person, Teacher, Student} och \texttt{Course}
		\item Klasserna \texttt{Teacher} och \texttt{Student} ska ärva från \texttt{Person}
		\item Till din hjälp har du klassdiagram på följande sidor
		\item Målet är att göra ett program som kan lägga till elever till kurser och ge dem beytg.
		\item Man ska även kunna räkna ut elevernas merit
		\item Sist i bland sidorna hittar du även hur olika utskrifter ska se ut
	\end{itemize}
\end{frame}

\begin{frame}[fragile]
	
	
	\begin{multicols}{2}
		\begin{tabular}{|l|}
			\hline
			Person \\ \hline
			name: str\\
			birth\_year: int \\ \hline
			\_\_str\_\_(): str\\
			\_\_repr\_\_():str\\ \hline
		\end{tabular}
		
		\begin{tabular}{|l|}
			\hline
			Course\\ \hline
			name: str\\
			points: int\\
			students: [Student]\\
			teachers: [Teachers]\\ \hline
			add\_student(Student): void\\
			add\_teacher(Teacher): void\\
			set\_student\_grade(Student, str)\\
			\_\_repr\_\_(): str\\ \hline
		\end{tabular}
		
	\end{multicols}
	
\end{frame}

\begin{frame}

	\begin{multicols}{2}
		
		\begin{tabular}{|l|}
			\hline
			Teacher(Person) \\ \hline
			name: str\\
			birth\_year: int\\
			school: str\\
			subjects: [str]\\ \hline
			\_\_str\_\_(): str \\ \hline
		\end{tabular}
		
		\begin{tabular}{|l|}
			\hline
			Student(Person) \\ \hline
			name: str\\
			birth\_year: int\\
			school: str\\
			grades: \{[str,int]\}\\
			group: str \\ \hline
			calculate\_merit(): float\\
			add\_grade(course: str, grade: str, points: int):  void\\
			\_\_str\_\_(): str\\
			\_\_repr\_\_():str\\ \hline
		\end{tabular}
			
	\end{multicols}
	
\end{frame}

\begin{frame}
	\frametitle{Klassdiagram}
	
	\begin{multicols}{2}
		\begin{tabular}{|l|}
			\hline
			School()\\ \hline
			name: str\\
			groups: [Group]\\
			teachers: [Teacher]\\
			courses: [Course]\\
			principal: Person\\ \hline
			count\_students(): int\\
			\hline
		\end{tabular}
		
		\begin{tabular}{|l|}
			\hline
			Group()\\ \hline
			name: str\\
			students: [Student]\\
			mentorrs: [Teacher]\\ \hline
			add\_student(Student): void\\
			remove\_student(Student): void\\ \hline
		\end{tabular}
	\end{multicols}
	
\end{frame}

\begin{frame}[fragile]
	\frametitle{Utskrifter}
	
	\begin{lstlisting}
s = Student("Nisse", 2006, "Spyken", "Na3b")
t = Teacher("Calle", 1991, "Hedda", ["Matematik", "Programmering"])
c = Course("Programmering 1", 100)
c.add_teacher(t)

print(s)
print(t)
print(c)
	\end{lstlisting}
	\begin{lstlisting}
Nisse 18 Spyken Na3b
Calle 33 Hedda Matematik Programmering
Programmering 1 100 (Calle)
	\end{lstlisting}
	
\end{frame}

\begin{frame}[fragile]
	\frametitle{Pseudokod}
	\framesubtitle{Person}
	
	\begin{lstlisting}
class Person():
    func __init__(self, name, year):
        self.name := name
        self.birth_year := year
    func __str__(self):
        return self.name + " " + (2024-self.birth_year)
    func __repr__(self):
        return str(self)
	\end{lstlisting}

\end{frame}

\begin{frame}[fragile]
	\frametitle{Pseudokod}
	\framesubtitle{Course}
	
	\begin{lstlisting}
class Course():
	def __init__(self, name, points):
        self.name := name
        self.points := points
        self.students := []
        self.teachers := []
    def add_student(self, stud):
        if stud not in self.students:
            self.students.append(stud)
    def add_teacher(self, teacher):
        if teacher not in self.teachers:
            self.teachers.append(teacher)
	\end{lstlisting}
	
	Fortsätter på nästa slide

\end{frame}

\begin{frame}[fragile]
	\frametitle{Pseudokod}
	\framesubtitle{Course fortsättning}
	
	\begin{lstlisting}
    def set_student_grade(self, stud, grade):
        for s in self.students:
            if s = stud:
                s.add_grade(self.name, [grade, self.points])

    def __repr__(self):
        out := self.name + " " + str(self.points) + "("
        for t in self.teachers:
            out := out + " "+t.name
        return out +")"
	\end{lstlisting}
\end{frame}

\begin{frame}[fragile]
	\frametitle{Pseudokod}
	\framesubtitle{Teacher}
	
	\begin{lstlisting}
class Teacher(Person):

    def __init__(self, name, birth_year, school, subjects):
        super().__init__(name, birth_year)
        self.school := school
        self.subjects := subjects

    def __str__(self):
        out := super().__str__()
        for sub in self.subjects:
            out := out + " "+sub
        return out
	\end{lstlisting}

\end{frame}

\begin{frame}[fragile]
	\frametitle{Pseudokod}
	\framesubtitle{Student}

	\begin{lstlisting}
class Student(Person):

    def __init__(self, name, birth_year, school, group):
        super().__init__(name, birth_year)
    
        self.school = school
        self.group = group

        self.grades = {}

    def add_grade(self, course, grade):
        self.grades[course] = grade
	\end{lstlisting}

\end{frame}

\begin{frame}[fragile]
	\frametitle{Pseudokod}
	\framesubtitle{Student fortsättning}

	\begin{lstlisting}
    def calculate_merit(self):
        system = {"A": 20, "B": 17.5, "C": 15, "D": 12.5, "E": 10, "F": 0}
        point_sum = 2400
        merit = 0
        for course in self.grades:
            if "Gymnasiearbete" not in course:
                merit += (system[self.grades[course][0]]*self.grades[course][1])/point_sum
        return merit
    def __str__(self):
        out = super().__str__() + " " + self.school + " " + self.group
        return out
	\end{lstlisting}

\end{frame}

\end{document}