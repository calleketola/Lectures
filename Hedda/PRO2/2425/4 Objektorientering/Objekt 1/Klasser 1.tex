\documentclass[aspectratio=169]{beamer}

\mode<presentation>

\usepackage[utf8]{inputenc}
\usepackage[T1]{fontenc}	%makes å,ä,ö etc. proper symbols
\usepackage{amsmath}
\usepackage{graphicx}
\usepackage{xcolor}
\usepackage{listings}
\usepackage{multicol}
\usepackage{hyperref}
\usepackage[swedish]{babel}

\definecolor{LundaGroen}{RGB}{00,68,71}
\definecolor{StabilaLila}{RGB}{85,19,78}
\definecolor{VarmOrange}{RGB}{237,104,63}

\definecolor{MagnoliaRosa}{RGB}{251,214,209}
\definecolor{LundaHimmel}{RGB}{204,225,225}
\definecolor{LundaLjus}{RGB}{255,242,191}

\usefonttheme{serif}
\usetheme{malmoe}
\setbeamercolor{palette primary}{bg=VarmOrange}
\setbeamercolor{palette quaternary}{bg=LundaGroen}
\setbeamercolor{background canvas}{bg=LundaLjus}
\setbeamercolor{structure}{fg=LundaGroen}

\usepackage[many]{tcolorbox}

\newtcolorbox{cross}{blank,breakable,parbox=false,
  overlay={\draw[red,line width=5pt] (interior.south west)--(interior.north east);
    \draw[red,line width=5pt] (interior.north west)--(interior.south east);}}

\lstset{language=[LaTeX]Tex,%C++,
    morekeywords={PassOptionsToPackage,selectlanguage},
    keywordstyle=\color{blue},%\bfseries,
    basicstyle=\small\ttfamily,
    %identifierstyle=\color{NavyBlue},
    commentstyle=\color{red}\ttfamily,
    stringstyle=\color{orange},
    numbers=left,%
    numberstyle=\scriptsize,%\tiny
    stepnumber=1,
    numbersep=8pt,
    showstringspaces=false,
    breaklines=true,
    %frameround=ftff,
    frame=single,
    belowcaptionskip=.75\baselineskip,
	tabsize=4
    %frame=L
}
\lstset{language=Python} 
\lstset{backgroundcolor=\color{white}}

\newcounter{uppgifter}

\begin{document}

\lstset{literate=
  {á}{{\'a}}1 {é}{{\'e}}1 {í}{{\'i}}1 {ó}{{\'o}}1 {ú}{{\'u}}1
  {Á}{{\'A}}1 {É}{{\'E}}1 {Í}{{\'I}}1 {Ó}{{\'O}}1 {Ú}{{\'U}}1
  {à}{{\`a}}1 {è}{{\`e}}1 {ì}{{\`i}}1 {ò}{{\`o}}1 {ù}{{\`u}}1
  {À}{{\`A}}1 {È}{{\'E}}1 {Ì}{{\`I}}1 {Ò}{{\`O}}1 {Ù}{{\`U}}1
  {ä}{{\"a}}1 {ë}{{\"e}}1 {ï}{{\"i}}1 {ö}{{\"o}}1 {ü}{{\"u}}1
  {Ä}{{\"A}}1 {Ë}{{\"E}}1 {Ï}{{\"I}}1 {Ö}{{\"O}}1 {Ü}{{\"U}}1
  {â}{{\^a}}1 {ê}{{\^e}}1 {î}{{\^i}}1 {ô}{{\^o}}1 {û}{{\^u}}1
  {Â}{{\^A}}1 {Ê}{{\^E}}1 {Î}{{\^I}}1 {Ô}{{\^O}}1 {Û}{{\^U}}1
  {œ}{{\oe}}1 {Œ}{{\OE}}1 {æ}{{\ae}}1 {Æ}{{\AE}}1 {ß}{{\ss}}1
  {ű}{{\H{u}}}1 {Ű}{{\H{U}}}1 {ő}{{\H{o}}}1 {Ő}{{\H{O}}}1
  {ç}{{\c c}}1 {Ç}{{\c C}}1 {ø}{{\o}}1 {å}{{\r a}}1 {Å}{{\r A}}1
  {€}{{\euro}}1 {£}{{\pounds}}1 {«}{{\guillemotleft}}1
  {»}{{\guillemotright}}1 {ñ}{{\~n}}1 {Ñ}{{\~N}}1 {¿}{{?`}}1
}

\lstset{escapeinside={(*@}{@*)}}

% NEW COMMANDS
\AtBeginSection[ ]
{
\begin{frame}{Outline}
\setbeamercolor{section in toc shaded}{fg=LundaGroen}
\setbeamercolor{subsection in toc shaded}{fg=black}
    \tableofcontents[currentsection]

\end{frame}
}
\title{Objektorientering}
\date{vt 25}

\maketitle

\section{Objektorientering}

\subsection{Abstraktion av verkligheten}

\begin{frame}
	\frametitle{Abstraktion av verkligheten}
	
	\begin{itemize}
		\item Mycket i verkligheten kan delas upp i olika egenskaper.
		\item En person kan ha en ålder, längd och ett namn
		\item En bil har ett märke, modellnamn, topphastighet, ägare m.m.
		\item I programmering kan vi \textit{modellera} detta med klasser
	\end{itemize}
	
\end{frame}

\subsection{En klass}

\begin{frame}[fragile]
	\frametitle{En klass}
	
	\begin{itemize}
		\item Här är en representation av en person
	\end{itemize}
	
	\begin{lstlisting}
class Person():
    def __init__(self, name, age, length):
        self.name = name
        self.age = age
        self.length = length
        
    def greet(self):
        print(f"Hello my my is {self.name}")
	\end{lstlisting}
	
\end{frame}

\begin{frame}[fragile]
	\frametitle{Klassen person}
	
	\begin{itemize}
		\item Vi kan nu skapa \textit{instanser} av klassen \texttt{Person}
	\end{itemize}
	
	\begin{lstlisting}
calle = Person("Calle", 32, 181)
enrique = Person("Enrique", 57, 175)
	\end{lstlisting}
	
	\begin{itemize}
		\item Båda kan nu också hälsa
	\end{itemize}
	
	\begin{lstlisting}
calle.greet()
enrique.greet()
	\end{lstlisting}
	
\end{frame}

\begin{frame}[fragile]
	\frametitle{Klassen person}
	
	\begin{itemize}
		\item Vi kan komma åt deras egenskaper, eller klassvariabler
	\end{itemize}
	
	\begin{lstlisting}
calle.name
enrique.age
calle.length
	\end{lstlisting}
	
	\begin{itemize}
		\item Notera att när vi använde \texttt{greet()} så har vi paranteser, medan \texttt{name} är utan
		\item Vi kallar \texttt{greet()} för en metod.
	\end{itemize}
	
\end{frame}

\section{class Car}

\subsection{Skapa en klass}

\begin{frame}[fragile]
	\frametitle{Klassen Car}
	\framesubtitle{}
	
	\begin{lstlisting}
class Car():
    def __init__(self, brand, year, color):
        self.brand = brand
        self.year = year
        self.color = color
    def drive(self):
        print(self.brand + ": Kör framåt")
    def honk(self):
        print(self.brand + ": Tut tut!")
    def breaking(self):
        print(self.brand+ ": Bromsar...")
	\end{lstlisting}
	
\end{frame}

\subsection{Begrepp}

\begin{frame}
	\frametitle{Klassen Car}
	\framesubtitle{Begrepp}
	
	\begin{itemize}
		\item \texttt{class Car} är en ny \textit{datatyp} som vi har skapat
		\item \texttt{\_\_init\_\_} är en \textit{konstruktor}
		\item \texttt{def drive(self):} är en \textit{metod}
		\item \texttt{self.brand} är en \textit{instansvariabel}
	\end{itemize}
\end{frame}

\subsection{Skapa en instans}

\begin{frame}[fragile]
	\frametitle{Klassen Car}
	\framesubtitle{Skapa en instans}
	
	\begin{lstlisting}
bil1 = Car("Volvo", 2018, "Vit")
bil2 = Car("BMW", 2005, "Black")

bil1.honk()
bil2.drive()
	\end{lstlisting}
	
	\begin{verbatim}
Volvo: Tut tut!
BMW: Kör framåt
	\end{verbatim}

\end{frame}

\subsection{Klassdiagram}

\begin{frame}
	\frametitle{Klassen Car}
	\framesubtitle{Klassdiagram}
	
	\centering
	\begin{tabular}{|l|}
		\hline
		\multicolumn{1}{|c|}{\textbf{Car}} \\
		\hline
		brand : str\\
		year : int\\
		color : str\\ \hline
		drive() \\
		honk() \\
		breaking() \\ \hline
	\end{tabular}

\end{frame}

\subsection{self}

\begin{frame}[fragile]
	\frametitle{Klassen Car}
	\framesubtitle{self}
	
	Som du märkt inleds varje \textit{metod} med parametern \texttt{self}. Exempelvis \texttt{honk(self)}
	
	\begin{lstlisting}
    def honk(self):
        print(self.brand+": Tut tut!")
	\end{lstlisting}
	
	Men \texttt{self} dyker inte upp i \textit{metodanropet} senare.
	
	\begin{lstlisting}
bil1.honk() # Inget mellan paranteserna
	\end{lstlisting}
	
	Det är för att Python skickar med en \textit{referens} till \textit{instansen} varje gång man anropar en \textit{metod}.

\end{frame}

\begin{frame}[fragile]
	\frametitle{Klassen Car}
	\framesubtitle{self}
	
	Som du säkert också har märkt så står det \texttt{self.} framför alla \textit{instansvariabler}.
	
	\begin{lstlisting}
    def honk(self):
        print(self.brand+": Tut tut!")
	\end{lstlisting}

\end{frame}

\section{Övningar}

\begin{frame}
	\frametitle{Övningar}

	\begin{itemize}
		\item Ladda ner filen \texttt{Klasser 1.py} från classroom.
	\end{itemize}

	\begin{enumerate}
		\item Skapa en ny instans av klassen \texttt{Person} med ditt eget namn, ålder och längd.
		\item Utveckla metoden \texttt{greet} så att den tar emot ett annat namn och skriver ut något i stil med: \texttt{''Hello, XXX, my name is Calle''}
		\item Utveckla den ytterliggare så att den tar emot en annan instans av klassen \texttt{Person} och hälsar på samma sätt.
		\item Skriv av klassen \texttt{Car}.
		\item Lägg till att den har en variabel med namnet \texttt{distance = 0}
		\item Lägg till att metoden \texttt{drive} ökar distance med en godtycklig distans.
		\item Lägg till en metod som skriver ut hur långt bilen har kört.
		\item Gör uppgifterna 13.1--13.3 i boken. s. 235--
	\end{enumerate}
\end{frame}


\end{document}