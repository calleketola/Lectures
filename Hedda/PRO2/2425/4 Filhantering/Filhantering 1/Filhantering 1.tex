\documentclass[aspectratio=169]{beamer}

\mode<presentation>

\usepackage[utf8]{inputenc}
\usepackage[T1]{fontenc}	%makes å,ä,ö etc. proper symbols
\usepackage{amsmath}
\usepackage{graphicx}
\usepackage{xcolor}
\usepackage{listings}
\usepackage{multicol}
\usepackage{hyperref}
\usepackage[os=win]{menukeys} % Lägger till tangenter


\definecolor{LundaGroen}{RGB}{00,68,71}
\definecolor{StabilaLila}{RGB}{85,19,78}
\definecolor{VarmOrange}{RGB}{237,104,63}

\definecolor{MagnoliaRosa}{RGB}{251,214,209}
\definecolor{LundaHimmel}{RGB}{204,225,225}
\definecolor{LundaLjus}{RGB}{255,242,191}

\usefonttheme{serif}
\usetheme{malmoe}
\setbeamercolor{palette primary}{bg=VarmOrange}
\setbeamercolor{palette quaternary}{bg=LundaGroen}
\setbeamercolor{background canvas}{bg=LundaLjus}
\setbeamercolor{structure}{fg=LundaGroen}

\usepackage[many]{tcolorbox}

\newtcolorbox{cross}{blank,breakable,parbox=false,
  overlay={\draw[red,line width=5pt] (interior.south west)--(interior.north east);
    \draw[red,line width=5pt] (interior.north west)--(interior.south east);}}



\lstset{language=Python} 
\lstset{%language=[LaTeX]Tex,%C++,
    morekeywords={PassOptionsToPackage,selectlanguage,True,False},
    keywordstyle=\color{blue},%\bfseries,
    basicstyle=\small\ttfamily,
    %identifierstyle=\color{NavyBlue},
    commentstyle=\color{red}\ttfamily,
    stringstyle=\color{VarmOrange},
    numbers=left,%
    numberstyle=\scriptsize,%\tiny
    stepnumber=1,
    numbersep=8pt,
    showstringspaces=false,
    breaklines=true,
    %frameround=ftff,
    frame=single,
    belowcaptionskip=.75\baselineskip,
	tabsize=4,
	backgroundcolor=\color{white}
    %frame=L
}

\newcommand{\code}[1]{\colorbox{white}{\lstinline{#1}}}


\begin{document}

\lstset{literate=
  {á}{{\'a}}1 {é}{{\'e}}1 {í}{{\'i}}1 {ó}{{\'o}}1 {ú}{{\'u}}1
  {Á}{{\'A}}1 {É}{{\'E}}1 {Í}{{\'I}}1 {Ó}{{\'O}}1 {Ú}{{\'U}}1
  {à}{{\`a}}1 {è}{{\`e}}1 {ì}{{\`i}}1 {ò}{{\`o}}1 {ù}{{\`u}}1
  {À}{{\`A}}1 {È}{{\'E}}1 {Ì}{{\`I}}1 {Ò}{{\`O}}1 {Ù}{{\`U}}1
  {ä}{{\"a}}1 {ë}{{\"e}}1 {ï}{{\"i}}1 {ö}{{\"o}}1 {ü}{{\"u}}1
  {Ä}{{\"A}}1 {Ë}{{\"E}}1 {Ï}{{\"I}}1 {Ö}{{\"O}}1 {Ü}{{\"U}}1
  {â}{{\^a}}1 {ê}{{\^e}}1 {î}{{\^i}}1 {ô}{{\^o}}1 {û}{{\^u}}1
  {Â}{{\^A}}1 {Ê}{{\^E}}1 {Î}{{\^I}}1 {Ô}{{\^O}}1 {Û}{{\^U}}1
  {œ}{{\oe}}1 {Œ}{{\OE}}1 {æ}{{\ae}}1 {Æ}{{\AE}}1 {ß}{{\ss}}1
  {ű}{{\H{u}}}1 {Ű}{{\H{U}}}1 {ő}{{\H{o}}}1 {Ő}{{\H{O}}}1
  {ç}{{\c c}}1 {Ç}{{\c C}}1 {ø}{{\o}}1 {å}{{\r a}}1 {Å}{{\r A}}1
  {€}{{\euro}}1 {£}{{\pounds}}1 {«}{{\guillemotleft}}1
  {»}{{\guillemotright}}1 {ñ}{{\~n}}1 {Ñ}{{\~N}}1 {¿}{{?`}}1
}

\AtBeginSection[ ]
{
\begin{frame}{Outline}
	\begin{multicols}{2}
		\tableofcontents[currentsection]
	\end{multicols}
\end{frame}
}

\lstset{escapeinside={(*@}{@*)}}

\title{Filhantering}
\date{2023/24}
\author{Programmering 2}

\maketitle{}

\section{Läsa från fil}

\subsection{Öppna filer sätt 1 (osäker)}

\begin{frame}[fragile]
	\frametitle{Öppna filer osäkert}

	\begin{lstlisting}
f = open("filnamn")
f.close() 
	\end{lstlisting}
	
	\begin{itemize}
		\item Standardläget är att filen bara ska läsas
		\item Om man vill försäkra sig om att den bara ska läsas kan man ange \code{'r'}
	\end{itemize}
	
	\begin{lstlisting}
f = open('filnamn', 'r')
f.close()
	\end{lstlisting}
	
\end{frame}

\subsection{Öppna filer sätt 2 (säker)}

\begin{frame}[fragile]
	\frametitle{Öppna filer säkert}
	
	\begin{itemize}
		\item Skulle programmet av någon anledning inte nå till raden för att stänga filen riskerar den att hållas öppen
		\item Det kan leda till att den ockuperar en plats i minnet
		\item Det kan också leda till att den hålls låst för andra program
	\end{itemize}
	
	\begin{lstlisting}
with open('filnamn') as f:
    # Kod
# Nu är filen stängd
	\end{lstlisting}
	
	\begin{itemize}
		\item \code{with} stänger automatiskt filen, även om något dumt skulle hända
	\end{itemize}
	
\end{frame}

\subsection{Läsa filen}

\begin{frame}[fragile]
	\frametitle{Läsa filen}
	
	\begin{itemize}
		\item Det finns tre sätt att läsa filen
	\end{itemize}
	
	\begin{enumerate}
		\item \code{f.read()} läser hela filen
		\item \code{f.readline()} läser in nästa rad
		\item \code{f.readlines()} läser in alla rader och sparar i en lista
	\end{enumerate}
	
	\begin{itemize}
		\item I fall ett och tre så är filen ''färdigläst'' och går inte att läsa igen om man inte öppnar filen på nytt.
		\item I fall två så hoppar du ner till nästa rad, och som i de två tidigare fallen så kan du inte backa igen utan att läsa in filen på nytt.
	\end{itemize}	
	
\end{frame}

\begin{frame}[fragile]
	\frametitle{Läsa filen}
	
	\begin{itemize}
		\item Tänk på att allt du läser in tolkas som strängar, precis som \code{input}
		\item Om en rad innehåller flera datapunkter så särskiljer du dem med \code{var.split()}
		\item En del filer särskiljer data med mellanslag och andra med kommatecken eller semikolon
		\item \code{var.split(',')} och \code{var.split(';')} delar upp på komma, respektive semikolon
	\end{itemize}
	
\end{frame}

\subsection{Övningar 1}

\begin{frame}
	\frametitle{Övningar 1}
	
	\begin{enumerate}
		\item Ladda ner filen \texttt{sonnet-xviii.txt} och skriv ut innehållet av filen.
	\end{enumerate}
	
\end{frame}

\section{Skriva till fil}

\subsection{Skriva över}

\begin{frame}[fragile]
	\frametitle{Skriva till fil}
	
	\begin{itemize}
		\item När du vill skriva till en fil behöver du ange antingen \code{'w'} eller \code{'a'}
		\item Anger du \code{'w'} så kommer du att skriva över hela filen
		\item Anger du \code{'a'} så lägger du till i slutet av filen
		\item Om en fil inte finns och du anger \code{'w'} så kommer filen att skapas
	\end{itemize}
	
\end{frame}

\subsection{Övningar}

\begin{frame}
	\frametitle{Övningar 2}
		
	\begin{enumerate}
		\setcounter{enumi}{1}
		\item Ladda ner filen \texttt{teknik.txt}. Använd Python för att lägga till den lärare som saknas.
	\end{enumerate}
	
\end{frame}

\section{Övningar}

\begin{frame}
	
	\begin{enumerate}
		\setcounter{enumi}{2}
		\item Ladda ner filen \texttt{text-1.txt}. Varje rad innehåller ett ord. 
		\item Hur många rader har filen?
		\item Plocka ut det ordet som är längst .
		\item Hur många gånger förekommer bokstaven ''i''?
		\item Vilken är den vanligast förekommande startbokstaven bland alla ord?
		\item Ladda ner filen \texttt{data-1.csv} multiplicera de tre talen på rad och summera de tio produkterna.
	\end{enumerate}
	
\end{frame}



\end{document}