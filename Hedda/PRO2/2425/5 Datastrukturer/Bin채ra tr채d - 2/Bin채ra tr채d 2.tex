\documentclass[aspectratio=169]{beamer}

\mode<presentation>

\usepackage[utf8]{inputenc}
\usepackage[T1]{fontenc}	%makes å,ä,ö etc. proper symbols
\usepackage{amsmath}
\usepackage{graphicx}
\usepackage{xcolor}
\usepackage{listings}
\usepackage{multicol}
\usepackage{hyperref}

\usepackage{tikz}
\usepackage{forest}
\usetikzlibrary{shapes}


\definecolor{LundaGroen}{RGB}{00,68,71}
\definecolor{StabilaLila}{RGB}{85,19,78}
\definecolor{VarmOrange}{RGB}{237,104,63}

\definecolor{MagnoliaRosa}{RGB}{251,214,209}
\definecolor{LundaHimmel}{RGB}{204,225,225}
\definecolor{LundaLjus}{RGB}{255,242,191}

\usefonttheme{serif}
\usetheme{malmoe}
\setbeamercolor{palette primary}{bg=VarmOrange}
\setbeamercolor{palette quaternary}{bg=LundaGroen}
\setbeamercolor{background canvas}{bg=LundaLjus}
\setbeamercolor{structure}{fg=LundaGroen}

\usepackage[many]{tcolorbox}

\newtcolorbox{cross}{blank,breakable,parbox=false,
  overlay={\draw[red,line width=5pt] (interior.south west)--(interior.north east);
    \draw[red,line width=5pt] (interior.north west)--(interior.south east);}}



\lstset{language=Python} 
\lstset{%language=[LaTeX]Tex,%C++,
    morekeywords={PassOptionsToPackage,selectlanguage,True,False},
    keywordstyle=\color{blue},%\bfseries,
    basicstyle=\small\ttfamily,
    %identifierstyle=\color{NavyBlue},
    commentstyle=\color{red}\ttfamily,
    stringstyle=\color{VarmOrange},
    numbers=left,%
    numberstyle=\scriptsize,%\tiny
    stepnumber=1,
    numbersep=8pt,
    showstringspaces=false,
    breaklines=true,
    %frameround=ftff,
    frame=single,
    belowcaptionskip=.75\baselineskip,
	tabsize=4,
	backgroundcolor=\color{white}
    %frame=L
}

\begin{document}

\lstset{literate=
  {á}{{\'a}}1 {é}{{\'e}}1 {í}{{\'i}}1 {ó}{{\'o}}1 {ú}{{\'u}}1
  {Á}{{\'A}}1 {É}{{\'E}}1 {Í}{{\'I}}1 {Ó}{{\'O}}1 {Ú}{{\'U}}1
  {à}{{\`a}}1 {è}{{\`e}}1 {ì}{{\`i}}1 {ò}{{\`o}}1 {ù}{{\`u}}1
  {À}{{\`A}}1 {È}{{\'E}}1 {Ì}{{\`I}}1 {Ò}{{\`O}}1 {Ù}{{\`U}}1
  {ä}{{\"a}}1 {ë}{{\"e}}1 {ï}{{\"i}}1 {ö}{{\"o}}1 {ü}{{\"u}}1
  {Ä}{{\"A}}1 {Ë}{{\"E}}1 {Ï}{{\"I}}1 {Ö}{{\"O}}1 {Ü}{{\"U}}1
  {â}{{\^a}}1 {ê}{{\^e}}1 {î}{{\^i}}1 {ô}{{\^o}}1 {û}{{\^u}}1
  {Â}{{\^A}}1 {Ê}{{\^E}}1 {Î}{{\^I}}1 {Ô}{{\^O}}1 {Û}{{\^U}}1
  {œ}{{\oe}}1 {Œ}{{\OE}}1 {æ}{{\ae}}1 {Æ}{{\AE}}1 {ß}{{\ss}}1
  {ű}{{\H{u}}}1 {Ű}{{\H{U}}}1 {ő}{{\H{o}}}1 {Ő}{{\H{O}}}1
  {ç}{{\c c}}1 {Ç}{{\c C}}1 {ø}{{\o}}1 {å}{{\r a}}1 {Å}{{\r A}}1
  {€}{{\euro}}1 {£}{{\pounds}}1 {«}{{\guillemotleft}}1
  {»}{{\guillemotright}}1 {ñ}{{\~n}}1 {Ñ}{{\~N}}1 {¿}{{?`}}1
}

% NEW COMMANDS
\newcommand{\fortt}{\texttt{for}}
\newcommand{\whilett}{\texttt{while}}
\newcommand{\iftt}{\texttt{if}}

\AtBeginSection[ ]
{
\begin{frame}{Outline}
    \tableofcontents[currentsection]
\end{frame}
}

\title{Binära träd}
\date{vt 24}
\author{Programmering 2}

\maketitle

\tableofcontents

\section{Repetition}



\begin{frame}
	\frametitle{Binära träd}

	\begin{itemize}
		\item Ett binärt träd är ett träd där varje nod har högst två barn
	\end{itemize}
	
	\centering
	\begin{forest}
		[8
			[4
				[2
					[1]
					[3]
				]
				[6
					[5]
					[7]
				]
			]
			[12
				[10
					[9]
					[11]
				]
				[14
					[13]
					[15]
				]
			]
		]
	\end{forest}

\end{frame}

\begin{frame}
	\frametitle{Binära träd}
	\framesubtitle{Balanserade träd}
	
	\begin{itemize}
		\item Träd kan vara balanserade, eller obalanserade
	\end{itemize}
	
	\begin{center}
		\begin{tabular}{cc}
		\begin{forest}
			[4
				[2
					[1]
					[3]
				]
				[6
					[5]
					[7]
				]
			]
		\end{forest}
		&
		\begin{forest}
			[5
				[2
					[1]
					[3
						[4]
					]
				]
				[6
					[7]
				]
			]
		\end{forest}
		\end{tabular}
	\end{center}
	
\end{frame}

\begin{frame}
	\frametitle{Binära träd}
	\framesubtitle{Att hitta i ett binärt träd}
	
	\begin{itemize}
		\item När du ska hitta i ett binärt träd så börjar du med den översta noden.
		\item Om det är elementet du letar efter är du klar
		\item Annars går du letar efter ett större tal och till vänster om ditt tal är mindre
		\item Den här processen upprepas tills du har hittat rätt.
	\end{itemize}
	
	Tidskomplexitet för att hitta rätt plats i ett binärt träd är \(O(log_2{(n)})\) (i en länkad lista är tidskomplexiteten \(O(n)\)), du behöver alltså göra ungefär tre kontroller om det finns åtta element i listan (\(2^3=8\)) och bara tio kontroller om det finns 1000 element i listan (\(2^{10}=1024\)).
	
\end{frame}

\section{Gå igenom ett träd}

\begin{frame}
	\frametitle{Gå igenom trädet}
	
	Det finns flera olika sätt att gå igenom ett träd och skriva ut värden. De tre vanligaste är:
	
	\begin{itemize}
		\item in order traversal
		\item pre order traversal
		\item post order traversal
	\end{itemize}
	
	Alla tre är så kallade \textit{depth first}.
	
\end{frame}

\subsection{in order traversal}

\begin{frame}[fragile]
	\frametitle{In order traversal}
	
	\begin{itemize}
		\item Säg att du har följande träd:
	\end{itemize}
	
	\centering
	\begin{forest}
		[6
			[4
				[2
					[1]
					[3]
				]
				[5]
			]
			[10
				[8
					[7]
					[9]
				]	
				[12
					[11]
					[13]
				]
			]
		]
	\end{forest}
	
	\begin{itemize}
		\item \textit{In order traversal} hade skrivit ut talen i storleksordning: 1, 2, 3, 4, 5, 6, 7, 8, 9, 10, 11, 12 och 13
	\end{itemize}

\end{frame}

\subsection{Pre order traversal}

\begin{frame}[fragile]
	\frametitle{Pre order traversal}

	\centering
	\begin{forest}
		[6
			[4
				[2
					[1]
					[3]
				]
				[5]
			]
			[10
				[8
					[7]
					[9]
				]	
				[12
					[11]
					[13]
				]
			]
		]
	\end{forest}
	
	\begin{itemize}
		\item \textit{Pre order traversal} hade skrivit ut talen i följande ordning: 6, 4, 2, 1, 3, 5, 10, 8, 7, 9, 12, 11, 13
		\item Detta är användbart om du vill kopiera ett träd. 
	\end{itemize}

\end{frame}

\subsection{Post order traversal}

\begin{frame}[fragile]
	\frametitle{Post order traversal}
	
	\centering
	\begin{forest}
		[6
			[4
				[2
					[1]
					[3]
				]
				[5]
			]
			[10
				[8
					[7]
					[9]
				]	
				[12
					[11]
					[13]
				]
			]
		]
	\end{forest}
	
	\begin{itemize}
		\item \textit{Post order traversal} hade skrivit ut talen i följande ordning: 1, 3, 2, 5, 4, 7, 9, 8, 11, 13, 12, 10, 6
		\item Detta är användbart om du vill radera alla värden i ett träd.
	\end{itemize}

\end{frame}

\section{Övningar}

\begin{frame}
	\frametitle{Övningar}
	
	\begin{enumerate}
		\item Utgå från koden från tidigare lektion och lägg till funktioner för att printa ut trädet
		\item Skapa \texttt{show\_in\_order}
		\item Skapa \texttt{show\_pre\_order}
		\item Skapa \texttt{show\_post\_order}
	\end{enumerate}
	
\end{frame}

\end{document}