\documentclass[aspectratio=169]{beamer}

\mode<presentation>

\usepackage[utf8]{inputenc}
\usepackage[T1]{fontenc}	%makes å,ä,ö etc. proper symbols
\usepackage{amsmath}
\usepackage{graphicx}
\usepackage{xcolor}
\usepackage{listings}
\usepackage{multicol}
\usepackage{hyperref}
\usepackage[os=win]{menukeys} % Lägger till tangenter


\definecolor{LundaGroen}{RGB}{00,68,71}
\definecolor{StabilaLila}{RGB}{85,19,78}
\definecolor{VarmOrange}{RGB}{237,104,63}

\definecolor{MagnoliaRosa}{RGB}{251,214,209}
\definecolor{LundaHimmel}{RGB}{204,225,225}
\definecolor{LundaLjus}{RGB}{255,242,191}

\usefonttheme{serif}
\usetheme{malmoe}
\setbeamercolor{palette primary}{bg=VarmOrange}
\setbeamercolor{palette quaternary}{bg=LundaGroen}
\setbeamercolor{background canvas}{bg=LundaLjus}
\setbeamercolor{structure}{fg=LundaGroen}

\usepackage[many]{tcolorbox}

\newtcolorbox{cross}{blank,breakable,parbox=false,
  overlay={\draw[red,line width=5pt] (interior.south west)--(interior.north east);
    \draw[red,line width=5pt] (interior.north west)--(interior.south east);}}



\lstset{language=Python} 
\lstset{%language=[LaTeX]Tex,%C++,
    morekeywords={PassOptionsToPackage,selectlanguage,True,False},
    keywordstyle=\color{blue},%\bfseries,
    basicstyle=\small\ttfamily,
    %identifierstyle=\color{NavyBlue},
    commentstyle=\color{red}\ttfamily,
    stringstyle=\color{VarmOrange},
    numbers=left,%
    numberstyle=\scriptsize,%\tiny
    stepnumber=1,
    numbersep=8pt,
    showstringspaces=false,
    breaklines=true,
    %frameround=ftff,
    frame=single,
    belowcaptionskip=.75\baselineskip,
	tabsize=4,
	backgroundcolor=\color{white}
    %frame=L
}


\begin{document}

\lstset{literate=
  {á}{{\'a}}1 {é}{{\'e}}1 {í}{{\'i}}1 {ó}{{\'o}}1 {ú}{{\'u}}1
  {Á}{{\'A}}1 {É}{{\'E}}1 {Í}{{\'I}}1 {Ó}{{\'O}}1 {Ú}{{\'U}}1
  {à}{{\`a}}1 {è}{{\`e}}1 {ì}{{\`i}}1 {ò}{{\`o}}1 {ù}{{\`u}}1
  {À}{{\`A}}1 {È}{{\'E}}1 {Ì}{{\`I}}1 {Ò}{{\`O}}1 {Ù}{{\`U}}1
  {ä}{{\"a}}1 {ë}{{\"e}}1 {ï}{{\"i}}1 {ö}{{\"o}}1 {ü}{{\"u}}1
  {Ä}{{\"A}}1 {Ë}{{\"E}}1 {Ï}{{\"I}}1 {Ö}{{\"O}}1 {Ü}{{\"U}}1
  {â}{{\^a}}1 {ê}{{\^e}}1 {î}{{\^i}}1 {ô}{{\^o}}1 {û}{{\^u}}1
  {Â}{{\^A}}1 {Ê}{{\^E}}1 {Î}{{\^I}}1 {Ô}{{\^O}}1 {Û}{{\^U}}1
  {œ}{{\oe}}1 {Œ}{{\OE}}1 {æ}{{\ae}}1 {Æ}{{\AE}}1 {ß}{{\ss}}1
  {ű}{{\H{u}}}1 {Ű}{{\H{U}}}1 {ő}{{\H{o}}}1 {Ő}{{\H{O}}}1
  {ç}{{\c c}}1 {Ç}{{\c C}}1 {ø}{{\o}}1 {å}{{\r a}}1 {Å}{{\r A}}1
  {€}{{\euro}}1 {£}{{\pounds}}1 {«}{{\guillemotleft}}1
  {»}{{\guillemotright}}1 {ñ}{{\~n}}1 {Ñ}{{\~N}}1 {¿}{{?`}}1
}

\AtBeginSection[ ]
{
\begin{frame}{Outline}
	\begin{multicols}{2}
		\tableofcontents[currentsection]
	\end{multicols}
\end{frame}
}

\lstset{escapeinside={(*@}{@*)}}

\title{Felhantering}
\date{2024/25}
\author{Programmering 2}

\maketitle{}


\section{Exceptions}

\begin{frame}
	\frametitle{Exceptions}
	
	Vilka \textit{Exceptions} finns det?
	
	\begin{itemize}
		\item \texttt{EOFError}
		\item \texttt{KeyboardInterrupt}
		\item \texttt{ValueError}
		\item \texttt{ZeroDivisionError}
		\item \texttt{TypeError}
		\item \texttt{IndexError}
		\item \texttt{NameError}
		\item \texttt{UnboundLocalError}
	\end{itemize}
	
	För att hitta alla inbyggda exceptions kan du klicka \href{https://docs.python.org/3/library/exceptions.html}{här}.
	
\end{frame}

\begin{frame}[fragile]
	\frametitle{Kod som genererar errors}

	\begin{itemize}
		\item Vilka fel kan vi få i den här koden?
	\end{itemize}

	\begin{lstlisting}
tal = []
for i in range(5):
    tal.append(int(input(f"Tal {i+1}")))
gissning = int(input("Vilken plats har det största talet? "))
if tal[gissning] == max(tal):
    print("Bra gjort!")
else:
    print("n00b")
	\end{lstlisting}

\end{frame}

\begin{frame}
	\frametitle{Möjliga fel}
	
	Koden på förra sliden kunde ge följande fel:

	\begin{itemize}
		\item \texttt{ValueError}
		\item \texttt{IndexError}
	\end{itemize}

\end{frame}

\begin{frame}
	\frametitle{Vanligaste stället där det blir fel}

	\begin{itemize}
		\item Det vanligaste stället där man riskerar fel är vid användarinput
		\item De flesta andra tillfällena något kan gå fel beror oftast på ett kodningsmisstag
		\item Användare är däremot \textit{kreativa} med inmatning
	\end{itemize}

\end{frame}

\section{Stoppa fel}

\begin{frame}[fragile]
	\frametitle{Try och Except}

	\begin{itemize}
		\item Du kan stoppa programmet från att krascha med \lstinline!try! och \lstinline!except!
	\end{itemize}

	\begin{lstlisting}
try:
    tal = int(input("Skriv ett heltal: "))
    print(tal*2, "är dubbelt så stort")
except ValueError:
    print("Du skrev inte ett heltal.")
	\end{lstlisting}

	\begin{itemize}
		\item Du ska alltid eftersträva att skriva vilket fel man stoppar med \lstinline!except!
		\item Du ska också fäörsöka ha så lite kod innanför \lstinline!try!-blocket som möjligt
		\begin{itemize}
			\item Om det kan gå fel på flera stället så ska du helst ha flera \lstinline!try!-block
		\end{itemize}
	\end{itemize}

\end{frame}

\section{Övningar}

\begin{frame}
	\frametitle{Övningar}

	\begin{itemize}
		\item Utgå från filen \texttt{felhantering - repetition.py} på Classroom och gör övningarna i den.
	\end{itemize}

\end{frame}


\end{document}