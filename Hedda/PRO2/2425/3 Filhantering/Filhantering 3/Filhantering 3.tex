\documentclass[aspectratio=169]{beamer}

\mode<presentation>

\usepackage[utf8]{inputenc}
\usepackage[T1]{fontenc}	%makes å,ä,ö etc. proper symbols
\usepackage{amsmath}
\usepackage{graphicx}
\usepackage{xcolor}
\usepackage{listings}
\usepackage{multicol}
\usepackage{hyperref}
\usepackage[os=win]{menukeys} % Lägger till tangenter


\definecolor{LundaGroen}{RGB}{00,68,71}
\definecolor{StabilaLila}{RGB}{85,19,78}
\definecolor{VarmOrange}{RGB}{237,104,63}

\definecolor{MagnoliaRosa}{RGB}{251,214,209}
\definecolor{LundaHimmel}{RGB}{204,225,225}
\definecolor{LundaLjus}{RGB}{255,242,191}

\usefonttheme{serif}
\usetheme{malmoe}
\setbeamercolor{palette primary}{bg=VarmOrange}
\setbeamercolor{palette quaternary}{bg=LundaGroen}
\setbeamercolor{background canvas}{bg=LundaLjus}
\setbeamercolor{structure}{fg=LundaGroen}

\usepackage[many]{tcolorbox}

\newtcolorbox{cross}{blank,breakable,parbox=false,
  overlay={\draw[red,line width=5pt] (interior.south west)--(interior.north east);
    \draw[red,line width=5pt] (interior.north west)--(interior.south east);}}



\lstset{language=Python} 
\lstset{%language=[LaTeX]Tex,%C++,
    	morekeywords={PassOptionsToPackage,selectlanguage,True,False,with},
    	keywordstyle=\color{blue},%\bfseries,
    	basicstyle=\small\ttfamily,
    	%identifierstyle=\color{red},
    	commentstyle=\color{red}\ttfamily,
    	stringstyle=\color{VarmOrange},
    	numbers=left,%
    	numberstyle=\scriptsize,%\tiny
    	stepnumber=1,
    	numbersep=8pt,
    	showstringspaces=false,
    	breaklines=true,
    	%frameround=ftff,
    	frame=single,
    	belowcaptionskip=.75\baselineskip,
	tabsize=4,
	backgroundcolor=\color{white}
    %frame=L
}

\newcommand{\code}[1]{\colorbox{white}{\lstinline{#1}}}

\begin{document}

\lstset{literate=
  {á}{{\'a}}1 {é}{{\'e}}1 {í}{{\'i}}1 {ó}{{\'o}}1 {ú}{{\'u}}1
  {Á}{{\'A}}1 {É}{{\'E}}1 {Í}{{\'I}}1 {Ó}{{\'O}}1 {Ú}{{\'U}}1
  {à}{{\`a}}1 {è}{{\`e}}1 {ì}{{\`i}}1 {ò}{{\`o}}1 {ù}{{\`u}}1
  {À}{{\`A}}1 {È}{{\'E}}1 {Ì}{{\`I}}1 {Ò}{{\`O}}1 {Ù}{{\`U}}1
  {ä}{{\"a}}1 {ë}{{\"e}}1 {ï}{{\"i}}1 {ö}{{\"o}}1 {ü}{{\"u}}1
  {Ä}{{\"A}}1 {Ë}{{\"E}}1 {Ï}{{\"I}}1 {Ö}{{\"O}}1 {Ü}{{\"U}}1
  {â}{{\^a}}1 {ê}{{\^e}}1 {î}{{\^i}}1 {ô}{{\^o}}1 {û}{{\^u}}1
  {Â}{{\^A}}1 {Ê}{{\^E}}1 {Î}{{\^I}}1 {Ô}{{\^O}}1 {Û}{{\^U}}1
  {œ}{{\oe}}1 {Œ}{{\OE}}1 {æ}{{\ae}}1 {Æ}{{\AE}}1 {ß}{{\ss}}1
  {ű}{{\H{u}}}1 {Ű}{{\H{U}}}1 {ő}{{\H{o}}}1 {Ő}{{\H{O}}}1
  {ç}{{\c c}}1 {Ç}{{\c C}}1 {ø}{{\o}}1 {å}{{\r a}}1 {Å}{{\r A}}1
  {€}{{\euro}}1 {£}{{\pounds}}1 {«}{{\guillemotleft}}1
  {»}{{\guillemotright}}1 {ñ}{{\~n}}1 {Ñ}{{\~N}}1 {¿}{{?`}}1
}

\lstset{escapeinside={(*@}{@*)}}

% NEW COMMANDS
\newcommand{\fortt}{\texttt{for}}
\newcommand{\whilett}{\texttt{while}}
\newcommand{\iftt}{\texttt{if}}

\title{Filhantering}
\date{2023/24}
\author{Programmering 2}

\maketitle


\section{Övningar}

\subsection{Blad 1}

\begin{frame}
	\frametitle{Övningar}
	\framesubtitle{Blad 1}

	\begin{enumerate}
		\item Läs in filen \texttt{filhantering-3-a.txt}
		\item Vilket är det största talet i filen?
		\item Vilket tal är närmast noll?
		\item Vad är medelvärdet av talen?
		\item Vad är medianen?
	\end{enumerate}

\end{frame}

\subsection{Blad 2}

\begin{frame}
	\frametitle{Övningar}
	\framesubtitle{Blad 2}
	
	\begin{enumerate}
		\setcounter{enumi}{5}
		\item Läs in filen \texttt{filhantering-3-b.txt}
		\item Skriv ut innehållet i filen.
		\item Verket saknar fem verser. Lägg till dem (tillsammans med det du har) i en ny fil som du döper till \texttt{the-nomad.txt}.
		\item Vilket ord är vanligast förekommande i hela texten?
		\item Vilken rad i hela texten är längst?
		\item Hur många rader slutar på bokstaven \texttt{n}?
	\end{enumerate}

\end{frame}

\subsection{Blad 3}

\begin{frame}[fragile]
	\frametitle{Övningar}
	\framesubtitle{Blad 3}
	
	\begin{enumerate}
		\setcounter{enumi}{11}
		\item Läs in filen \texttt{filhantering-3-c.txt}
		\item Använd funktionen \code{least_square} för att hitta den linjära funktionen som bäst passar till datan
		\item Vad är \(y\)-värdet då \(x\) är 461?
		\item Vad borde \(y\)-värdet vara då \(x\) är 5?
		\item Vad borde \(y\)-värdet vara då \(x\) är 6123?
	\end{enumerate}
	
\end{frame}

\end{document}