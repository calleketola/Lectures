\documentclass[aspectratio=169]{beamer}

\mode<presentation>

\usepackage[utf8]{inputenc}
\usepackage[T1]{fontenc}	%makes å,ä,ö etc. proper symbols
\usepackage{amsmath}
\usepackage{graphicx}
\usepackage{xcolor}
\usepackage{listings}
\usepackage{multicol}
\usepackage{hyperref}
\usepackage[os=win]{menukeys} % Lägger till tangenter


\definecolor{LundaGroen}{RGB}{00,68,71}
\definecolor{StabilaLila}{RGB}{85,19,78}
\definecolor{VarmOrange}{RGB}{237,104,63}

\definecolor{MagnoliaRosa}{RGB}{251,214,209}
\definecolor{LundaHimmel}{RGB}{204,225,225}
\definecolor{LundaLjus}{RGB}{255,242,191}

\usefonttheme{serif}
\usetheme{malmoe}
\setbeamercolor{palette primary}{bg=VarmOrange}
\setbeamercolor{palette quaternary}{bg=LundaGroen}
\setbeamercolor{background canvas}{bg=LundaLjus}
\setbeamercolor{structure}{fg=LundaGroen}

\usepackage[many]{tcolorbox}

\newtcolorbox{cross}{blank,breakable,parbox=false,
  overlay={\draw[red,line width=5pt] (interior.south west)--(interior.north east);
    \draw[red,line width=5pt] (interior.north west)--(interior.south east);}}



\lstset{language=Python} 
\lstset{%language=[LaTeX]Tex,%C++,
    	morekeywords={PassOptionsToPackage,selectlanguage,True,False,with},
    	keywordstyle=\color{blue},%\bfseries,
    	basicstyle=\small\ttfamily,
    	%identifierstyle=\color{red},
    	commentstyle=\color{red}\ttfamily,
    	stringstyle=\color{VarmOrange},
    	numbers=left,%
    	numberstyle=\scriptsize,%\tiny
    	stepnumber=1,
    	numbersep=8pt,
    	showstringspaces=false,
    	breaklines=true,
    	%frameround=ftff,
    	frame=single,
    	belowcaptionskip=.75\baselineskip,
	tabsize=4,
	backgroundcolor=\color{white}
    %frame=L
}

\newcommand{\code}[1]{\colorbox{white}{\lstinline{#1}}}

\begin{document}

\lstset{literate=
  {á}{{\'a}}1 {é}{{\'e}}1 {í}{{\'i}}1 {ó}{{\'o}}1 {ú}{{\'u}}1
  {Á}{{\'A}}1 {É}{{\'E}}1 {Í}{{\'I}}1 {Ó}{{\'O}}1 {Ú}{{\'U}}1
  {à}{{\`a}}1 {è}{{\`e}}1 {ì}{{\`i}}1 {ò}{{\`o}}1 {ù}{{\`u}}1
  {À}{{\`A}}1 {È}{{\'E}}1 {Ì}{{\`I}}1 {Ò}{{\`O}}1 {Ù}{{\`U}}1
  {ä}{{\"a}}1 {ë}{{\"e}}1 {ï}{{\"i}}1 {ö}{{\"o}}1 {ü}{{\"u}}1
  {Ä}{{\"A}}1 {Ë}{{\"E}}1 {Ï}{{\"I}}1 {Ö}{{\"O}}1 {Ü}{{\"U}}1
  {â}{{\^a}}1 {ê}{{\^e}}1 {î}{{\^i}}1 {ô}{{\^o}}1 {û}{{\^u}}1
  {Â}{{\^A}}1 {Ê}{{\^E}}1 {Î}{{\^I}}1 {Ô}{{\^O}}1 {Û}{{\^U}}1
  {œ}{{\oe}}1 {Œ}{{\OE}}1 {æ}{{\ae}}1 {Æ}{{\AE}}1 {ß}{{\ss}}1
  {ű}{{\H{u}}}1 {Ű}{{\H{U}}}1 {ő}{{\H{o}}}1 {Ő}{{\H{O}}}1
  {ç}{{\c c}}1 {Ç}{{\c C}}1 {ø}{{\o}}1 {å}{{\r a}}1 {Å}{{\r A}}1
  {€}{{\euro}}1 {£}{{\pounds}}1 {«}{{\guillemotleft}}1
  {»}{{\guillemotright}}1 {ñ}{{\~n}}1 {Ñ}{{\~N}}1 {¿}{{?`}}1
}

\lstset{escapeinside={(*@}{@*)}}

% NEW COMMANDS
\newcommand{\fortt}{\texttt{for}}
\newcommand{\whilett}{\texttt{while}}
\newcommand{\iftt}{\texttt{if}}

\title{Filhantering}
\date{2023/24}
\author{Programmering 2}

\maketitle

\section{Repetition}

	\subsection{Öppna filer}

	\begin{frame}[fragile]
		\frametitle{Öppna fil}
		\framesubtitle{open() \& close()}
		
		\begin{itemize}
			\item Vill vi öppna en fil i Python använder vi kommandona \texttt{open()} och \texttt{close()}
		\end{itemize}
		
		\begin{lstlisting}
with open("filnamn") as f:
	# Gör saker
		\end{lstlisting}
		
		\begin{itemize}
			\item Alla filer vi öppnar måste vi stänga. Med \code{with}-blocket stängs filen automatiskt.
			\item Tänk på att filnamnet inkluderar filändelsen.
		\end{itemize}
		
	\end{frame}
	
	\subsection{Läsa in från fil}
	
	\begin{frame}[fragile]
		\frametitle{Läsa fil}
		\framesubtitle{read(), readline(), readlines()}
		
		\begin{itemize}
			\item Om vi vill läsa in innehållet i filen:
		\end{itemize}
		
		\begin{lstlisting}
with open("filnamn") as f:
	text = f.read() # Läser in hela
		\end{lstlisting}
		
		\begin{lstlisting}
with open("filnamn") as f:
	text = f.readline() # Läser in en rad
		\end{lstlisting}
		
		\begin{lstlisting}
with open("filnamn") as f:
	text = f.readlines() # Läser in alla rader
		\end{lstlisting}
		
	\end{frame}

	\subsection{Skriva till fil}
	
	\begin{frame}[fragile]
		\frametitle{Skriva till fil}
		\framesubtitle{write()}
		
		\begin{itemize}
			\item Om vi vill skriva till en fil måste vi öppna den på rätt sätt:
		\end{itemize}
		
		\begin{lstlisting}
with open("filnamn", 'w') as f:# w står för write
	f.write("Hejsan") # Skriver in Hejsan i filen
		\end{lstlisting}
		
		\begin{itemize}
			\item När man använder \code{write} får man vara försiktig eftersom det skriver över filen.
			\item Vill man lägga till i slutet av filen får man istället ange \code{'a'}:
		\end{itemize}
		
		\begin{lstlisting}
with open("filnamn", 'a') as f: # a står för append
	f.write("Hejsan") # Skriver in Hejsan i filen
		\end{lstlisting}
	
	\end{frame}
	
	\subsection{Read and Write}
	
	\begin{frame}[fragile]
		\frametitle{Läsa och skriva}
		\framesubtitle{r+, w+}
		
		\begin{itemize}
			\item Om man vill kunna både läsa och skriva till en fil får man ange \code{'r+'} eller \code{'w+'} som mode:
		\end{itemize}
		
		\begin{lstlisting}
with open("filnamn", 'r+') as f:
	# Do stuff
		\end{lstlisting}
		
	\end{frame}

\section{Specialgrejor}

	\subsection{Encoding}
	
	\begin{frame}[fragile]
		\frametitle{Encoding}
		\framesubtitle{utf-8}
		
		\begin{itemize}
			\item Vill man vara säker på att trevliga symboler som å, ä och ö kan läsas in korrekt så behöver man ange att filen är encodad i utf-8 (eller motsvarande).
		\end{itemize}
		
		\begin{lstlisting}
with open("filnamn", 'r', encoding="utf-8") as f:
	text = f.read() # Vi kan få med å, ä, ö nu :)
		\end{lstlisting}
	
	\end{frame}
	

\section{Övningar}

\begin{frame}
\frametitle{Övningar}

Utgå från filen \texttt{filhantering 2.py}

\begin{enumerate}
	\item Läs in datan från filen ''filhantering2.csv''
	\item Filen innehåller elevers resultat på ett prov. Betygsgränserna för godkända betyg var: 50 (Acceptable), 70 (Exceed Expectations) och 90 (Outstanding). Skriv ut alla med godkända betyg och vad de fick.
	\item Alla under 50 har fått underkänt. Där gick gränserna på allt under 20 (Troll), 40 (Dreadful), 50 (Poor). Skriv ut vilka som fick vilka underkända betyg.
	\item Få ditt program att skriva ut den som hade högst poäng.
	\item Beräkna medelvärdet av deras resultat.
	\item Vad är medianen?
	\item Skriv ut hur många som fick varje betyg.
\end{enumerate}

\end{frame}

\end{document}