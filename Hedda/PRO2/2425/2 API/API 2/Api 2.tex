\documentclass[aspectratio=169]{beamer}

\mode<presentation>

\usepackage[utf8]{inputenc}
\usepackage[T1]{fontenc}	%makes å,ä,ö etc. proper symbols
\usepackage{amsmath}
\usepackage{graphicx}
\usepackage{xcolor}
\usepackage{listings}
\usepackage{multicol}
\usepackage{hyperref}

\definecolor{LundaGroen}{RGB}{00,68,71}
\definecolor{StabilaLila}{RGB}{85,19,78}
\definecolor{VarmOrange}{RGB}{237,104,63}

\definecolor{MagnoliaRosa}{RGB}{251,214,209}
\definecolor{LundaHimmel}{RGB}{204,225,225}
\definecolor{LundaLjus}{RGB}{255,242,191}

\usefonttheme{serif}
\usetheme{malmoe}
\setbeamercolor{palette primary}{bg=VarmOrange}
\setbeamercolor{palette quaternary}{bg=LundaGroen}
\setbeamercolor{background canvas}{bg=LundaLjus}
\setbeamercolor{structure}{fg=LundaGroen}

\usepackage[many]{tcolorbox}

\newtcolorbox{cross}{blank,breakable,parbox=false,
  overlay={\draw[red,line width=5pt] (interior.south west)--(interior.north east);
    \draw[red,line width=5pt] (interior.north west)--(interior.south east);}}

\lstset{language=[LaTeX]Tex,%C++,
    morekeywords={PassOptionsToPackage,selectlanguage},
    keywordstyle=\color{blue},%\bfseries,
    basicstyle=\small\ttfamily,
    %identifierstyle=\color{NavyBlue},
    commentstyle=\color{red}\ttfamily,
    stringstyle=\color{orange},
    numbers=left,%
    numberstyle=\scriptsize,%\tiny
    stepnumber=1,
    numbersep=8pt,
    showstringspaces=false,
    breaklines=true,
    %frameround=ftff,
    frame=single,
    belowcaptionskip=.75\baselineskip,
	tabsize=4
    %frame=L
}
\lstset{language=Python} 
\lstset{backgroundcolor=\color{white}}

\newcounter{uppgifter}

\begin{document}

\lstset{literate=
  {á}{{\'a}}1 {é}{{\'e}}1 {í}{{\'i}}1 {ó}{{\'o}}1 {ú}{{\'u}}1
  {Á}{{\'A}}1 {É}{{\'E}}1 {Í}{{\'I}}1 {Ó}{{\'O}}1 {Ú}{{\'U}}1
  {à}{{\`a}}1 {è}{{\`e}}1 {ì}{{\`i}}1 {ò}{{\`o}}1 {ù}{{\`u}}1
  {À}{{\`A}}1 {È}{{\'E}}1 {Ì}{{\`I}}1 {Ò}{{\`O}}1 {Ù}{{\`U}}1
  {ä}{{\"a}}1 {ë}{{\"e}}1 {ï}{{\"i}}1 {ö}{{\"o}}1 {ü}{{\"u}}1
  {Ä}{{\"A}}1 {Ë}{{\"E}}1 {Ï}{{\"I}}1 {Ö}{{\"O}}1 {Ü}{{\"U}}1
  {â}{{\^a}}1 {ê}{{\^e}}1 {î}{{\^i}}1 {ô}{{\^o}}1 {û}{{\^u}}1
  {Â}{{\^A}}1 {Ê}{{\^E}}1 {Î}{{\^I}}1 {Ô}{{\^O}}1 {Û}{{\^U}}1
  {œ}{{\oe}}1 {Œ}{{\OE}}1 {æ}{{\ae}}1 {Æ}{{\AE}}1 {ß}{{\ss}}1
  {ű}{{\H{u}}}1 {Ű}{{\H{U}}}1 {ő}{{\H{o}}}1 {Ő}{{\H{O}}}1
  {ç}{{\c c}}1 {Ç}{{\c C}}1 {ø}{{\o}}1 {å}{{\r a}}1 {Å}{{\r A}}1
  {€}{{\euro}}1 {£}{{\pounds}}1 {«}{{\guillemotleft}}1
  {»}{{\guillemotright}}1 {ñ}{{\~n}}1 {Ñ}{{\~N}}1 {¿}{{?`}}1
}

\lstset{escapeinside={(*@}{@*)}}

% NEW COMMANDS
\AtBeginSection[ ]
{
\begin{frame}{Outline}
\setbeamercolor{section in toc shaded}{fg=LundaGroen}
\setbeamercolor{subsection in toc shaded}{fg=black}
    \tableofcontents[currentsection]

\end{frame}
}

\title{API}
\date{ht 24}
\author{Programmering 2}

\maketitle

\section{Repetition}

\subsection{Använda requests}

\begin{frame}[fragile]
\frametitle{Repetition}
\framesubtitle{Använda requests}

\begin{lstlisting}
import requests

request = requests.get(url)
innehåll = request.text # Läser in datan som rå text
data = request.json() # Läser in datan som ett JSON-objekt
\end{lstlisting}

\end{frame}

\subsection{JSON-objekt}

\begin{frame}[fragile]
\frametitle{Repetition}
\framesubtitle{JSON-objekt}

Ofta när vi får data så måste vi kolla på hur den ser ut innan vi kan använda den.

Ta exempelvis datan från Koleda med information från kommuner:

\begin{lstlisting}
{"count": 9, "values": [{"kpi": "U07801", "municipality": "1281", "period": 2011, "values": [{"count": 1, "gender": "T", "status": "", "value": 574.9621911}]}, {"kpi": "U07801", "municipality": "1281", "period": 2012, "values": [{"count": 1, "gender": "T", "status": "", "value": 509.35}]}, {"kpi": "U07801", "municipality": "1281", "period": 2013, "values": [{"count": 1, "gender": "T", "status": "", "value": 545.0}]}, ...]}
\end{lstlisting}

\end{frame}

\subsection{Dictionaries}

\begin{frame}[fragile]
\frametitle{Repetition}
\framesubtitle{Dictionaries}

Ett JSON-objekt översätts i Python till en \texttt{dictionary}. De fungerar typ som listor fast istället för index från 0 så har varje plats en nyckel (eller ett namn om man vill).

\begin{lstlisting}
ordlista = {"volvo": ["PV444", "Amazon", "140", "240", "740", "V70"], "saab": ["92", "95", "900", "9-3"], "koenigsegg": ["CC", "CCR", "CCX", "Hundra", "Agera"]}

print(ordlista["saab"])
=> ["92", "95", "900", "9-3"]
\end{lstlisting}

\end{frame}

\section{Övningar}

\begin{frame}
\frametitle{Övningar}

Utgå från filen \texttt{api 2.py}.

\begin{enumerate}
	\item Utveckla funktionen \texttt{plot\_data\_multiple(x,y)} så att den kan plotta ett godtyckligt antal data-set.
	\item Läs in data från följande kommuner och plotta i samma graf: 1280, 1281, 1282, 1283, 1284, 1285 (\href{https://skr.se/skr/tjanster/kommunerochregioner/faktakommunerochregioner/kommunkoder.2052.html}{klicka här för kommunernas namn}).
	\item Gör nu ett program som visar nya grafer för varje av följande nyckeltal: N45900, N45905, N45912 och N45920.
\end{enumerate}

\end{frame}

\end{document}