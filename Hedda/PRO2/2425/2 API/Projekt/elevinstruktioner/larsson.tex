\section{Linus Larsson}

\subsection{Projektet}

Du ska läsa in data från \textit{Kolada} --- en databas för kommuner och regioner --- och presentera denna med en graf. Du kommer att ha en vecka på dig att genomföra och lämna in projektet.

\subsubsection{API-anrop}

Du anropar Koladas API på följande sätt:

\texttt{http://api.kolada.se/v2/data/municipality/KOMMUN/kpi/KPI}

\noindent där du ersätter \texttt{KOMMUN} med kommunkoden och \texttt{KPI} med kpi-värdet från tabellen längre ner.

\subsection{Bedömning}

För att bedömmas som godkänt ska ditt program uppfylla följande krav:

\begin{itemize}
	\item Du hämtar in all data med API:et.
	\item Du presenterar information i ett korrekt diagram.
	\item Du löser uppgifterna.
	\item Din kod är \textit{okommenterad}.
	\item Du kan förklara din kod muntligt för mig.
\end{itemize}

\subsection{Data}

Du ska presentera följande data från Karlstad kommun (kommunkod 1780):

\begin{center}
	\begin{tabular}{ll}
		\rowcolor{blue!25}
		\textbf{Nyckeltal} & \textbf{KPI}\\
		röster på centerpartiet i senaste riksdagsvalet, andel (\%) & N65841\\
		röster på kristdemokraterna i senaste riksdagsvalet, andel (\%) & N65842
	\end{tabular}
\end{center}

Datan ska presenteras med ett linjediagram. Dessutom ska du plotta summan av deras andelar röster i ett tredje linjediagram.

\clearpage