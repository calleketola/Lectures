\documentclass[aspectratio=169]{beamer}

\mode<presentation>

\usepackage[utf8]{inputenc}
\usepackage[T1]{fontenc}	%makes å,ä,ö etc. proper symbols
\usepackage{amsmath}
\usepackage{graphicx}
\usepackage{xcolor}
\usepackage{listings}
\usepackage{multicol}
\usepackage{hyperref}

\definecolor{LundaGroen}{RGB}{00,68,71}
\definecolor{StabilaLila}{RGB}{85,19,78}
\definecolor{VarmOrange}{RGB}{237,104,63}

\definecolor{MagnoliaRosa}{RGB}{251,214,209}
\definecolor{LundaHimmel}{RGB}{204,225,225}
\definecolor{LundaLjus}{RGB}{255,242,191}

\usefonttheme{serif}
\usetheme{malmoe}
\setbeamercolor{palette primary}{bg=VarmOrange}
\setbeamercolor{palette quaternary}{bg=LundaGroen}
\setbeamercolor{background canvas}{bg=LundaLjus}
\setbeamercolor{structure}{fg=LundaGroen}

\usepackage[many]{tcolorbox}

\newtcolorbox{cross}{blank,breakable,parbox=false,
  overlay={\draw[red,line width=5pt] (interior.south west)--(interior.north east);
    \draw[red,line width=5pt] (interior.north west)--(interior.south east);}}

\lstset{language=[LaTeX]Tex,%C++,
    morekeywords={PassOptionsToPackage,selectlanguage},
    keywordstyle=\color{blue},%\bfseries,
    basicstyle=\small\ttfamily,
    %identifierstyle=\color{NavyBlue},
    commentstyle=\color{red}\ttfamily,
    stringstyle=\color{orange},
    numbers=left,%
    numberstyle=\scriptsize,%\tiny
    stepnumber=1,
    numbersep=8pt,
    showstringspaces=false,
    breaklines=true,
    %frameround=ftff,
    frame=single,
    belowcaptionskip=.75\baselineskip,
	tabsize=4
    %frame=L
}
\lstset{language=Python} 
\lstset{backgroundcolor=\color{white}}

\newcounter{uppgifter}

\begin{document}

\lstset{literate=
  {á}{{\'a}}1 {é}{{\'e}}1 {í}{{\'i}}1 {ó}{{\'o}}1 {ú}{{\'u}}1
  {Á}{{\'A}}1 {É}{{\'E}}1 {Í}{{\'I}}1 {Ó}{{\'O}}1 {Ú}{{\'U}}1
  {à}{{\`a}}1 {è}{{\`e}}1 {ì}{{\`i}}1 {ò}{{\`o}}1 {ù}{{\`u}}1
  {À}{{\`A}}1 {È}{{\'E}}1 {Ì}{{\`I}}1 {Ò}{{\`O}}1 {Ù}{{\`U}}1
  {ä}{{\"a}}1 {ë}{{\"e}}1 {ï}{{\"i}}1 {ö}{{\"o}}1 {ü}{{\"u}}1
  {Ä}{{\"A}}1 {Ë}{{\"E}}1 {Ï}{{\"I}}1 {Ö}{{\"O}}1 {Ü}{{\"U}}1
  {â}{{\^a}}1 {ê}{{\^e}}1 {î}{{\^i}}1 {ô}{{\^o}}1 {û}{{\^u}}1
  {Â}{{\^A}}1 {Ê}{{\^E}}1 {Î}{{\^I}}1 {Ô}{{\^O}}1 {Û}{{\^U}}1
  {œ}{{\oe}}1 {Œ}{{\OE}}1 {æ}{{\ae}}1 {Æ}{{\AE}}1 {ß}{{\ss}}1
  {ű}{{\H{u}}}1 {Ű}{{\H{U}}}1 {ő}{{\H{o}}}1 {Ő}{{\H{O}}}1
  {ç}{{\c c}}1 {Ç}{{\c C}}1 {ø}{{\o}}1 {å}{{\r a}}1 {Å}{{\r A}}1
  {€}{{\euro}}1 {£}{{\pounds}}1 {«}{{\guillemotleft}}1
  {»}{{\guillemotright}}1 {ñ}{{\~n}}1 {Ñ}{{\~N}}1 {¿}{{?`}}1
}

\lstset{escapeinside={(*@}{@*)}}

% NEW COMMANDS
\AtBeginSection[ ]
{
\begin{frame}{Outline}
\setbeamercolor{section in toc shaded}{fg=LundaGroen}
\setbeamercolor{subsection in toc shaded}{fg=black}
    \tableofcontents[currentsection]

\end{frame}
}

\title{API}
\date{ht 23}
\author{Programmering 2}

\maketitle

\section{Vad är ett API}

\subsection{Start}

\begin{frame}
\frametitle{Vad är API?}

\begin{itemize}
	\item API står för \textit{Application Programming Interface} och är ett sätt att kommunicera med andra tjänster. Oftast är det andra tjänster på internet.
	\item Antingen vill man skicka information/instruktioner till en tjänst eller så vill man hämta information från en tjänst.
\end{itemize}

\end{frame}

\subsection{Exempel på webbplatser med API}

\begin{frame}
\frametitle{Webbplatser med API}

\begin{itemize}
	\item YouTube
	\item Spotify
	\item Skatteverket
	\item SMHI
\end{itemize}

För att använda en del API:er behöver man ha ett registrerat konto på webbplatsen. Vi kommer att gå igenom hur man gör med ''öppna'' API:er.

\end{frame}

\section{Requests}

\subsection{Installera requests}

\begin{frame}[fragile]
\frametitle{Installera requests}

\texttt{requests} är ett bibliotek som kan hämta och skicka data över nätet. Det är dock inte förinstallerat i Python så du behöver installera det med \texttt{pip}:

Metod 1:
\begin{lstlisting}
pip install requests
\end{lstlisting}

Metod 2:
\begin{lstlisting}
python -m pip install requests
\end{lstlisting}

Metod 3:
\begin{lstlisting}
cd C:\Users\(*@\textcolor{red}{user}@*)\AppData\Local\Programs\Python\Python39\Scripts
pip install requests
\end{lstlisting}

Tänk på att byta ut ''user'' till ditt användarnamn


\end{frame}

\subsection{Get}

\begin{frame}[fragile]
\frametitle{Hämta data}

Grundläggande användning av \texttt{requests}

\begin{lstlisting}
import requests

request = requests.get(url)
\end{lstlisting}

Nu har du laddat ner text-innehållet på platsen \texttt{url}. Vill du läsa den texten kan du skriva:

\begin{lstlisting}
print(request.text)
\end{lstlisting}

Vill du läsa mer om Web Scraping kan du klicka \href{https://realpython.com/python-web-scraping-practical-introduction/}{\textcolor{blue}{\underline{här}}}.

\end{frame}

\begin{frame}[fragile]
\frametitle{Exempel}

\begin{lstlisting}
import requests

request = requests.get('http://numbersapi.com/1337/trivia')
print(request.text)
\end{lstlisting}

\begin{verbatim}
'1337 is the number that spells "leet" in leetspeak.'
\end{verbatim}

\end{frame}

\subsection{JSON}

\begin{frame}[fragile]
\frametitle{Hämta data}
\framesubtitle{Hämta JSON data}

Om du har en URL till en sida med ett JSON-objekt kan du läsa in den så här:

\begin{lstlisting}
import requests
request = requests.get(url)
data = request.json()
\end{lstlisting}

Nu har du läst in ett objekt från nätet och gjort om det till en dictionary.

\end{frame}

\begin{frame}[fragile]
\frametitle{Exempel från Världsbanken}

\begin{lstlisting}
import requests
request = requests.get('https://api.worldbank.org/v2/country/swe?format=json')
data = request.json()
for key in data[1][0]:
  print(data[1][0][key])
\end{lstlisting}

Som du märker så måste man krångla lite med index här. Detta är för att Världsbanken skickar sitt JSON-objekt som en lista som innehåller flera JSON-objekt/dictionaries.

\end{frame}

\section{Övningar}

\begin{frame}
\frametitle{Övningar}

\begin{enumerate}
	\item Installera paketen \texttt{requests} och \texttt{matplotlib}
	
	\item Använd URL:en från exemplet ovan och skriv ut longitud och latitud för Sverige
	\item Ändra URL:en så att du istället hämtar information om Danmark (dk)
	
	Utgå från filen \texttt{api 1.py}.
	
	\item Läs in följande URL som innehåller information om hushållens sopor: \href{https://api.kolada.se/v2/data/municipality/1281/kpi/U07801}{https://api.kolada.se/v2/data/municipality/1281/kpi/U07801}
	\item Skriv ut värdena i formatet: ''årtal: värde''
	\item Läs in motsvarande data från Helsingborg (1283).
	\item Vilken kommun samlade in mest sopor per hushåll 2021?
	\item Hur mycket sopor per hushåll borde samlas in år 2031?
	\item Kan kommunerna följa sina trendkurvor i 50 år?
\end{enumerate}

\end{frame}

\end{document}