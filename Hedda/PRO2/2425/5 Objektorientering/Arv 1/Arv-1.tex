\documentclass[aspectratio=169]{beamer}

\mode<presentation>

\usepackage[utf8]{inputenc}
\usepackage[T1]{fontenc}	%makes å,ä,ö etc. proper symbols
\usepackage{amsmath}
\usepackage{graphicx}
\usepackage{xcolor}
\usepackage{listings}
\usepackage{multicol}
\usepackage{multirow}
\usepackage{hyperref}
\usepackage[swedish]{babel}

\definecolor{LundaGroen}{RGB}{00,68,71}
\definecolor{StabilaLila}{RGB}{85,19,78}
\definecolor{VarmOrange}{RGB}{237,104,63}

\definecolor{MagnoliaRosa}{RGB}{251,214,209}
\definecolor{LundaHimmel}{RGB}{204,225,225}
\definecolor{LundaLjus}{RGB}{255,242,191}

\usefonttheme{serif}
\usetheme{malmoe}
\setbeamercolor{palette primary}{bg=LundaHimmel, fg=StabilaLila}
\setbeamercolor{palette quaternary}{bg=LundaGroen, fg=MagnoliaRosa}
\setbeamercolor{background canvas}{bg=LundaLjus}
\setbeamercolor{structure}{fg=LundaGroen}

\usepackage[many]{tcolorbox}

\newtcolorbox{cross}{blank,breakable,parbox=false,
  overlay={\draw[red,line width=5pt] (interior.south west)--(interior.north east);
    \draw[red,line width=5pt] (interior.north west)--(interior.south east);}}
    
\newcommand{\code}[1]{\colorbox{white}{\lstinline{#1}}}



\lstset{language=Python} 
\lstset{%language=[LaTeX]Tex,%C++,
    morekeywords={PassOptionsToPackage,selectlanguage,True,False},
    keywordstyle=\color{blue},%\bfseries,
    basicstyle=\small\ttfamily,
    %identifierstyle=\color{NavyBlue},
    commentstyle=\color{red}\ttfamily,
    stringstyle=\color{VarmOrange},
    numbers=left,%
    numberstyle=\scriptsize,%\tiny
    stepnumber=1,
    numbersep=8pt,
    showstringspaces=false,
    breaklines=true,
    %frameround=ftff,
    frame=single,
    belowcaptionskip=.75\baselineskip,
	tabsize=4,
	backgroundcolor=\color{white}
    %frame=L
}
\lstset{
	escapeinside={(*@}{@*)}
}


\begin{document}

\lstset{literate=
  {á}{{\'a}}1 {é}{{\'e}}1 {í}{{\'i}}1 {ó}{{\'o}}1 {ú}{{\'u}}1
  {Á}{{\'A}}1 {É}{{\'E}}1 {Í}{{\'I}}1 {Ó}{{\'O}}1 {Ú}{{\'U}}1
  {à}{{\`a}}1 {è}{{\`e}}1 {ì}{{\`i}}1 {ò}{{\`o}}1 {ù}{{\`u}}1
  {À}{{\`A}}1 {È}{{\'E}}1 {Ì}{{\`I}}1 {Ò}{{\`O}}1 {Ù}{{\`U}}1
  {ä}{{\"a}}1 {ë}{{\"e}}1 {ï}{{\"i}}1 {ö}{{\"o}}1 {ü}{{\"u}}1
  {Ä}{{\"A}}1 {Ë}{{\"E}}1 {Ï}{{\"I}}1 {Ö}{{\"O}}1 {Ü}{{\"U}}1
  {â}{{\^a}}1 {ê}{{\^e}}1 {î}{{\^i}}1 {ô}{{\^o}}1 {û}{{\^u}}1
  {Â}{{\^A}}1 {Ê}{{\^E}}1 {Î}{{\^I}}1 {Ô}{{\^O}}1 {Û}{{\^U}}1
  {œ}{{\oe}}1 {Œ}{{\OE}}1 {æ}{{\ae}}1 {Æ}{{\AE}}1 {ß}{{\ss}}1
  {ű}{{\H{u}}}1 {Ű}{{\H{U}}}1 {ő}{{\H{o}}}1 {Ő}{{\H{O}}}1
  {ç}{{\c c}}1 {Ç}{{\c C}}1 {ø}{{\o}}1 {å}{{\r a}}1 {Å}{{\r A}}1
  {€}{{\euro}}1 {£}{{\pounds}}1 {«}{{\guillemotleft}}1
  {»}{{\guillemotright}}1 {ñ}{{\~n}}1 {Ñ}{{\~N}}1 {¿}{{?`}}1
}

\AtBeginSection[ ]
{
\begin{frame}{Innehåll}
    	\tableofcontents[currentsection]
\end{frame}
}

\title{Arv 1}
\date{vt 25}

\maketitle

\section{Arv}

\subsection{Repetition}

\begin{frame}
	\frametitle{Klasserna sen sist}
	
	\centering
	\begin{multicols}{2}
	
	\begin{tabular}{|l|}
		\hline
		\multicolumn{1}{|c|}{Sköldpadda} \\ \hline
		x: int \\
		y: int \\
		färg: str\\
		hastighet: int \\
		riktning: int\\ \hline
		rita():\\
		gå():\\
		upp():\\
		ner():\\
		vänster():\\
		höger():\\ \hline
	\end{tabular}
	
	\begin{tabular}{|l|}
		\hline
		\multicolumn{1}{|c|}{Godis} \\ \hline
		x: int \\
		y: int \\
		färg: str\\ \hline
		rita():\\
		ny\_position():\\ \hline
	\end{tabular}
	
	\end{multicols}
	
	Notera att de har en del gemensamt
	
\end{frame}

\subsection{Gemensamt}

\begin{frame}
	\frametitle{Gemensamt}
	
	\begin{itemize}
		\item Båda klasserna har \textit{attributen} x, y och färg
		\item Båda klasserna har \textit{metoden} rita()
		\item Dessutom har båda klasserna en massa Turtle-specifik kod gemensam
		\item Vi kan \textit{förenkla} våra klasser och \textit{bryta ut} de gemensamma delarna
	\end{itemize}
	
\end{frame}

\subsection{Arv}

\begin{frame}
	\frametitle{Klassen objekt}
	\framesubtitle{Klassdiagram}
	
	\centering
	\begin{tabular}{|l|}
		\hline
		\multicolumn{1}{|c|}{Objekt} \\ \hline
		x: int \\
		y: int \\ 
		färg: str \\\hline
		rita():\\ \hline
	\end{tabular}
	
\end{frame}

\begin{frame}[fragile]
	\frametitle{Klassen objekt}
	\framesubtitle{Implementation}
	
	\begin{lstlisting}
class Objekt():

    def __init__(self, färg):
        self.x = 0
        self.y = 0
        self.färg = färg
        self.turtle = turtle.Turtle()
        self.turtle.pu()
        
    def rita(self):
        self.turtle.setpos((self.x, self.y))
	\end{lstlisting}
	
\end{frame}

\begin{frame}[fragile]
	\frametitle{Sköldpadda som ärver objekt}
	
	Nu kan vi koppla samman klassen Sköldpadda med klassen Objekt.
	
	\begin{lstlisting}
class Sköldpadda(Objekt):
    def __init__(self):
        super().__init__() # Kör init från Objekt
        
        self.turtle.shape('turtle') # Detta var inte gemensamt
	\end{lstlisting}
	
	Nu kommer Sköldpadda ha samma attribut och metoder som klassen Objekt.
	
\end{frame}

\begin{frame}[fragile]
	\frametitle{Godis som ärver objekt}
	
	Vi kan också koppla klassen Godis till Objekt:
	
	\begin{lstlisting}
class Godis(Objekt):
    def __init__(self):
        super().__init__() # Kör init från Objekt
        
        self.turtle.shape('square') # Detta var inte gemensamt
	\end{lstlisting}

\end{frame}

\section{Övningar}


\begin{frame}
	\frametitle{Övningar}
	
	\begin{enumerate}
		\item Implementera klassen Objekt
		\item Uppdatera klasserna Sköldpadda och Godis så att de ärver av Objekt
		\item Skapa en ny fil med namnet hus.py
		\item Utgå från boken, s. 269--276, och gör uppgifterna 13.13 och 13.14 
	\end{enumerate}
	
\end{frame}



\end{document}