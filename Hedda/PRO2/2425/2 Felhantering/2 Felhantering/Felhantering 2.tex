\documentclass[aspectratio=169]{beamer}

\mode<presentation>

\usepackage[utf8]{inputenc}
\usepackage[T1]{fontenc}	%makes å,ä,ö etc. proper symbols
\usepackage{amsmath}
\usepackage{graphicx}
\usepackage{xcolor}
\usepackage{listings}
\usepackage{multicol}
\usepackage{hyperref}
\usepackage[os=win]{menukeys} % Lägger till tangenter


\definecolor{LundaGroen}{RGB}{00,68,71}
\definecolor{StabilaLila}{RGB}{85,19,78}
\definecolor{VarmOrange}{RGB}{237,104,63}

\definecolor{MagnoliaRosa}{RGB}{251,214,209}
\definecolor{LundaHimmel}{RGB}{204,225,225}
\definecolor{LundaLjus}{RGB}{255,242,191}

\usefonttheme{serif}
\usetheme{malmoe}
\setbeamercolor{palette primary}{bg=VarmOrange}
\setbeamercolor{palette quaternary}{bg=LundaGroen}
\setbeamercolor{background canvas}{bg=LundaLjus}
\setbeamercolor{structure}{fg=LundaGroen}

\usepackage[many]{tcolorbox}

\newtcolorbox{cross}{blank,breakable,parbox=false,
  overlay={\draw[red,line width=5pt] (interior.south west)--(interior.north east);
    \draw[red,line width=5pt] (interior.north west)--(interior.south east);}}



\lstset{language=Python} 
\lstset{%language=[LaTeX]Tex,%C++,
    morekeywords={PassOptionsToPackage,selectlanguage,True,False},
    keywordstyle=\color{blue},%\bfseries,
    basicstyle=\small\ttfamily,
    %identifierstyle=\color{NavyBlue},
    commentstyle=\color{red}\ttfamily,
    stringstyle=\color{VarmOrange},
    numbers=left,%
    numberstyle=\scriptsize,%\tiny
    stepnumber=1,
    numbersep=8pt,
    showstringspaces=false,
    breaklines=true,
    %frameround=ftff,
    frame=single,
    belowcaptionskip=.75\baselineskip,
	tabsize=4,
	backgroundcolor=\color{white}
    %frame=L
}


\begin{document}

\lstset{literate=
  {á}{{\'a}}1 {é}{{\'e}}1 {í}{{\'i}}1 {ó}{{\'o}}1 {ú}{{\'u}}1
  {Á}{{\'A}}1 {É}{{\'E}}1 {Í}{{\'I}}1 {Ó}{{\'O}}1 {Ú}{{\'U}}1
  {à}{{\`a}}1 {è}{{\`e}}1 {ì}{{\`i}}1 {ò}{{\`o}}1 {ù}{{\`u}}1
  {À}{{\`A}}1 {È}{{\'E}}1 {Ì}{{\`I}}1 {Ò}{{\`O}}1 {Ù}{{\`U}}1
  {ä}{{\"a}}1 {ë}{{\"e}}1 {ï}{{\"i}}1 {ö}{{\"o}}1 {ü}{{\"u}}1
  {Ä}{{\"A}}1 {Ë}{{\"E}}1 {Ï}{{\"I}}1 {Ö}{{\"O}}1 {Ü}{{\"U}}1
  {â}{{\^a}}1 {ê}{{\^e}}1 {î}{{\^i}}1 {ô}{{\^o}}1 {û}{{\^u}}1
  {Â}{{\^A}}1 {Ê}{{\^E}}1 {Î}{{\^I}}1 {Ô}{{\^O}}1 {Û}{{\^U}}1
  {œ}{{\oe}}1 {Œ}{{\OE}}1 {æ}{{\ae}}1 {Æ}{{\AE}}1 {ß}{{\ss}}1
  {ű}{{\H{u}}}1 {Ű}{{\H{U}}}1 {ő}{{\H{o}}}1 {Ő}{{\H{O}}}1
  {ç}{{\c c}}1 {Ç}{{\c C}}1 {ø}{{\o}}1 {å}{{\r a}}1 {Å}{{\r A}}1
  {€}{{\euro}}1 {£}{{\pounds}}1 {«}{{\guillemotleft}}1
  {»}{{\guillemotright}}1 {ñ}{{\~n}}1 {Ñ}{{\~N}}1 {¿}{{?`}}1
}

\AtBeginSection[ ]
{
\begin{frame}{Outline}
	\begin{multicols}{2}
		\tableofcontents[currentsection]
	\end{multicols}
\end{frame}
}

\lstset{escapeinside={(*@}{@*)}}

\title{Felhantering}
\date{2024/25}
\author{Programmering 2}

\maketitle{}

\section{Try \& Except}

\subsection{Repetition}

\begin{frame}[fragile]
	\frametitle{Try \& Except}
	\framesubtitle{Repetition}
	
	\begin{lstlisting}
while True:
    tal = int(input("Ange ett heltal: "))
    svar = tal**2
    print("Kvadraten på " + str(tal) + " är " + str(svar))
	\end{lstlisting}
	
\end{frame}

\begin{frame}[fragile]
	\frametitle{Try \& Except}
	\framesubtitle{Repetition}
	
	\begin{lstlisting}
while True:
    try:
        tal = int(input("Ange ett heltal: "))
        svar = tal**2
        print("Kvadraten på " + str(tal) + " är " + str(svar))
    except(EOFError):
        print("Programmet avbröts")
        quit()
    except(KeyboardInterrupt):
        print("Du försökte döda programmet")
	\end{lstlisting}
	
\end{frame}

\begin{frame}
	\frametitle{Try \& Except}
	\framesubtitle{Repetition}
	
	Vi har garderat oss mot två \textit{undantag} (fel). Vilka fler kan vi räkna med?
	
	\begin{itemize}
		\item \texttt{EOFError}, End Of File
		\item \texttt{KeyboardInterrupt}, ctrl+c
		\item<2-> \texttt{ValueError}, värdefel
	\end{itemize}
\end{frame}

\begin{frame}[fragile]
	\frametitle{Try \& Except}
	\framesubtitle{ValueError}
	
	\begin{lstlisting}
while True:
    try:
        tal = int(input("Ange ett heltal: "))
        svar = tal**2
        print("Kvadraten på " + str(tal) + " är " + str(svar))
    except(EOFError):
        print("Programmet avbröts")
        quit()
    except(KeyboardInterrupt):
        print("Du försökte döda programmet")
    except(ValueError):
        print("Du måste skriva in ett heltal")
	\end{lstlisting}
	
\end{frame}

\subsection{Exceptions}

\begin{frame}
	\frametitle{Try \& Except}
	\framesubtitle{Exceptions}
	
	Vilka \textit{Exceptions} finns det?
	
	\begin{itemize}
		\item \texttt{EOFError}
		\item \texttt{KeyboardInterrupt}
		\item \texttt{ValueError}
		\item<2-> \texttt{ZeroDivisionError}
		\item<2-> \texttt{TypeError}
		\item<2-> \texttt{IndexError}
		\item<2-> \texttt{NameError}
		\item<2-> \texttt{UnboundLocalError}
	\end{itemize}
	
	\onslide<3->För att hitta alla inbyggda exceptions kan du klicka \href{https://docs.python.org/3/library/exceptions.html}{här}.
	
\end{frame}

\section{raise exception}
\subsection{Ett exempel}

\begin{frame}[fragile]
	\frametitle{Raise exception}
	\framesubtitle{Exempel}
	
	\begin{lstlisting}
def dela(a,b):
    return a/b
    
while True:
    tal = input("Skriv två heltal: ")
    tal = tal.split()
    svar = dela(int(tal[0]),int(tal[1]))
    print("Kvoten mellan "+ tal[0] + " och " + tal[1] +  " är " + str(svar))
	\end{lstlisting}
	
	Om vi stoppar in talen 5 och 0 får vi \texttt{ZeroDivisionError}.
	
\end{frame}


\begin{frame}[fragile]
	\frametitle{Assert}
	\framesubtitle{Exempel}
	
	Eftersom vi vet att funktionen inte fungerar om vi skickar med en nolla kan vi förebygga ett fel på två sätt:

	\begin{lstlisting}
def dela(a,b):
    assert b != 0, "Du får inte dela med noll"
    return a/b
    
while True:
    tal = input("Skriv två heltal: ")
    tal = tal.split()
    svar = dela(int(tal[0]),int(tal[1]))
    print("Kvoten mellan "+ tal[0] + " och " + tal[1] +  " är " + str(svar))
	\end{lstlisting}
	
\end{frame}

\begin{frame}[fragile]
	\frametitle{raise}
	\framesubtitle{Exempel}
	
	Eftersom vi vet att funktionen inte fungerar om vi skickar med en nolla kan vi förebygga ett fel på två sätt:

	\begin{lstlisting}
def dela(a,b):
    if b == 0:
        raise ZeroDivisionError("Du får inte dela med noll")
    return a/b
    
while True:
    tal = input("Skriv två heltal: ")
    tal = tal.split()
    svar = dela(int(tal[0]),int(tal[1]))
    print("Kvoten mellan "+ tal[0] + " och " + tal[1] +  " är " + str(svar))
	\end{lstlisting}
	
\end{frame}

\subsection{assert vs. raise}

\begin{frame}[fragile]
	\frametitle{raise}
	\framesubtitle{assert vs. raise}
	
	\texttt{assert} ska användas för att kontrollera koden \textit{under utveckling}. Den ska hitta fel som utvecklaren gör. Om ett fel kan uppstå på grund av en slarvig användare så ska det istället täckas av \texttt{raise exception}. En anledning till detta är att när man skeppar en färdig produkt så stänger man av \textit{debug}-läget som skriver ut errors.
	
	Testa skillnaden med ett program med \texttt{assert} i kommandotolken
	
	\begin{lstlisting}
python "mitt program.py"
python -O "mitt program.py"
	\end{lstlisting}
	
\end{frame}

\subsection{Använda raise exception}

\begin{frame}[fragile]
	\frametitle{raise}
	\framesubtitle{Förutse fel}
	
	I exemplet från tidigare:
	
	\begin{lstlisting}
def dela(a,b):
    if b == 0:
        raise ZeroDivisionError("Du får inte dela med noll")
    return a/b
    \end{lstlisting}
    
    Så är vi förutseende. Vi förutser att användaren kommer att göra fel och vi tar upp det \textbf{innan} det kan leda till problem längre ner i koden. Säg att vi har en stor funktion där divisionen med noll hade kommit sent, då är det fortfarande god sed att \textit{fånga} den direkt. Varför?
	
\end{frame}

\subsection{Fel som inte är fel men ändå fel}

\begin{frame}
	\frametitle{raise}
	\framesubtitle{Fel som inte är fel men ändå fel}
	
	Ibland täcker inte de undantag som finns något som händer i ens kod.
	
	Exempelvis i Sänka skepp, om man skrev ett negativt tal.
	
	Eller så har man ett program där man tar emot input där första tecknet måste vara en \#.
	
\end{frame}


\begin{frame}[fragile]
	\frametitle{raise}
	\framesubtitle{Fel som inte är fel men ändå fel}

	Sänka skepp:
	
	\begin{lstlisting}
def fråga_rad():
rad = input("Skriv en rad: ")
    rad = int(rad)
    if rad < 1 or rad > 11:
        raise ValueError("Ett tal måste vara mellan 1 och 10")
    return rad
	\end{lstlisting}
	
\end{frame}

\begin{frame}[fragile]
	\frametitle{raise}
	\framesubtitle{Fel som inte är fel men ändå fel}
	
	\begin{lstlisting}
färgkod = input("Skriv en färgkod: ")
if färgkod[0] != "#":
    raise ValueError("Alla färgkoder börjar med #")
    \end{lstlisting}
	
\end{frame}

\section{Övningar}

\begin{frame}
\frametitle{Övningar}

Utgå från filen \texttt{felhantering 2.py}

\begin{enumerate}
	\item Kontrollera att du förstår vad koden gör i varje steg.
	\item Se till att funktionen \texttt{dela} inte kraschar
	\item Se till att inget i filen gör att den kraschar
	\item Se till att funktionen \texttt{list\_delare} inte accepterar listor som har färre än två element
	\item Se till att funktionen \texttt{färgväljare} lyfter ett felmeddelande när man anger en annan än de åtta färgerna som finns inprogrammerade
\end{enumerate}

\end{frame}

\end{document}